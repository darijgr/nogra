\documentclass[numbers=enddot,12pt,final,onecolumn,notitlepage]{scrartcl}%
\usepackage[headsepline,footsepline,manualmark]{scrlayer-scrpage}
\usepackage[all,cmtip]{xy}
\usepackage{amssymb}
\usepackage{amsmath}
\usepackage{amsthm}
\usepackage{framed}
\usepackage{comment}
\usepackage{color}
\usepackage{hyperref}
\usepackage[sc]{mathpazo}
\usepackage[T1]{fontenc}
\usepackage{tikz}
\usepackage{needspace}
\usepackage{tabls}
\usepackage{wasysym}
%TCIDATA{OutputFilter=latex2.dll}
%TCIDATA{Version=5.50.0.2960}
%TCIDATA{LastRevised=Tuesday, May 10, 2022 13:24:39}
%TCIDATA{SuppressPackageManagement}
%TCIDATA{<META NAME="GraphicsSave" CONTENT="32">}
%TCIDATA{<META NAME="SaveForMode" CONTENT="1">}
%TCIDATA{BibliographyScheme=Manual}
%TCIDATA{Language=American English}
%BeginMSIPreambleData
\providecommand{\U}[1]{\protect\rule{.1in}{.1in}}
%EndMSIPreambleData
\usetikzlibrary{arrows.meta}
\usetikzlibrary{chains}
\newcounter{exer}
\numberwithin{exer}{subsection}
\theoremstyle{definition}
\newtheorem{theo}{Theorem}[subsection]
\newenvironment{theorem}[1][]
{\begin{theo}[#1]\begin{leftbar}}
{\end{leftbar}\end{theo}}
\newtheorem{lem}[theo]{Lemma}
\newenvironment{lemma}[1][]
{\begin{lem}[#1]\begin{leftbar}}
{\end{leftbar}\end{lem}}
\newtheorem{prop}[theo]{Proposition}
\newenvironment{proposition}[1][]
{\begin{prop}[#1]\begin{leftbar}}
{\end{leftbar}\end{prop}}
\newtheorem{defi}[theo]{Definition}
\newenvironment{definition}[1][]
{\begin{defi}[#1]\begin{leftbar}}
{\end{leftbar}\end{defi}}
\newtheorem{remk}[theo]{Remark}
\newenvironment{remark}[1][]
{\begin{remk}[#1]\begin{leftbar}}
{\end{leftbar}\end{remk}}
\newtheorem{coro}[theo]{Corollary}
\newenvironment{corollary}[1][]
{\begin{coro}[#1]\begin{leftbar}}
{\end{leftbar}\end{coro}}
\newtheorem{conv}[theo]{Convention}
\newenvironment{convention}[1][]
{\begin{conv}[#1]\begin{leftbar}}
{\end{leftbar}\end{conv}}
\newtheorem{quest}[theo]{Question}
\newenvironment{question}[1][]
{\begin{quest}[#1]\begin{leftbar}}
{\end{leftbar}\end{quest}}
\newtheorem{warn}[theo]{Warning}
\newenvironment{conclusion}[1][]
{\begin{warn}[#1]\begin{leftbar}}
{\end{leftbar}\end{warn}}
\newtheorem{conj}[theo]{Conjecture}
\newenvironment{conjecture}[1][]
{\begin{conj}[#1]\begin{leftbar}}
{\end{leftbar}\end{conj}}
\newtheorem{exam}[theo]{Example}
\newenvironment{example}[1][]
{\begin{exam}[#1]\begin{leftbar}}
{\end{leftbar}\end{exam}}
\newtheorem{exmp}[exer]{Exercise}
\newenvironment{exercise}[1][]
{\begin{exmp}[#1]\begin{leftbar}}
{\end{leftbar}\end{exmp}}
\newenvironment{statement}{\begin{quote}}{\end{quote}}
\newenvironment{fineprint}{\begin{small}}{\end{small}}
\iffalse
\newenvironment{proof}[1][Proof]{\noindent\textbf{#1.} }{\ \rule{0.5em}{0.5em}}
\newenvironment{question}[1][Question]{\noindent\textbf{#1.} }{\ \rule{0.5em}{0.5em}}
\newenvironment{teachingnote}[1][Teaching note]{\noindent\textbf{#1.} }{\ \rule{0.5em}{0.5em}}
\fi
\let\sumnonlimits\sum
\let\prodnonlimits\prod
\let\cupnonlimits\bigcup
\let\capnonlimits\bigcap
\renewcommand{\sum}{\sumnonlimits\limits}
\renewcommand{\prod}{\prodnonlimits\limits}
\renewcommand{\bigcup}{\cupnonlimits\limits}
\renewcommand{\bigcap}{\capnonlimits\limits}
\setlength\tablinesep{3pt}
\setlength\arraylinesep{3pt}
\setlength\extrarulesep{3pt}
\voffset=0cm
\hoffset=-0.7cm
\setlength\textheight{22.5cm}
\setlength\textwidth{15.5cm}
\newcommand\arxiv[1]{\href{http://www.arxiv.org/abs/#1}{\texttt{arXiv:#1}}}
\newenvironment{verlong}{}{}
\newenvironment{vershort}{}{}
\newenvironment{noncompile}{}{}
\newenvironment{teachingnote}{}{}
\excludecomment{verlong}
\includecomment{vershort}
\excludecomment{noncompile}
\excludecomment{teachingnote}
\newcommand{\CC}{\mathbb{C}}
\newcommand{\RR}{\mathbb{R}}
\newcommand{\QQ}{\mathbb{Q}}
\newcommand{\NN}{\mathbb{N}}
\newcommand{\ZZ}{\mathbb{Z}}
\newcommand{\KK}{\mathbb{K}}
\newcommand{\id}{\operatorname{id}}
\newcommand{\lcm}{\operatorname{lcm}}
\newcommand{\rev}{\operatorname{rev}}
\newcommand{\powset}[2][]{\ifthenelse{\equal{#2}{}}{\mathcal{P}\left(#1\right)}{\mathcal{P}_{#1}\left(#2\right)}}
\newcommand{\set}[1]{\left\{ #1 \right\}}
\newcommand{\abs}[1]{\left| #1 \right|}
\newcommand{\tup}[1]{\left( #1 \right)}
\newcommand{\ive}[1]{\left[ #1 \right]}
\newcommand{\floor}[1]{\left\lfloor #1 \right\rfloor}
\newcommand{\lf}[2]{#1^{\underline{#2}}}
\newcommand{\underbrack}[2]{\underbrace{#1}_{\substack{#2}}}
\newcommand{\horrule}[1]{\rule{\linewidth}{#1}}
\newcommand{\nnn}{\nonumber\\}
\newcommand{\sslash}{\mathbin{/\mkern-6mu/}}
\ihead{Math 5707 Spring 2017 (Darij Grinberg): Lecture 8}
\ohead{page \thepage}
\cfoot{}
\begin{document}

\title{UMN, Spring 2017, Math 5707: Lecture 8 (Vandermonde determinant using tournaments)}
\author{Darij Grinberg}
\date{digitized and improved version,
%TCIMACRO{\TeXButton{TeX field}{\today}}%
%BeginExpansion
\today
%EndExpansion
}
\maketitle
\tableofcontents

\section{Tournaments and the Vandermonde determinant}

The goal of this lecture is to demonstrate a curious application of digraphs:
a combinatorial proof of the Vandermonde determinant identity. This proof --
which was found by Gessel in 1979 -- is neither the simplest nor the shortest
proof (several others can be found in \cite[\S 6.7]{detnotes} and in most
textbooks on linear algebra), but it illustrates several techniques in
enumerative combinatorics and in the application of combinatorics to other fields.

\subsection{$3$-cycles in tournaments}

We shall use the notations introduced in \cite[\S 1.1 and \S 1.4]{lec7}. In
particular, we use the word \textquotedblleft\emph{digraph}\textquotedblright%
\ as shorthand for \textquotedblleft simple digraph\textquotedblright. When
$i$ and $j$ are two vertices of a digraph, we sometimes use the notation $ij$
for the pair $\left(  i,j\right)  $. We recall that a \emph{tournament} is a
loopless simple digraph $D$ that satisfies the \emph{tournament axiom}: For
any two distinct vertices $u$ and $v$ of $D$, \textbf{exactly} one of the two
pairs $\left(  u,v\right)  =uv$ and $\left(  v,u\right)  =vu$ is an arc of $D$.

We shall focus on counting certain triples of vertices, which we will refer to
as \textquotedblleft$3$-cycles\textquotedblright:

\begin{definition}
Let $D=\left(  V,A\right)  $ be a simple digraph. A $3$\emph{-cycle} of $D$
shall mean a triple $\left(  u,v,w\right)  $ of distinct vertices $u,v,w\in V$
satisfying $uv,vw,wu\in A$.
\end{definition}

\begin{example}
\label{exa.digr.3-cycs.1}Consider the following digraph:%
\[%
%TCIMACRO{\TeXButton{tikz 5-vertex tournament}{\begin{tikzpicture}[scale=1.2]
%\begin{scope}[every node/.style={circle,thick,draw=green!60!black}]
%\node(1) at (0*360/5 : 2) {$1$};
%\node(2) at (1*360/5 : 2) {$2$};
%\node(3) at (2*360/5 : 2) {$3$};
%\node(4) at (3*360/5 : 2) {$4$};
%\node(5) at (4*360/5 : 2) {$5$};
%\end{scope}
%\begin{scope}[every edge/.style={draw=black,very thick}, every loop/.style={}]
%\path[->] (1) edge (2) edge (4) edge (5);
%\path[->] (2) edge (5);
%\path[->] (3) edge (1) edge (2);
%\path[->] (4) edge (2) edge (3) edge (5);
%\path[->] (5) edge (3);
%\end{scope}
%\end{tikzpicture}}}%
%BeginExpansion
\begin{tikzpicture}[scale=1.2]
\begin{scope}[every node/.style={circle,thick,draw=green!60!black}]
\node(1) at (0*360/5 : 2) {$1$};
\node(2) at (1*360/5 : 2) {$2$};
\node(3) at (2*360/5 : 2) {$3$};
\node(4) at (3*360/5 : 2) {$4$};
\node(5) at (4*360/5 : 2) {$5$};
\end{scope}
\begin{scope}[every edge/.style={draw=black,very thick}, every loop/.style={}]
\path[->] (1) edge (2) edge (4) edge (5);
\path[->] (2) edge (5);
\path[->] (3) edge (1) edge (2);
\path[->] (4) edge (2) edge (3) edge (5);
\path[->] (5) edge (3);
\end{scope}
\end{tikzpicture}%
%EndExpansion
\ \ .
\]
This digraph has nine $3$-cycles:%
\begin{align*}
&  \left(  1,4,3\right)  ,\ \ \ \ \ \ \ \ \ \ \left(  1,5,3\right)
,\ \ \ \ \ \ \ \ \ \ \left(  2,5,3\right)  ,\ \ \ \ \ \ \ \ \ \ \left(
3,1,4\right)  ,\ \ \ \ \ \ \ \ \ \ \left(  3,1,5\right)  ,\\
&  \left(  3,2,5\right)  ,\ \ \ \ \ \ \ \ \ \ \left(  4,3,1\right)
,\ \ \ \ \ \ \ \ \ \ \left(  5,3,1\right)  ,\ \ \ \ \ \ \ \ \ \ \left(
5,3,2\right)  .
\end{align*}
(We do not count the two $3$-cycles $\left(  1,4,3\right)  $ and $\left(
4,3,1\right)  $ as identical, even though they are just cyclic rotations of
one another.) On the other hand, the triple $\left(  1,2,3\right)  $ is not a
$3$-cycle (since $23$ is not an arc). The triple $\left(  1,3,4\right)  $ is
not a $3$-cycle either (since none of $13$, $34$ and $41$ is an arc).
\end{example}

We note that our notion of $3$-cycles is essentially equivalent to the notion
of cycles\footnote{See \cite[Definition 1.5.1 \textbf{(b)}]{lec7} for the
definition of our notion of cycles.} of length $3$. Indeed, if $\left(
u,v,w\right)  $ is a $3$-cycle of a simple digraph $D$, then $\left(
u,v,w,u\right)  $ is a cycle of length $3$. Conversely, if $\left(
u,v,w,u\right)  $ is a cycle of length $3$, then $\left(  u,v,w\right)  $ is a
$3$-cycle.

\begin{definition}
Let $D=\left(  V,A\right)  $ be a digraph, and let $\left(  u,v\right)  $ be
an arc of $D$. To \emph{reverse} this arc $\left(  u,v\right)  $ means to
replace this arc $\left(  u,v\right)  $ by $\left(  v,u\right)  $ in the arc
set of $D$. The result of this operation is a new digraph $\left(
V,\ \ \left(  A\setminus\left\{  uv\right\}  \right)  \cup\left\{  vu\right\}
\right)  $.
\end{definition}

We note that if $D$ is a tournament, then the new digraph $\left(
V,\ \ \left(  A\setminus\left\{  uv\right\}  \right)  \cup\left\{  vu\right\}
\right)  $ obtained by reversing the arc $\left(  u,v\right)  $ will again be
a tournament.

\begin{convention}
In the following, the symbol \textquotedblleft$\#$\textquotedblright\ stands
for the word \textquotedblleft number\textquotedblright\ (as in
\textquotedblleft the number of\textquotedblright). For example,
\[
\left(  \#\text{ of subsets of }\left\{  1,2,3\right\}  \right)  =2^{3}=8.
\]

\end{convention}

\begin{proposition}
\label{prop.tour-vand.1}Let $D$ be a tournament. Let $\left(  u,v,w\right)  $
be a $3$-cycle of $D$. Let $D^{\prime}$ be the tournament obtained from $D$ by
reversing the arcs $uv$, $vw$ and $wu$ (this means replacing them by $vu$,
$wv$ and $uw$). Then,%
\[
\left(  \#\text{ of }3\text{-cycles of }D^{\prime}\right)  =\left(  \#\text{
of }3\text{-cycles of }D\right)  .
\]

\end{proposition}

Here is an illustration for Proposition \ref{prop.tour-vand.1} showing $D$ on
the left and $D^{\prime}$ on the right (the arcs $uv$, $vw$ and $wu$ of $D$
are painted blue; the arcs $vu$, $wv$ and $uw$ of $D^{\prime}$ are painted
red; all other arcs are the same in $D$ and in $D^{\prime}$):%
\[%
\begin{tabular}
[c]{|c|c|}\hline
$%
%TCIMACRO{\TeXButton{tikz D}{\begin{tikzpicture}
%\draw[draw=black] (-2.5, -3) rectangle (3, 3);
%\begin{scope}[every node/.style={circle,thick,draw=green!60!black}]
%\node(u) at (0:1.5) {$u$};
%\node(v) at (120:1.5) {$v$};
%\node(w) at (240:1.5) {$w$};
%\end{scope}
%\node(u1) at (35:2.5) {};
%\node(u2) at (15:3) {};
%\node(u3) at (-5:3) {};
%\node(v1) at (155:2.5) {};
%\node(v2) at (125:3) {};
%\node(v3) at (105:3) {};
%\node(w1) at (260:3) {};
%\node(w2) at (220:3) {};
%\begin{scope}[every edge/.style={draw=blue,line width=1.7pt}]
%\path[->] (u) edge (v) (v) edge (w) (w) edge (u);
%\end{scope}
%\begin{scope}[every edge/.style={draw=black,dashed,very thick}]
%\path[->] (u) edge (u1) edge (u2) (u3) edge (u);
%\path[->] (v) edge (v1) edge (v2) edge (v3);
%\path[->] (w2) edge (w) (w1) edge (w);
%\end{scope}
%\end{tikzpicture}}}%
%BeginExpansion
\begin{tikzpicture}
\draw[draw=black] (-2.5, -3) rectangle (3, 3);
\begin{scope}[every node/.style={circle,thick,draw=green!60!black}]
\node(u) at (0:1.5) {$u$};
\node(v) at (120:1.5) {$v$};
\node(w) at (240:1.5) {$w$};
\end{scope}
\node(u1) at (35:2.5) {};
\node(u2) at (15:3) {};
\node(u3) at (-5:3) {};
\node(v1) at (155:2.5) {};
\node(v2) at (125:3) {};
\node(v3) at (105:3) {};
\node(w1) at (260:3) {};
\node(w2) at (220:3) {};
\begin{scope}[every edge/.style={draw=blue,line width=1.7pt}]
\path[->] (u) edge (v) (v) edge (w) (w) edge (u);
\end{scope}
\begin{scope}[every edge/.style={draw=black,dashed,very thick}]
\path[->] (u) edge (u1) edge (u2) (u3) edge (u);
\path[->] (v) edge (v1) edge (v2) edge (v3);
\path[->] (w2) edge (w) (w1) edge (w);
\end{scope}
\end{tikzpicture}%
%EndExpansion
$ & $%
%TCIMACRO{\TeXButton{tikz Dprime}{\begin{tikzpicture}
%\draw[draw=black] (-2.5, -3) rectangle (3, 3);
%\begin{scope}[every node/.style={circle,thick,draw=green!60!black}]
%\node(u) at (0:1.5) {$u$};
%\node(v) at (120:1.5) {$v$};
%\node(w) at (240:1.5) {$w$};
%\end{scope}
%\node(u1) at (35:2.5) {};
%\node(u2) at (15:3) {};
%\node(u3) at (-5:3) {};
%\node(v1) at (155:2.5) {};
%\node(v2) at (125:3) {};
%\node(v3) at (105:3) {};
%\node(w1) at (260:3) {};
%\node(w2) at (220:3) {};
%\begin{scope}[every edge/.style={draw=red,line width=1.7pt}]
%\path[->] (u) edge (w) (w) edge (v) (v) edge (u);
%\end{scope}
%\begin{scope}[every edge/.style={draw=black,dashed,very thick}]
%\path[->] (u) edge (u1) edge (u2) (u3) edge (u);
%\path[->] (v) edge (v1) edge (v2) edge (v3);
%\path[->] (w2) edge (w) (w1) edge (w);
%\end{scope}
%\end{tikzpicture}}}%
%BeginExpansion
\begin{tikzpicture}
\draw[draw=black] (-2.5, -3) rectangle (3, 3);
\begin{scope}[every node/.style={circle,thick,draw=green!60!black}]
\node(u) at (0:1.5) {$u$};
\node(v) at (120:1.5) {$v$};
\node(w) at (240:1.5) {$w$};
\end{scope}
\node(u1) at (35:2.5) {};
\node(u2) at (15:3) {};
\node(u3) at (-5:3) {};
\node(v1) at (155:2.5) {};
\node(v2) at (125:3) {};
\node(v3) at (105:3) {};
\node(w1) at (260:3) {};
\node(w2) at (220:3) {};
\begin{scope}[every edge/.style={draw=red,line width=1.7pt}]
\path[->] (u) edge (w) (w) edge (v) (v) edge (u);
\end{scope}
\begin{scope}[every edge/.style={draw=black,dashed,very thick}]
\path[->] (u) edge (u1) edge (u2) (u3) edge (u);
\path[->] (v) edge (v1) edge (v2) edge (v3);
\path[->] (w2) edge (w) (w1) edge (w);
\end{scope}
\end{tikzpicture}%
%EndExpansion
$\\
$D$ & $D^{\prime}$\\\hline
\end{tabular}
\ \ \ \ \ .
\]


We shall give two proofs of Proposition \ref{prop.tour-vand.1}. The first
relies on the notion of indegrees, which we shall now recall along with that
of outdegrees:

\begin{definition}
\label{def.digr.indeg}Let $D=\left(  V,A\right)  $ be a digraph. Let $v\in V$
be any vertex. Then:

\begin{enumerate}
\item[\textbf{(a)}] The \emph{outdegree} of $v$ denotes the number of arcs of
$D$ whose source is $v$. This outdegree is denoted $\deg^{+}v$.

\item[\textbf{(b)}] The \emph{indegree} of $v$ denotes the number of arcs of
$D$ whose target is $v$. This indegree is denoted $\deg^{-}v$.
\end{enumerate}
\end{definition}

For instance, if $D$ is the digraph in Example \ref{exa.digr.3-cycs.1}, then
$\deg^{+}2=1$ and $\deg^{-}2=3$.

We also recall the following fact (Exercise 5 on homework set \#2; see
\cite{hw2s} for a solution):

\begin{proposition}
\label{prop.hw2e5b}Let $D=\left(  V,A\right)  $ be a tournament. Set
$n=\left\vert V\right\vert $ and $m=\sum_{v\in V}\dbinom{\deg^{-}v}{2}$. Then:

\begin{enumerate}
\item[\textbf{(a)}] We have $m=\sum_{v\in V}\dbinom{\deg^{+}v}{2}$.

\item[\textbf{(b)}] The number of $3$-cycles in $D$ is $3\left(  \dbinom{n}%
{3}-m\right)  $.
\end{enumerate}
\end{proposition}

\begin{proof}
[First proof of Proposition \ref{prop.tour-vand.1}.]Proposition
\ref{prop.hw2e5b} \textbf{(b)} yields that the $\#$ of $3$-cycles in $D$ is
$3\left(  \dbinom{n}{3}-m\right)  $, where $m=\sum_{v\in V}\dbinom{\deg^{-}%
v}{2}$. Thus, the $\#$ of $3$-cycles of $D$ depends only on $V$ and the
indegrees $\deg^{-}v$ of the vertices $v\in V$. But these indegrees do not
change when we reverse the arcs $uv$, $vw$ and $wu$ (since each of the
vertices $u$, $v$ and $w$ loses one incoming arc\footnote{An \textquotedblleft
incoming arc\textquotedblright\ of a vertex $r\in V$ means an arc whose target
is $r$. The number of such arcs is the indegree $\deg^{-}r$ of $r$.} and gains
another). Hence, the $\#$ of $3$-cycles, too, does not change when we reverse
these arcs. This proves Proposition \ref{prop.tour-vand.1}.
\end{proof}

\begin{noncompile}


\begin{proof}
For each subset $W$ of $V$, we let $D\mid_{W}$ denote the digraph%
\[
\left(  W,\ \ A\cap\left(  W\times W\right)  \right)  .
\]
This is the digraph whose vertices are the elements of $W$ and whose arcs are
those arcs of $D$ whose sources and targets belong to $W$. This digraph
$D\mid_{W}$ is called the \emph{induced subdigraph of }$D$ on the set $W$. It
is easy to see that $D\mid_{W}$ is a tournament (since $D$ is a tournament).

It is easy to prove Proposition \ref{prop.tour-vand.1} in the case when
$\left\vert V\right\vert \leq4$ (since there are only finitely many cases to
check). Hence, for each $x\in V$, we have%
\begin{align*}
&  \left(  \#\text{ of }3\text{-cycles of }D^{\prime}\text{ whose vertices
belong to }\left\{  u,v,w,x\right\}  \right) \\
&  =\left(  \#\text{ of }3\text{-cycles of }D\text{ whose vertices belong to
}\left\{  u,v,w,x\right\}  \right)
\end{align*}
(because this is what we obtain if we apply Proposition \ref{prop.tour-vand.1}
to the induced subdigraph $D\mid_{\left\{  u,v,w,x\right\}  }$ instead of $D$).
\end{proof}
\end{noncompile}

\begin{proof}
[Second proof of Proposition \ref{prop.tour-vand.1}.]Write the digraph $D$ in
the form $D=\left(  V,A\right)  $.

The $3$-cycles of $D$ can be classified into the following three types:

\begin{itemize}
\item \textbf{Type 1:} those $3$-cycles that contain \textbf{at most one} of
the vertices $u$, $v$ and $w$.

\item \textbf{Type 2:} those $3$-cycles that contain \textbf{precisely two} of
the vertices $u$, $v$ and $w$.

\item \textbf{Type 3:} those $3$-cycles that contain \textbf{all three} of the
vertices $u$, $v$ and $w$.
\end{itemize}

The $3$-cycles of Type 2 can be classified further: Any $3$-cycle of Type 2
contains precisely two of the vertices $u$, $v$ and $w$ and one further
vertex. If this further vertex is $x$, then we call this $3$-cycle a
\textquotedblleft$3$-cycle of Type 2$_{x}$\textquotedblright. Thus, a
$3$-cycle of Type 2$_{x}$ (for a vertex $x\in V\setminus\left\{
u,v,w\right\}  $) means a $3$-cycle that contains the vertex $x$ and precisely
two of the vertices $u$, $v$ and $w$.

The $3$-cycles of $D^{\prime}$ can be classified in the exact same way.

Now, when we reverse the arcs $uv$, $vw$ and $wu$ of the digraph $D$, all
$3$-cycles of Type 1 are preserved, and no new $3$-cycles of Type 1 are
created. Thus,%
\begin{align}
&  \left(  \#\text{ of }3\text{-cycles of }D^{\prime}\text{ of Type 1}\right)
\nonumber\\
&  =\left(  \#\text{ of }3\text{-cycles of }D\text{ of Type 1}\right)  .
\label{pf.prop.tour-vand.1.2nd.type1}%
\end{align}


Furthermore, for each $x\in V\setminus\left\{  u,v,w\right\}  $, we have%
\begin{align}
&  \left(  \#\text{ of }3\text{-cycles of }D^{\prime}\text{ of Type 2}%
_{x}\right) \nonumber\\
&  =\left(  \#\text{ of }3\text{-cycles of }D\text{ of Type 2}_{x}\right)  .
\label{pf.prop.tour-vand.1.2nd.type2x}%
\end{align}


[\textit{Proof of (\ref{pf.prop.tour-vand.1.2nd.type2x}):} Let $x\in
V\setminus\left\{  u,v,w\right\}  $. We must prove that the two numbers
\[
\left(  \#\text{ of }3\text{-cycles of }D^{\prime}\text{ of Type 2}%
_{x}\right)  \ \ \ \ \ \ \ \ \ \ \text{and}\ \ \ \ \ \ \ \ \ \ \left(
\#\text{ of }3\text{-cycles of }D\text{ of Type 2}_{x}\right)
\]
are equal. These numbers depend only on the presence or absence of the pairs
$\left(  x,u\right)  $, $\left(  x,v\right)  $, $\left(  x,w\right)  $,
$\left(  u,x\right)  $, $\left(  v,x\right)  $ and $\left(  w,x\right)  $ in
the set $A$ (since a $3$-cycle of Type 2$_{x}$ must consist entirely of
vertices from the set $\left\{  u,v,w,x\right\}  $, and thus its existence or
non-existence depends only on the arcs that join the four vertices in this
set); thus, we only have finitely many cases to check. Moreover, since $D$ is
a tournament, it suffices to know which of the three pairs $\left(
x,u\right)  $, $\left(  x,v\right)  $ and $\left(  x,w\right)  $ belong to
$A$, because the tournament axiom shows that (e.g.) the pair $\left(
u,x\right)  $ belongs to $A$ if and only if the pair $\left(  x,u\right)  $
does not. Thus, we have eight cases left to consider:

\textit{Case 1:} We have $\left(  x,u\right)  \in A$ and $\left(  x,v\right)
\in A$ and $\left(  x,w\right)  \in A$ (and thus $\left(  u,x\right)  \notin
A$ and $\left(  v,x\right)  \notin A$ and $\left(  w,x\right)  \notin A$).

\textit{Case 2:} We have $\left(  x,u\right)  \in A$ and $\left(  x,v\right)
\in A$ and $\left(  x,w\right)  \notin A$ (and thus $\left(  u,x\right)
\notin A$ and $\left(  v,x\right)  \notin A$ and $\left(  w,x\right)  \in A$).

\textit{Case 3:} We have $\left(  x,u\right)  \in A$ and $\left(  x,v\right)
\notin A$ and $\left(  x,w\right)  \in A$ (and thus $\left(  u,x\right)
\notin A$ and $\left(  v,x\right)  \in A$ and $\left(  w,x\right)  \notin A$).

\textit{Case 4:} We have $\left(  x,u\right)  \in A$ and $\left(  x,v\right)
\notin A$ and $\left(  x,w\right)  \notin A$ (and thus $\left(  u,x\right)
\notin A$ and $\left(  v,x\right)  \in A$ and $\left(  w,x\right)  \in A$).

\textit{Case 5:} We have $\left(  x,u\right)  \notin A$ and $\left(
x,v\right)  \in A$ and $\left(  x,w\right)  \in A$ (and thus $\left(
u,x\right)  \in A$ and $\left(  v,x\right)  \notin A$ and $\left(  w,x\right)
\notin A$).

\textit{Case 6:} We have $\left(  x,u\right)  \notin A$ and $\left(
x,v\right)  \in A$ and $\left(  x,w\right)  \notin A$ (and thus $\left(
u,x\right)  \in A$ and $\left(  v,x\right)  \notin A$ and $\left(  w,x\right)
\in A$).

\textit{Case 7:} We have $\left(  x,u\right)  \notin A$ and $\left(
x,v\right)  \notin A$ and $\left(  x,w\right)  \in A$ (and thus $\left(
u,x\right)  \in A$ and $\left(  v,x\right)  \in A$ and $\left(  w,x\right)
\notin A$).

\textit{Case 8:} We have $\left(  x,u\right)  \notin A$ and $\left(
x,v\right)  \notin A$ and $\left(  x,w\right)  \notin A$ (and thus $\left(
u,x\right)  \in A$ and $\left(  v,x\right)  \in A$ and $\left(  w,x\right)
\in A$).

All eight cases are straightforward. For example, in Case 5, the relevant
parts of the digraphs $D$ and $D^{\prime}$ (that is, the vertices $u,v,w,x$
and the arcs that join them) look as follows:%
\[%
\begin{tabular}
[c]{|c|c|}\hline
$%
%TCIMACRO{\TeXButton{tikz D}{\begin{tikzpicture}
%\draw[draw=black] (-2.3, -3) rectangle (3.2, 3);
%\begin{scope}[every node/.style={circle,thick,draw=green!60!black}]
%\node(u) at (0:2.5) {$u$};
%\node(v) at (120:2.5) {$v$};
%\node(w) at (240:2.5) {$w$};
%\node(x) at (0,0) {$x$};
%\end{scope}
%\begin{scope}[every edge/.style={draw=blue,line width=1.7pt}]
%\path[->] (u) edge (v) (v) edge (w) (w) edge (u);
%\end{scope}
%\begin{scope}[every edge/.style={draw=black,very thick}]
%\path[->] (u) edge (x) (x) edge (v) (x) edge (w);
%\end{scope}
%\end{tikzpicture}}}%
%BeginExpansion
\begin{tikzpicture}
\draw[draw=black] (-2.3, -3) rectangle (3.2, 3);
\begin{scope}[every node/.style={circle,thick,draw=green!60!black}]
\node(u) at (0:2.5) {$u$};
\node(v) at (120:2.5) {$v$};
\node(w) at (240:2.5) {$w$};
\node(x) at (0,0) {$x$};
\end{scope}
\begin{scope}[every edge/.style={draw=blue,line width=1.7pt}]
\path[->] (u) edge (v) (v) edge (w) (w) edge (u);
\end{scope}
\begin{scope}[every edge/.style={draw=black,very thick}]
\path[->] (u) edge (x) (x) edge (v) (x) edge (w);
\end{scope}
\end{tikzpicture}%
%EndExpansion
$ & $%
%TCIMACRO{\TeXButton{tikz Dprime}{\begin{tikzpicture}
%\draw[draw=black] (-2.3, -3) rectangle (3.2, 3);
%\begin{scope}[every node/.style={circle,thick,draw=green!60!black}]
%\node(u) at (0:2.5) {$u$};
%\node(v) at (120:2.5) {$v$};
%\node(w) at (240:2.5) {$w$};
%\node(x) at (0,0) {$x$};
%\end{scope}
%\begin{scope}[every edge/.style={draw=red,line width=1.7pt}]
%\path[->] (u) edge (w) (w) edge (v) (v) edge (u);
%\end{scope}
%\begin{scope}[every edge/.style={draw=black,very thick}]
%\path[->] (u) edge (x) (x) edge (v) (x) edge (w);
%\end{scope}
%\end{tikzpicture}}}%
%BeginExpansion
\begin{tikzpicture}
\draw[draw=black] (-2.3, -3) rectangle (3.2, 3);
\begin{scope}[every node/.style={circle,thick,draw=green!60!black}]
\node(u) at (0:2.5) {$u$};
\node(v) at (120:2.5) {$v$};
\node(w) at (240:2.5) {$w$};
\node(x) at (0,0) {$x$};
\end{scope}
\begin{scope}[every edge/.style={draw=red,line width=1.7pt}]
\path[->] (u) edge (w) (w) edge (v) (v) edge (u);
\end{scope}
\begin{scope}[every edge/.style={draw=black,very thick}]
\path[->] (u) edge (x) (x) edge (v) (x) edge (w);
\end{scope}
\end{tikzpicture}%
%EndExpansion
$\\
$D$ & $D^{\prime}$\\\hline
\end{tabular}
\ \ \ \ \ \ .
\]
In this case, the $3$-cycles of $D^{\prime}$ of Type 2$_{x}$ are $\left(
x,v,u\right)  $, $\left(  v,u,x\right)  $ and $\left(  u,x,v\right)  $ (this
is, of course, essentially just one $3$-cycle and its cyclic rotations),
whereas the $3$-cycles of $D$ of Type 2$_{x}$ are $\left(  x,w,u\right)  $,
$\left(  w,u,x\right)  $ and $\left(  u,x,w\right)  $. The number of the
former equals the number of the latter (namely, both numbers are $3$). This
proves (\ref{pf.prop.tour-vand.1.2nd.type2x}) in Case 5. A similar argument
works in each of the other seven cases. Thus,
(\ref{pf.prop.tour-vand.1.2nd.type2x}) is proved.] \medskip

Finally, the only $3$-cycles of $D^{\prime}$ of Type 3 are $\left(
u,w,v\right)  $, $\left(  w,v,u\right)  $ and $\left(  v,u,w\right)  $,
whereas those of $D$ are $\left(  u,v,w\right)  $, $\left(  v,w,u\right)  $
and $\left(  w,u,v\right)  $. Thus, we have%
\begin{align}
&  \left(  \#\text{ of }3\text{-cycles of }D^{\prime}\text{ of Type 3}\right)
\nonumber\\
&  =\left(  \#\text{ of }3\text{-cycles of }D\text{ of Type 3}\right)
\label{pf.prop.tour-vand.1.2nd.type3}%
\end{align}
(since both of these numbers equal $3$).

Now, recall that each $3$-cycle of $D$ has exactly one of the three Types 1, 2
and 3; moreover, if it has Type 2, then it has Type 2$_{x}$ for a unique $x\in
V\setminus\left\{  u,v,w\right\}  $. Thus,%
\begin{align}
&  \left(  \#\text{ of }3\text{-cycles of }D\right) \nonumber\\
&  =\left(  \#\text{ of }3\text{-cycles of }D\text{ of Type 1}\right)
+\sum_{x\in V\setminus\left\{  u,v,w\right\}  }\left(  \#\text{ of
}3\text{-cycles of }D\text{ of Type 2}_{x}\right) \nonumber\\
&  \ \ \ \ \ \ \ \ \ \ +\left(  \#\text{ of }3\text{-cycles of }D\text{ of
Type 3}\right)  . \label{pf.prop.tour-vand.1.2nd.4}%
\end{align}
The same argument can be made for $D^{\prime}$ instead of $D$, and thus we
obtain%
\begin{align}
&  \left(  \#\text{ of }3\text{-cycles of }D^{\prime}\right) \nonumber\\
&  =\left(  \#\text{ of }3\text{-cycles of }D^{\prime}\text{ of Type
1}\right)  +\sum_{x\in V\setminus\left\{  u,v,w\right\}  }\left(  \#\text{ of
}3\text{-cycles of }D^{\prime}\text{ of Type 2}_{x}\right) \nonumber\\
&  \ \ \ \ \ \ \ \ \ \ +\left(  \#\text{ of }3\text{-cycles of }D^{\prime
}\text{ of Type 3}\right)  . \label{pf.prop.tour-vand.1.2nd.5}%
\end{align}
The equalities (\ref{pf.prop.tour-vand.1.2nd.type1}),
(\ref{pf.prop.tour-vand.1.2nd.type2x}) and
(\ref{pf.prop.tour-vand.1.2nd.type3}) show that the right hand side of
(\ref{pf.prop.tour-vand.1.2nd.5}) equals the right hand side of
(\ref{pf.prop.tour-vand.1.2nd.4}). Hence, the left hand side of
(\ref{pf.prop.tour-vand.1.2nd.5}) equals the left hand side of
(\ref{pf.prop.tour-vand.1.2nd.4}). In other words, we have%
\[
\left(  \#\text{ of }3\text{-cycles of }D^{\prime}\right)  =\left(  \#\text{
of }3\text{-cycles of }D\right)  .
\]
This proves Proposition \ref{prop.tour-vand.1} again.
\end{proof}

\subsection{Reminders on permutations}

Now, let us recall the definitions and a few basic properties of permutations
and symmetric groups:

\begin{definition}
A \emph{permutation} of a set $X$ means a bijection from $X$ to $X$.
\end{definition}

\begin{definition}
For each $n\in\mathbb{N}$, we let $S_{n}$ denote the set of all permutations
of $\left\{  1,2,\ldots,n\right\}  $. Note that $\left\vert S_{n}\right\vert
=n!$.
\end{definition}

There are several ways to represent (i.e., write down) a permutation
$\sigma\in S_{n}$ for a given $n\in\mathbb{N}$:

\begin{itemize}
\item \emph{One-line notation:} We can represent $\sigma$ by the $n$-tuple
$\left[  \sigma\left(  1\right)  ,\sigma\left(  2\right)  ,\ldots
,\sigma\left(  n\right)  \right]  $ of all its values. (Note that we would
normally use parentheses rather than square brackets here, but it is a habit
of combinatorialists to use square brackets in this specific situation.)

\item \emph{Two-line notation:} We can represent $\sigma$ by the $2\times
n$-table $\left(
\begin{array}
[c]{cccc}%
1 & 2 & \cdots & n\\
\sigma\left(  1\right)  & \sigma\left(  2\right)  & \cdots & \sigma\left(
n\right)
\end{array}
\right)  $.

\item \emph{Cycle digraph:} We can represent $\sigma$ by the digraph%
\[
\left(  \left\{  1,2,\ldots,n\right\}  ,\ \left\{  \left(  i,\sigma\left(
i\right)  \right)  \ \mid\ i\in\left\{  1,2,\ldots,n\right\}  \right\}
\right)  .
\]
This is the digraph whose vertices are $1,2,\ldots,n$, and whose arcs are the
pairs $\left(  i,\sigma\left(  i\right)  \right)  $ for all $i\in\left\{
1,2,\ldots,n\right\}  $. This digraph is called the \emph{cycle digraph} of
$\sigma$.
\end{itemize}

\begin{example}
\label{exa.perm.6}Let $\sigma$ be the permutation of $\left\{
1,2,3,4,5,6\right\}  $ that sends%
\[
1\mapsto3,\ \ \ \ \ \ \ \ \ \ 2\mapsto2,\ \ \ \ \ \ \ \ \ \ 3\mapsto
6,\ \ \ \ \ \ \ \ \ \ 4\mapsto5,\ \ \ \ \ \ \ \ \ \ 5\mapsto
4,\ \ \ \ \ \ \ \ \ \ 6\mapsto1.
\]
Then:

\begin{itemize}
\item The one-line notation for $\sigma$ is $\left[  3,2,6,5,4,1\right]  $.
(Some combinatorialists would drop the brackets and commas, and shorten this
to $326541$; but we have no need for this much shorthand.)

\item The two-line notation for $\sigma$ is $\left(
\begin{array}
[c]{cccccc}%
1 & 2 & 3 & 4 & 5 & 6\\
3 & 2 & 6 & 5 & 4 & 1
\end{array}
\right)  $.

\item The cycle digraph of $\sigma$ is%
\[%
%TCIMACRO{\TeXButton{tikz cycle digraph}{\begin{tikzpicture}[scale=1.2]
%\begin{scope}[every node/.style={circle,thick,draw=green!60!black}]
%\node(1) at (0:1.5) {$1$};
%\node(3) at (120:1.5) {$3$};
%\node(6) at (240:1.5) {$6$};
%\node(4) at (2.8, -1) {$4$};
%\node(5) at (4.2, -1) {$5$};
%\node(2) at (3.5, 0.5) {$2$};
%\end{scope}
%\begin{scope}[every edge/.style={draw=black,very thick}, every loop/.style={}]
%\path[->] (1) edge (3) (3) edge (6) (6) edge (1);
%\path[->] (4) edge[bend left=20] (5) (5) edge[bend left=20] (4);
%\path[->] (2) edge[loop above] (2);
%\end{scope}
%\end{tikzpicture}}}%
%BeginExpansion
\begin{tikzpicture}[scale=1.2]
\begin{scope}[every node/.style={circle,thick,draw=green!60!black}]
\node(1) at (0:1.5) {$1$};
\node(3) at (120:1.5) {$3$};
\node(6) at (240:1.5) {$6$};
\node(4) at (2.8, -1) {$4$};
\node(5) at (4.2, -1) {$5$};
\node(2) at (3.5, 0.5) {$2$};
\end{scope}
\begin{scope}[every edge/.style={draw=black,very thick}, every loop/.style={}]
\path[->] (1) edge (3) (3) edge (6) (6) edge (1);
\path[->] (4) edge[bend left=20] (5) (5) edge[bend left=20] (4);
\path[->] (2) edge[loop above] (2);
\end{scope}
\end{tikzpicture}%
%EndExpansion
\ \ .
\]

\end{itemize}
\end{example}

\begin{remark}
Let $n\in\mathbb{N}$, and let $\sigma\in S_{n}$. Then, the cycle digraph of
$\sigma$ has the property that each vertex $v$ satisfies%
\[
\deg^{-}v=1\ \ \ \ \ \ \ \ \ \ \text{and}\ \ \ \ \ \ \ \ \ \ \deg^{+}v=1.
\]
From this fact, it follows easily that this digraph is a disjoint union of
cycles (including cycles of length $1$). This is easily seen to be equivalent
to the classical result that the permutation $\sigma$ can be uniquely written
as a composition of disjoint cycles. We will not need this fact, but we found
it worth mentioning; details can be found in \cite[proof of Theorem
5.5.2]{lecs}.
\end{remark}

For the theory of determinants, the most important feature of a permutation is
its sign. Let us recall its definition:

\begin{definition}
Let $n\in\mathbb{N}$, and let $\sigma\in S_{n}$.

\begin{enumerate}
\item[\textbf{(a)}] An \emph{inversion} of $\sigma$ means a pair $\left(
i,j\right)  $ of integers $i$ and $j$ such that%
\[
1\leq i<j\leq n\ \ \ \ \ \ \ \ \ \ \text{and}\ \ \ \ \ \ \ \ \ \ \sigma\left(
i\right)  >\sigma\left(  j\right)  .
\]


\item[\textbf{(b)}] The \emph{length} of $\sigma$ is defined to be the number
of inversions of $\sigma$. It is denoted by $\ell\left(  \sigma\right)  $.

\item[\textbf{(c)}] The \emph{sign} of $\sigma$ is defined to be the number
$\left(  -1\right)  ^{\ell\left(  \sigma\right)  }$. It is denoted by $\left(
-1\right)  ^{\sigma}$ or $\operatorname*{sign}\sigma$ or $\operatorname*{sgn}%
\sigma$ or $\varepsilon\left(  \sigma\right)  $. (We will use the notation
$\operatorname*{sign}\sigma$.)
\end{enumerate}
\end{definition}

For example, the permutation $\sigma\in S_{6}$ from Example \ref{exa.perm.6}
has inversions $\left(  1,2\right)  $, $\left(  1,6\right)  $, $\left(
2,6\right)  $, $\left(  3,4\right)  $, $\left(  3,5\right)  $, $\left(
3,6\right)  $, $\left(  4,5\right)  $, $\left(  4,6\right)  $ and $\left(
5,6\right)  $, and thus has length $\ell\left(  \sigma\right)  =9$ and sign
$\operatorname*{sign}\sigma=\left(  -1\right)  ^{\ell\left(  \sigma\right)
}=\left(  -1\right)  ^{9}=-1$.

Here are some well-known properties of signs and lengths:

\begin{proposition}
\label{prop.perms.sign.var}Let $n\in\mathbb{N}$.

\begin{enumerate}
\item[\textbf{(a)}] For any $\sigma\in S_{n}$, we have $\operatorname*{sign}%
\sigma\in\left\{  1,-1\right\}  $.

\item[\textbf{(b)}] The identity permutation $\operatorname*{id}\in S_{n}$ has
sign $\operatorname*{sign}\left(  \operatorname*{id}\right)  =1$.

\item[\textbf{(c)}] If $\tau\in S_{n}$ is any transposition, then
$\operatorname*{sign}\tau=-1$.

\item[\textbf{(d)}] For any two permutations $\sigma,\tau\in S_{n}$, we have
$\operatorname*{sign}\left(  \sigma\circ\tau\right)  =\operatorname*{sign}%
\sigma\cdot\operatorname*{sign}\tau$.

\item[\textbf{(e)}] For any $\sigma\in S_{n}$, we have $\operatorname*{sign}%
\left(  \sigma^{-1}\right)  =\operatorname*{sign}\sigma$.

\item[\textbf{(f)}] If the cycle digraph of $\sigma$ has $r$ cycles (counted
up to rotation -- so that, e.g., the cycle digraph in Example \ref{exa.perm.6}
has $3$ cycles), then $\operatorname*{sign}\sigma=\left(  -1\right)  ^{n-r}$.

\item[\textbf{(g)}] We have $\operatorname*{sign}\sigma=\prod_{1\leq i<j\leq
n}\dfrac{\sigma\left(  i\right)  -\sigma\left(  j\right)  }{i-j}$.

\item[\textbf{(h)}] If you write down the one-line notation for $\sigma$, and
sort it into increasing order by repeatedly swapping adjacent entries (this
way of sorting a tuple is called \textquotedblleft
bubblesort\textquotedblright, or more precisely, is a more general version of
bubblesort), then $\ell\left(  \sigma\right)  $ is the smallest possible
number of swaps you will need. (Actually, it is the exact number of swaps you
will need if you don't waste time by swapping pairs of entries that already
are in increasing order.)
\end{enumerate}
\end{proposition}

Proofs of the claims of Proposition \ref{prop.perms.sign.var} can be found in
\cite[\S 5.1--\S 5.3]{detnotes}, \cite{Conrad}, \cite[Chapter 6.B]{Day16},
\cite[Appendix B]{Strick20} and various other sources (including most serious
texts on linear algebra or introductory abstract algebra). We will not need
them here, however.

\subsection{Determinants}

Using the notion of signs, we now recall the definition of a determinant:

\begin{definition}
\label{def.det.det}Let $A$ be an $n\times n$-matrix (say, with real entries --
more generally, it can have entries from an arbitrary commutative ring).

For all $i,j\in\left\{  1,2,\ldots,n\right\}  $, we let $a_{i,j}$ be the
$\left(  i,j\right)  $-th entry of $A$ (that is, the entry of $A$ in the
$i$-th row and the $j$-th column).

Then, the \emph{determinant} of $A$ is the number $\det A$ defined by%
\begin{equation}
\det A:=\sum_{\sigma\in S_{n}}\operatorname*{sign}\sigma\cdot\prod_{i=1}%
^{n}a_{i,\sigma\left(  i\right)  }. \label{eq.def.det.det.eq}%
\end{equation}

\end{definition}

The formula (\ref{eq.def.det.det.eq}) is known as the \emph{Leibniz formula}.
Among many equivalent definitions of the determinant, it is the most explicit one.

\subsection{The Vandermonde determinant}

We can now state the theorem we shall apply our graph-theoretical machinery to:

\begin{theorem}
[Vandermonde determinant]\label{thm.det.vander}Let $n\in\mathbb{N}$.

Let $x_{1},x_{2},\ldots,x_{n}$ be $n$ numbers (or, more generally, $n$
elements of a commutative ring).

Let $V$ be the $n\times n$-matrix whose $\left(  i,j\right)  $-th entry is
$x_{j}^{i-1}$ for all $i,j\in\left\{  1,2,\ldots,n\right\}  $. That is, let
\[
V:=\left(
\begin{array}
[c]{ccccc}%
1 & 1 & 1 & \cdots & 1\\
x_{1} & x_{2} & x_{3} & \cdots & x_{n}\\
x_{1}^{2} & x_{2}^{2} & x_{3}^{2} & \cdots & x_{n}^{2}\\
\vdots & \vdots & \vdots & \ddots & \vdots\\
x_{1}^{n-1} & x_{2}^{n-1} & x_{3}^{n-1} & \cdots & x_{n}^{n-1}%
\end{array}
\right)  .
\]


Then,%
\begin{equation}
\det V=\prod_{1\leq i<j\leq n}\left(  x_{j}-x_{i}\right)  .
\label{eq.thm.det.vander.eq}%
\end{equation}

\end{theorem}

This theorem is one of several equivalent versions of the \textquotedblleft
Vandermonde determinant\textquotedblright. Some algebraic proofs can be found
in \cite[\S 6.7]{detnotes} and in \cite[Theorem 6.4.31]{lecs}. We shall here
give a combinatorial proof using tournaments. (This proof is a mild variation
of the proof found by Ira Gessel, published in \cite{Gessel79}.)

\subsection{Tournaments on $\left\{  1,2,\ldots,n\right\}  $}

\begin{definition}
The \emph{vertex set} of a digraph $D=\left(  V,A\right)  $ is defined to be
the set $V$. This is the set of vertices of $D$.
\end{definition}

\begin{convention}
We fix $n\in\mathbb{N}$, and we also fix $n$ numbers $x_{1},x_{2},\ldots
,x_{n}$.

We let $\mathcal{T}$ be the set of all tournaments with vertex set $\left\{
1,2,\ldots,n\right\}  $.
\end{convention}

Thus, a tournament $D\in\mathcal{T}$ has vertices $1,2,\ldots,n$. We introduce
another convenient notion:

\begin{definition}
Let $i$ and $j$ be two elements of $\left\{  1,2,\ldots,n\right\}  $. If
$i<j$, then the pair $\left(  i,j\right)  $ is said to be \emph{increasing}.
If $i>j$, then the pair $\left(  i,j\right)  $ is said to be \emph{decreasing}.
\end{definition}

Thus, for each increasing pair $\left(  i,j\right)  $, there is a
corresponding decreasing pair $\left(  j,i\right)  $. There are exactly
$n\left(  n-1\right)  /2$ many increasing pairs:%
\[%
\begin{array}
[c]{cccc}%
\left(  1,2\right)  , & \left(  1,3\right)  , & \ldots, & \left(  1,n\right)
,\\
& \left(  2,3\right)  , & \ldots, & \left(  2,n\right)  ,\\
&  & \ddots & \vdots\\
&  &  & \left(  n-1,n\right)  .
\end{array}
\]


\Needspace{35pc}

\begin{example}
\label{exa.tour1n.3}The following table shows all eight tournaments
$D\in\mathcal{T}$ in the case when $n=3$ (where we are drawing all increasing
arcs in blue and all decreasing arcs in red):%
\[%
\begin{tabular}
[c]{|c|c|c}\hline
$%
%TCIMACRO{\TeXButton{tikz tournament}{\begin{tikzpicture}
%\draw[draw=black] (-2, -1.4) rectangle (2, 2.2);
%\begin{scope}[every node/.style={circle,thick,draw=green!60!black}]
%\node(1) at (90:1.5) {$1$};
%\node(2) at (210:1.5) {$2$};
%\node(3) at (330:1.5) {$3$};
%\end{scope}
%\node(u1) at (0, -1.8) {$\left\{\right\}$};
%\begin{scope}[every edge/.style={draw=red,very thick}]
%\path[->] (2) edge (1) (3) edge (1) (3) edge (2);
%\end{scope}
%\end{tikzpicture}}}%
%BeginExpansion
\begin{tikzpicture}
\draw[draw=black] (-2, -1.4) rectangle (2, 2.2);
\begin{scope}[every node/.style={circle,thick,draw=green!60!black}]
\node(1) at (90:1.5) {$1$};
\node(2) at (210:1.5) {$2$};
\node(3) at (330:1.5) {$3$};
\end{scope}
\node(u1) at (0, -1.8) {$\left\{\right\}$};
\begin{scope}[every edge/.style={draw=red,very thick}]
\path[->] (2) edge (1) (3) edge (1) (3) edge (2);
\end{scope}
\end{tikzpicture}%
%EndExpansion
$ & $%
%TCIMACRO{\TeXButton{tikz tournament}{\begin{tikzpicture}
%\draw[draw=black] (-2, -1.4) rectangle (2, 2.2);
%\begin{scope}[every node/.style={circle,thick,draw=green!60!black}]
%\node(1) at (90:1.5) {$1$};
%\node(2) at (210:1.5) {$2$};
%\node(3) at (330:1.5) {$3$};
%\end{scope}
%\node(u1) at (0, -1.8) {$\left\{\left(1,2\right)\right\}$};
%\begin{scope}[every edge/.style={draw=red,very thick}]
%\path[->] (3) edge (1) (3) edge (2);
%\end{scope}
%\begin{scope}[every edge/.style={draw=blue,very thick}]
%\path[->] (1) edge (2);
%\end{scope}
%\end{tikzpicture}}}%
%BeginExpansion
\begin{tikzpicture}
\draw[draw=black] (-2, -1.4) rectangle (2, 2.2);
\begin{scope}[every node/.style={circle,thick,draw=green!60!black}]
\node(1) at (90:1.5) {$1$};
\node(2) at (210:1.5) {$2$};
\node(3) at (330:1.5) {$3$};
\end{scope}
\node(u1) at (0, -1.8) {$\left\{\left(1,2\right)\right\}$};
\begin{scope}[every edge/.style={draw=red,very thick}]
\path[->] (3) edge (1) (3) edge (2);
\end{scope}
\begin{scope}[every edge/.style={draw=blue,very thick}]
\path[->] (1) edge (2);
\end{scope}
\end{tikzpicture}%
%EndExpansion
$ & \multicolumn{1}{|c|}{$%
%TCIMACRO{\TeXButton{tikz tournament}{\begin{tikzpicture}
%\draw[draw=black] (-2, -1.4) rectangle (2, 2.2);
%\begin{scope}[every node/.style={circle,thick,draw=green!60!black}]
%\node(1) at (90:1.5) {$1$};
%\node(2) at (210:1.5) {$2$};
%\node(3) at (330:1.5) {$3$};
%\end{scope}
%\node(u1) at (0, -1.8) {$\left\{\left(1,3\right)\right\}$};
%\begin{scope}[every edge/.style={draw=red,very thick}]
%\path[->] (2) edge (1) (3) edge (2);
%\end{scope}
%\begin{scope}[every edge/.style={draw=blue,very thick}]
%\path[->] (1) edge (3);
%\end{scope}
%\end{tikzpicture}}}%
%BeginExpansion
\begin{tikzpicture}
\draw[draw=black] (-2, -1.4) rectangle (2, 2.2);
\begin{scope}[every node/.style={circle,thick,draw=green!60!black}]
\node(1) at (90:1.5) {$1$};
\node(2) at (210:1.5) {$2$};
\node(3) at (330:1.5) {$3$};
\end{scope}
\node(u1) at (0, -1.8) {$\left\{\left(1,3\right)\right\}$};
\begin{scope}[every edge/.style={draw=red,very thick}]
\path[->] (2) edge (1) (3) edge (2);
\end{scope}
\begin{scope}[every edge/.style={draw=blue,very thick}]
\path[->] (1) edge (3);
\end{scope}
\end{tikzpicture}%
%EndExpansion
$}\\\hline
$%
%TCIMACRO{\TeXButton{tikz tournament}{\begin{tikzpicture}
%\draw[draw=black] (-2, -1.4) rectangle (2, 2.2);
%\begin{scope}[every node/.style={circle,thick,draw=green!60!black}]
%\node(1) at (90:1.5) {$1$};
%\node(2) at (210:1.5) {$2$};
%\node(3) at (330:1.5) {$3$};
%\end{scope}
%\node(u1) at (0, -1.8) {$\left\{\left(2,3\right)\right\}$};
%\begin{scope}[every edge/.style={draw=red,very thick}]
%\path[->] (2) edge (1) (3) edge (1);
%\end{scope}
%\begin{scope}[every edge/.style={draw=blue,very thick}]
%\path[->] (2) edge (3);
%\end{scope}
%\end{tikzpicture}}}%
%BeginExpansion
\begin{tikzpicture}
\draw[draw=black] (-2, -1.4) rectangle (2, 2.2);
\begin{scope}[every node/.style={circle,thick,draw=green!60!black}]
\node(1) at (90:1.5) {$1$};
\node(2) at (210:1.5) {$2$};
\node(3) at (330:1.5) {$3$};
\end{scope}
\node(u1) at (0, -1.8) {$\left\{\left(2,3\right)\right\}$};
\begin{scope}[every edge/.style={draw=red,very thick}]
\path[->] (2) edge (1) (3) edge (1);
\end{scope}
\begin{scope}[every edge/.style={draw=blue,very thick}]
\path[->] (2) edge (3);
\end{scope}
\end{tikzpicture}%
%EndExpansion
$ & $%
%TCIMACRO{\TeXButton{tikz tournament}{\begin{tikzpicture}
%\draw[draw=black] (-2, -1.4) rectangle (2, 2.2);
%\begin{scope}[every node/.style={circle,thick,draw=green!60!black}]
%\node(1) at (90:1.5) {$1$};
%\node(2) at (210:1.5) {$2$};
%\node(3) at (330:1.5) {$3$};
%\end{scope}
%\node(u1) at (0, -1.8) {$\left\{\left(1,2\right),\left(1,3\right)\right\}$};
%\begin{scope}[every edge/.style={draw=red,very thick}]
%\path[->] (3) edge (2);
%\end{scope}
%\begin{scope}[every edge/.style={draw=blue,very thick}]
%\path[->] (1) edge (2) (1) edge (3);
%\end{scope}
%\end{tikzpicture}}}%
%BeginExpansion
\begin{tikzpicture}
\draw[draw=black] (-2, -1.4) rectangle (2, 2.2);
\begin{scope}[every node/.style={circle,thick,draw=green!60!black}]
\node(1) at (90:1.5) {$1$};
\node(2) at (210:1.5) {$2$};
\node(3) at (330:1.5) {$3$};
\end{scope}
\node(u1) at (0, -1.8) {$\left\{\left(1,2\right),\left(1,3\right)\right\}$};
\begin{scope}[every edge/.style={draw=red,very thick}]
\path[->] (3) edge (2);
\end{scope}
\begin{scope}[every edge/.style={draw=blue,very thick}]
\path[->] (1) edge (2) (1) edge (3);
\end{scope}
\end{tikzpicture}%
%EndExpansion
$ & \multicolumn{1}{|c|}{$%
%TCIMACRO{\TeXButton{tikz tournament}{\begin{tikzpicture}
%\draw[draw=black] (-2, -1.4) rectangle (2, 2.2);
%\begin{scope}[every node/.style={circle,thick,draw=green!60!black}]
%\node(1) at (90:1.5) {$1$};
%\node(2) at (210:1.5) {$2$};
%\node(3) at (330:1.5) {$3$};
%\end{scope}
%\node(u1) at (0, -1.8) {$\left\{\left(1,2\right),\left(2,3\right)\right\}$};
%\begin{scope}[every edge/.style={draw=red,very thick}]
%\path[->] (3) edge (1);
%\end{scope}
%\begin{scope}[every edge/.style={draw=blue,very thick}]
%\path[->] (1) edge (2) (2) edge (3);
%\end{scope}
%\end{tikzpicture}}}%
%BeginExpansion
\begin{tikzpicture}
\draw[draw=black] (-2, -1.4) rectangle (2, 2.2);
\begin{scope}[every node/.style={circle,thick,draw=green!60!black}]
\node(1) at (90:1.5) {$1$};
\node(2) at (210:1.5) {$2$};
\node(3) at (330:1.5) {$3$};
\end{scope}
\node(u1) at (0, -1.8) {$\left\{\left(1,2\right),\left(2,3\right)\right\}$};
\begin{scope}[every edge/.style={draw=red,very thick}]
\path[->] (3) edge (1);
\end{scope}
\begin{scope}[every edge/.style={draw=blue,very thick}]
\path[->] (1) edge (2) (2) edge (3);
\end{scope}
\end{tikzpicture}%
%EndExpansion
$}\\\hline
$%
%TCIMACRO{\TeXButton{tikz tournament}{\begin{tikzpicture}
%\draw[draw=black] (-2, -1.4) rectangle (2, 2.2);
%\begin{scope}[every node/.style={circle,thick,draw=green!60!black}]
%\node(1) at (90:1.5) {$1$};
%\node(2) at (210:1.5) {$2$};
%\node(3) at (330:1.5) {$3$};
%\end{scope}
%\node(u1) at (0, -1.8) {$\left\{\left(1,3\right),\left(2,3\right)\right\}$};
%\begin{scope}[every edge/.style={draw=red,very thick}]
%\path[->] (2) edge (1);
%\end{scope}
%\begin{scope}[every edge/.style={draw=blue,very thick}]
%\path[->] (1) edge (3) (2) edge (3);
%\end{scope}
%\end{tikzpicture}}}%
%BeginExpansion
\begin{tikzpicture}
\draw[draw=black] (-2, -1.4) rectangle (2, 2.2);
\begin{scope}[every node/.style={circle,thick,draw=green!60!black}]
\node(1) at (90:1.5) {$1$};
\node(2) at (210:1.5) {$2$};
\node(3) at (330:1.5) {$3$};
\end{scope}
\node(u1) at (0, -1.8) {$\left\{\left(1,3\right),\left(2,3\right)\right\}$};
\begin{scope}[every edge/.style={draw=red,very thick}]
\path[->] (2) edge (1);
\end{scope}
\begin{scope}[every edge/.style={draw=blue,very thick}]
\path[->] (1) edge (3) (2) edge (3);
\end{scope}
\end{tikzpicture}%
%EndExpansion
$ & $%
%TCIMACRO{\TeXButton{tikz tournament}{\begin{tikzpicture}
%\draw[draw=black] (-2, -1.4) rectangle (2, 2.2);
%\begin{scope}[every node/.style={circle,thick,draw=green!60!black}]
%\node(1) at (90:1.5) {$1$};
%\node(2) at (210:1.5) {$2$};
%\node(3) at (330:1.5) {$3$};
%\end{scope}
%\node(u1) at (0, -1.8) {$\left\{\left(1,2\right),\left(1,3\right
%),\left(2,3\right)\right\}$};
%\begin{scope}[every edge/.style={draw=red,very thick}]
%\path[->] (1) edge (2) (1) edge (3) (2) edge (3);
%\end{scope}
%\end{tikzpicture}}}%
%BeginExpansion
\begin{tikzpicture}
\draw[draw=black] (-2, -1.4) rectangle (2, 2.2);
\begin{scope}[every node/.style={circle,thick,draw=green!60!black}]
\node(1) at (90:1.5) {$1$};
\node(2) at (210:1.5) {$2$};
\node(3) at (330:1.5) {$3$};
\end{scope}
\node(u1) at (0, -1.8) {$\left\{\left(1,2\right),\left(1,3\right
),\left(2,3\right)\right\}$};
\begin{scope}[every edge/.style={draw=red,very thick}]
\path[->] (1) edge (2) (1) edge (3) (2) edge (3);
\end{scope}
\end{tikzpicture}%
%EndExpansion
$ & \\\cline{1-2}%
\end{tabular}
\
\]
Underneath each tournament, we have written down the set of all increasing
arcs of this tournament.
\end{example}

Now, let $D\in\mathcal{T}$ be a tournament. Then, $D$ is loopless (by
definition), so that each of its arcs is either increasing or decreasing.
Moreover, the tournament axiom shows that for any increasing pair $\left(
i,j\right)  $, exactly one of the two pairs $\left(  i,j\right)  $ and
$\left(  j,i\right)  $ is an arc of $D$. In other words, an increasing pair
$\left(  i,j\right)  $ is an arc of $D$ if and only if the corresponding
decreasing pair $\left(  j,i\right)  $ is not. Thus, $D$ is uniquely
determined if we know which increasing pairs are arcs of $D$.

Forget that we fixed $D$. We thus have shown that a tournament $D\in
\mathcal{T}$ is uniquely determined if we know which increasing pairs are arcs
of $D$. In other words, a tournament $D\in\mathcal{T}$ is uniquely determined
by the set of all increasing arcs of $D$. Moreover, for any set $S$ of
increasing pairs, there is a unique tournament $D\in\mathcal{T}$ such that $S$
is the set of all increasing arcs of $D$. Thus, the map%
\begin{align*}
\mathcal{T}  &  \rightarrow\left\{  \text{all sets of increasing
pairs}\right\}  ,\\
D  &  \mapsto\left\{  \text{all increasing arcs of }D\right\}
\end{align*}
is a bijection from the set $\mathcal{T}$ to the set of all sets of increasing
pairs. The bijection principle\footnote{The \emph{bijection principle} says
that if $f:X\rightarrow Y$ is a bijection from a set $X$ to a set $Y$, then
$\left\vert X\right\vert =\left\vert Y\right\vert $.} therefore yields%
\[
\left\vert \mathcal{T}\right\vert =\left\vert \left\{  \text{all sets of
increasing pairs}\right\}  \right\vert =2^{n\left(  n-1\right)  /2}%
\]
(since there are exactly $n\left(  n-1\right)  /2$ many increasing pairs).

\begin{convention}
We shall use the \emph{Iverson bracket notation}: If $\mathcal{A}$ is any
logical statement, then $\left[  \mathcal{A}\right]  $ will mean the number $%
\begin{cases}
1, & \text{if }\mathcal{A}\text{ is true;}\\
0, & \text{if }\mathcal{A}\text{ is false.}%
\end{cases}
$

For instance, $\left[  2+2=4\right]  =1$ and $\left[  2+2=5\right]  =0$.

The number $\left[  \mathcal{A}\right]  $ is called the \emph{truth value} of
the statement $\mathcal{A}$.
\end{convention}

\begin{definition}
\label{def.tour1n.wD}Let $D\in\mathcal{T}$ be a tournament. Then:

\begin{enumerate}
\item[\textbf{(a)}] The \emph{sign} of $D$ is defined to be the integer%
\begin{equation}
\operatorname*{sign}D:=\prod_{\substack{\left(  i,j\right)  \text{ is
an}\\\text{arc of }D}}\left(  -1\right)  ^{\left[  i>j\right]  }\in\left\{
1,-1\right\}  .\label{eq.def.tour1n.wD.sign}%
\end{equation}


\item[\textbf{(b)}] The \emph{x-weight} of $D$ is defined to be the number%
\begin{equation}
w\left(  D\right)  :=\prod_{\substack{\left(  i,j\right)  \text{ is
an}\\\text{arc of }D}}\left(  \left(  -1\right)  ^{\left[  i>j\right]  }%
x_{j}\right)  . \label{eq.def.tour1n.wD.eq}%
\end{equation}

\end{enumerate}
\end{definition}

\begin{example}
For $n=3$, the tournament%
\[
\left(  \left\{  1,2,3\right\}  ,\ \left\{  \left(  2,1\right)  ,\ \left(
2,3\right)  ,\ \left(  3,1\right)  \right\}  \right)
\]
can be drawn as follows:%
\[%
%TCIMACRO{\TeXButton{tikz tournament}{\begin{tikzpicture}
%\begin{scope}[every node/.style={circle,thick,draw=green!60!black}]
%\node(1) at (90:1.5) {$1$};
%\node(2) at (210:1.5) {$2$};
%\node(3) at (330:1.5) {$3$};
%\end{scope}
%\begin{scope}[every edge/.style={draw=red,very thick}]
%\path[->] (2) edge (1) (3) edge (1);
%\end{scope}
%\begin{scope}[every edge/.style={draw=blue,very thick}]
%\path[->] (2) edge (3);
%\end{scope}
%\end{tikzpicture}}}%
%BeginExpansion
\begin{tikzpicture}
\begin{scope}[every node/.style={circle,thick,draw=green!60!black}]
\node(1) at (90:1.5) {$1$};
\node(2) at (210:1.5) {$2$};
\node(3) at (330:1.5) {$3$};
\end{scope}
\begin{scope}[every edge/.style={draw=red,very thick}]
\path[->] (2) edge (1) (3) edge (1);
\end{scope}
\begin{scope}[every edge/.style={draw=blue,very thick}]
\path[->] (2) edge (3);
\end{scope}
\end{tikzpicture}%
%EndExpansion
\ \ .
\]
Its sign is
\[
\underbrace{\left(  -1\right)  ^{\left[  2>1\right]  }}_{=\left(  -1\right)
^{1}=-1}\cdot\underbrace{\left(  -1\right)  ^{\left[  2>3\right]  }}_{=\left(
-1\right)  ^{0}=1}\cdot\underbrace{\left(  -1\right)  ^{\left[  3>1\right]  }%
}_{=\left(  -1\right)  ^{1}=-1}=\left(  -1\right)  \cdot1\cdot\left(
-1\right)  =1.
\]
Its x-weight is%
\[
\underbrace{\left(  -1\right)  ^{\left[  2>1\right]  }}_{=\left(  -1\right)
^{1}=-1}x_{1}\cdot\underbrace{\left(  -1\right)  ^{\left[  2>3\right]  }%
}_{=\left(  -1\right)  ^{0}=1}x_{3}\cdot\underbrace{\left(  -1\right)
^{\left[  3>1\right]  }}_{=\left(  -1\right)  ^{1}=-1}x_{1}=\left(  -1\right)
x_{1}\cdot1x_{3}\cdot\left(  -1\right)  x_{1}=x_{1}^{2}x_{3}.
\]

\end{example}

\Needspace{35pc}

\begin{example}
For $n=3$, here are all the tournaments $D\in\mathcal{T}$ listed along with
their x-weights $w\left(  D\right)  $:%
\[%
\begin{tabular}
[c]{|c|c|c}\hline
$%
%TCIMACRO{\TeXButton{tikz tournament}{\begin{tikzpicture}
%\draw[draw=black] (-2, -1.4) rectangle (2, 2.2);
%\begin{scope}[every node/.style={circle,thick,draw=green!60!black}]
%\node(1) at (90:1.5) {$1$};
%\node(2) at (210:1.5) {$2$};
%\node(3) at (330:1.5) {$3$};
%\end{scope}
%\node(u1) at (0, -1.8) {$- x_1^2 x_2$};
%\begin{scope}[every edge/.style={draw=red,very thick}]
%\path[->] (2) edge (1) (3) edge (1) (3) edge (2);
%\end{scope}
%\end{tikzpicture}}}%
%BeginExpansion
\begin{tikzpicture}
\draw[draw=black] (-2, -1.4) rectangle (2, 2.2);
\begin{scope}[every node/.style={circle,thick,draw=green!60!black}]
\node(1) at (90:1.5) {$1$};
\node(2) at (210:1.5) {$2$};
\node(3) at (330:1.5) {$3$};
\end{scope}
\node(u1) at (0, -1.8) {$- x_1^2 x_2$};
\begin{scope}[every edge/.style={draw=red,very thick}]
\path[->] (2) edge (1) (3) edge (1) (3) edge (2);
\end{scope}
\end{tikzpicture}%
%EndExpansion
$ & $%
%TCIMACRO{\TeXButton{tikz tournament}{\begin{tikzpicture}
%\draw[draw=black] (-2, -1.4) rectangle (2, 2.2);
%\begin{scope}[every node/.style={circle,thick,draw=green!60!black}]
%\node(1) at (90:1.5) {$1$};
%\node(2) at (210:1.5) {$2$};
%\node(3) at (330:1.5) {$3$};
%\end{scope}
%\node(u1) at (0, -1.8) {$x_1 x_2^2$};
%\begin{scope}[every edge/.style={draw=red,very thick}]
%\path[->] (3) edge (1) (3) edge (2);
%\end{scope}
%\begin{scope}[every edge/.style={draw=blue,very thick}]
%\path[->] (1) edge (2);
%\end{scope}
%\end{tikzpicture}}}%
%BeginExpansion
\begin{tikzpicture}
\draw[draw=black] (-2, -1.4) rectangle (2, 2.2);
\begin{scope}[every node/.style={circle,thick,draw=green!60!black}]
\node(1) at (90:1.5) {$1$};
\node(2) at (210:1.5) {$2$};
\node(3) at (330:1.5) {$3$};
\end{scope}
\node(u1) at (0, -1.8) {$x_1 x_2^2$};
\begin{scope}[every edge/.style={draw=red,very thick}]
\path[->] (3) edge (1) (3) edge (2);
\end{scope}
\begin{scope}[every edge/.style={draw=blue,very thick}]
\path[->] (1) edge (2);
\end{scope}
\end{tikzpicture}%
%EndExpansion
$ & \multicolumn{1}{|c|}{$%
%TCIMACRO{\TeXButton{tikz tournament}{\begin{tikzpicture}
%\draw[draw=black] (-2, -1.4) rectangle (2, 2.2);
%\begin{scope}[every node/.style={circle,thick,draw=green!60!black}]
%\node(1) at (90:1.5) {$1$};
%\node(2) at (210:1.5) {$2$};
%\node(3) at (330:1.5) {$3$};
%\end{scope}
%\node(u1) at (0, -1.8) {$x_1 x_2 x_3$};
%\begin{scope}[every edge/.style={draw=red,very thick}]
%\path[->] (2) edge (1) (3) edge (2);
%\end{scope}
%\begin{scope}[every edge/.style={draw=blue,very thick}]
%\path[->] (1) edge (3);
%\end{scope}
%\end{tikzpicture}}}%
%BeginExpansion
\begin{tikzpicture}
\draw[draw=black] (-2, -1.4) rectangle (2, 2.2);
\begin{scope}[every node/.style={circle,thick,draw=green!60!black}]
\node(1) at (90:1.5) {$1$};
\node(2) at (210:1.5) {$2$};
\node(3) at (330:1.5) {$3$};
\end{scope}
\node(u1) at (0, -1.8) {$x_1 x_2 x_3$};
\begin{scope}[every edge/.style={draw=red,very thick}]
\path[->] (2) edge (1) (3) edge (2);
\end{scope}
\begin{scope}[every edge/.style={draw=blue,very thick}]
\path[->] (1) edge (3);
\end{scope}
\end{tikzpicture}%
%EndExpansion
$}\\\hline
$%
%TCIMACRO{\TeXButton{tikz tournament}{\begin{tikzpicture}
%\draw[draw=black] (-2, -1.4) rectangle (2, 2.2);
%\begin{scope}[every node/.style={circle,thick,draw=green!60!black}]
%\node(1) at (90:1.5) {$1$};
%\node(2) at (210:1.5) {$2$};
%\node(3) at (330:1.5) {$3$};
%\end{scope}
%\node(u1) at (0, -1.8) {$x_1^2 x_3$};
%\begin{scope}[every edge/.style={draw=red,very thick}]
%\path[->] (2) edge (1) (3) edge (1);
%\end{scope}
%\begin{scope}[every edge/.style={draw=blue,very thick}]
%\path[->] (2) edge (3);
%\end{scope}
%\end{tikzpicture}}}%
%BeginExpansion
\begin{tikzpicture}
\draw[draw=black] (-2, -1.4) rectangle (2, 2.2);
\begin{scope}[every node/.style={circle,thick,draw=green!60!black}]
\node(1) at (90:1.5) {$1$};
\node(2) at (210:1.5) {$2$};
\node(3) at (330:1.5) {$3$};
\end{scope}
\node(u1) at (0, -1.8) {$x_1^2 x_3$};
\begin{scope}[every edge/.style={draw=red,very thick}]
\path[->] (2) edge (1) (3) edge (1);
\end{scope}
\begin{scope}[every edge/.style={draw=blue,very thick}]
\path[->] (2) edge (3);
\end{scope}
\end{tikzpicture}%
%EndExpansion
$ & $%
%TCIMACRO{\TeXButton{tikz tournament}{\begin{tikzpicture}
%\draw[draw=black] (-2, -1.4) rectangle (2, 2.2);
%\begin{scope}[every node/.style={circle,thick,draw=green!60!black}]
%\node(1) at (90:1.5) {$1$};
%\node(2) at (210:1.5) {$2$};
%\node(3) at (330:1.5) {$3$};
%\end{scope}
%\node(u1) at (0, -1.8) {$- x_2^2 x_3$};
%\begin{scope}[every edge/.style={draw=red,very thick}]
%\path[->] (3) edge (2);
%\end{scope}
%\begin{scope}[every edge/.style={draw=blue,very thick}]
%\path[->] (1) edge (2) (1) edge (3);
%\end{scope}
%\end{tikzpicture}}}%
%BeginExpansion
\begin{tikzpicture}
\draw[draw=black] (-2, -1.4) rectangle (2, 2.2);
\begin{scope}[every node/.style={circle,thick,draw=green!60!black}]
\node(1) at (90:1.5) {$1$};
\node(2) at (210:1.5) {$2$};
\node(3) at (330:1.5) {$3$};
\end{scope}
\node(u1) at (0, -1.8) {$- x_2^2 x_3$};
\begin{scope}[every edge/.style={draw=red,very thick}]
\path[->] (3) edge (2);
\end{scope}
\begin{scope}[every edge/.style={draw=blue,very thick}]
\path[->] (1) edge (2) (1) edge (3);
\end{scope}
\end{tikzpicture}%
%EndExpansion
$ & \multicolumn{1}{|c|}{$%
%TCIMACRO{\TeXButton{tikz tournament}{\begin{tikzpicture}
%\draw[draw=black] (-2, -1.4) rectangle (2, 2.2);
%\begin{scope}[every node/.style={circle,thick,draw=green!60!black}]
%\node(1) at (90:1.5) {$1$};
%\node(2) at (210:1.5) {$2$};
%\node(3) at (330:1.5) {$3$};
%\end{scope}
%\node(u1) at (0, -1.8) {$- x_1 x_2 x_3$};
%\begin{scope}[every edge/.style={draw=red,very thick}]
%\path[->] (3) edge (1);
%\end{scope}
%\begin{scope}[every edge/.style={draw=blue,very thick}]
%\path[->] (1) edge (2) (2) edge (3);
%\end{scope}
%\end{tikzpicture}}}%
%BeginExpansion
\begin{tikzpicture}
\draw[draw=black] (-2, -1.4) rectangle (2, 2.2);
\begin{scope}[every node/.style={circle,thick,draw=green!60!black}]
\node(1) at (90:1.5) {$1$};
\node(2) at (210:1.5) {$2$};
\node(3) at (330:1.5) {$3$};
\end{scope}
\node(u1) at (0, -1.8) {$- x_1 x_2 x_3$};
\begin{scope}[every edge/.style={draw=red,very thick}]
\path[->] (3) edge (1);
\end{scope}
\begin{scope}[every edge/.style={draw=blue,very thick}]
\path[->] (1) edge (2) (2) edge (3);
\end{scope}
\end{tikzpicture}%
%EndExpansion
$}\\\hline
$%
%TCIMACRO{\TeXButton{tikz tournament}{\begin{tikzpicture}
%\draw[draw=black] (-2, -1.4) rectangle (2, 2.2);
%\begin{scope}[every node/.style={circle,thick,draw=green!60!black}]
%\node(1) at (90:1.5) {$1$};
%\node(2) at (210:1.5) {$2$};
%\node(3) at (330:1.5) {$3$};
%\end{scope}
%\node(u1) at (0, -1.8) {$- x_1 x_3^2$};
%\begin{scope}[every edge/.style={draw=red,very thick}]
%\path[->] (2) edge (1);
%\end{scope}
%\begin{scope}[every edge/.style={draw=blue,very thick}]
%\path[->] (1) edge (3) (2) edge (3);
%\end{scope}
%\end{tikzpicture}}}%
%BeginExpansion
\begin{tikzpicture}
\draw[draw=black] (-2, -1.4) rectangle (2, 2.2);
\begin{scope}[every node/.style={circle,thick,draw=green!60!black}]
\node(1) at (90:1.5) {$1$};
\node(2) at (210:1.5) {$2$};
\node(3) at (330:1.5) {$3$};
\end{scope}
\node(u1) at (0, -1.8) {$- x_1 x_3^2$};
\begin{scope}[every edge/.style={draw=red,very thick}]
\path[->] (2) edge (1);
\end{scope}
\begin{scope}[every edge/.style={draw=blue,very thick}]
\path[->] (1) edge (3) (2) edge (3);
\end{scope}
\end{tikzpicture}%
%EndExpansion
$ & $%
%TCIMACRO{\TeXButton{tikz tournament}{\begin{tikzpicture}
%\draw[draw=black] (-2, -1.4) rectangle (2, 2.2);
%\begin{scope}[every node/.style={circle,thick,draw=green!60!black}]
%\node(1) at (90:1.5) {$1$};
%\node(2) at (210:1.5) {$2$};
%\node(3) at (330:1.5) {$3$};
%\end{scope}
%\node(u1) at (0, -1.8) {$x_2 x_3^2$};
%\begin{scope}[every edge/.style={draw=red,very thick}]
%\path[->] (1) edge (2) (1) edge (3) (2) edge (3);
%\end{scope}
%\end{tikzpicture}}}%
%BeginExpansion
\begin{tikzpicture}
\draw[draw=black] (-2, -1.4) rectangle (2, 2.2);
\begin{scope}[every node/.style={circle,thick,draw=green!60!black}]
\node(1) at (90:1.5) {$1$};
\node(2) at (210:1.5) {$2$};
\node(3) at (330:1.5) {$3$};
\end{scope}
\node(u1) at (0, -1.8) {$x_2 x_3^2$};
\begin{scope}[every edge/.style={draw=red,very thick}]
\path[->] (1) edge (2) (1) edge (3) (2) edge (3);
\end{scope}
\end{tikzpicture}%
%EndExpansion
$ & \\\cline{1-2}%
\end{tabular}
\
\]

\end{example}

The equality (\ref{eq.def.tour1n.wD.sign}) can be rewritten as follows:

\begin{proposition}
\label{prop.tour1n.sign1}Let $D\in\mathcal{T}$ be a tournament. Then,%
\[
\operatorname*{sign}D=\left(  -1\right)  ^{\left(  \#\text{ of decreasing arcs
of }D\right)  }.
\]

\end{proposition}

\begin{proof}
Let us simplify the product $\prod_{\substack{\left(  i,j\right)  \text{ is
an}\\\text{arc of }D}}\left(  -1\right)  ^{\left[  i>j\right]  }$. The factor
$\left(  -1\right)  ^{\left[  i>j\right]  }$ of this product equals $1$ if the
arc $\left(  i,j\right)  $ is increasing (because in this case, we have $i<j$,
and thus $\left[  i>j\right]  =0$, and therefore $\left(  -1\right)  ^{\left[
i>j\right]  }=\left(  -1\right)  ^{0}=1$), and equals $-1$ if the arc $\left(
i,j\right)  $ is decreasing (because in this case, we have $i>j$, and thus
$\left[  i>j\right]  =1$, and therefore $\left(  -1\right)  ^{\left[
i>j\right]  }=\left(  -1\right)  ^{1}=-1$). Since any arc $\left(  i,j\right)
$ of $D$ is either increasing or decreasing (but cannot be both at the same
time), we thus conclude that%
\begin{align*}
\prod_{\substack{\left(  i,j\right)  \text{ is an}\\\text{arc of }D}}\left(
-1\right)  ^{\left[  i>j\right]  }  &  =\underbrace{\left(  \prod
_{\substack{\left(  i,j\right)  \text{ is an}\\\text{increasing}\\\text{arc of
}D}}1\right)  }_{=1}\cdot\underbrace{\left(  \prod_{\substack{\left(
i,j\right)  \text{ is a}\\\text{decreasing}\\\text{arc of }D}}\left(
-1\right)  \right)  }_{=\left(  -1\right)  ^{\left(  \#\text{ of decreasing
arcs of }D\right)  }}\\
&  =\left(  -1\right)  ^{\left(  \#\text{ of decreasing arcs of }D\right)  }.
\end{align*}
Hence, (\ref{eq.def.tour1n.wD.sign}) becomes%
\[
\operatorname*{sign}D=\prod_{\substack{\left(  i,j\right)  \text{ is
an}\\\text{arc of }D}}\left(  -1\right)  ^{\left[  i>j\right]  }=\left(
-1\right)  ^{\left(  \#\text{ of decreasing arcs of }D\right)  }.
\]
This proves Proposition \ref{prop.tour1n.sign1}.
\end{proof}

The equality (\ref{eq.def.tour1n.wD.eq}) can be rewritten as follows:

\begin{proposition}
\label{prop.tour1n.wD1}Let $D\in\mathcal{T}$ be a tournament. Then,%
\[
w\left(  D\right)  =\left(  \operatorname*{sign}D\right)  \cdot\prod_{j=1}%
^{n}x_{j}^{\deg^{-}j},
\]
where $\deg^{-}j$ denotes the indegree of $j$ as a vertex of $D$.
\end{proposition}

\begin{proof}
From (\ref{eq.def.tour1n.wD.eq}), we obtain%
\[
w\left(  D\right)  =\prod_{\substack{\left(  i,j\right)  \text{ is
an}\\\text{arc of }D}}\left(  \left(  -1\right)  ^{\left[  i>j\right]  }%
x_{j}\right)  =\left(  \prod_{\substack{\left(  i,j\right)  \text{ is
an}\\\text{arc of }D}}\left(  -1\right)  ^{\left[  i>j\right]  }\right)
\left(  \prod_{\substack{\left(  i,j\right)  \text{ is an}\\\text{arc of }%
D}}x_{j}\right)  .
\]


Now, let us consider the product $\prod_{\substack{\left(  i,j\right)  \text{
is an}\\\text{arc of }D}}x_{j}$. How often does the factor $x_{j}$ (for a
given $j\in\left\{  1,2,\ldots,n\right\}  $) appear in this product? It
appears once for each arc of $D$ whose target is $j$. Thus, in total, it
appears $\deg^{-}j$ many times (since $\deg^{-}j$ is the $\#$ of arcs of $D$
whose target is $j$). Hence, the product $\prod_{\substack{\left(  i,j\right)
\text{ is an}\\\text{arc of }D}}x_{j}$ contains each $x_{j}$ exactly $\deg
^{-}j$ many times (and contains no further factors). Consequently,%
\begin{equation}
\prod_{\substack{\left(  i,j\right)  \text{ is an}\\\text{arc of }D}%
}x_{j}=\prod_{j=1}^{n}x_{j}^{\deg^{-}j}. \label{pf.prop.tour1n.wD1.mon}%
\end{equation}


Now,%
\[
w\left(  D\right)  =\underbrace{\left(  \prod_{\substack{\left(  i,j\right)
\text{ is an}\\\text{arc of }D}}\left(  -1\right)  ^{\left[  i>j\right]
}\right)  }_{\substack{=\operatorname*{sign}D\\\text{(by
(\ref{eq.def.tour1n.wD.sign}))}}}\underbrace{\left(  \prod_{\substack{\left(
i,j\right)  \text{ is an}\\\text{arc of }D}}x_{j}\right)  }_{\substack{=\prod
_{j=1}^{n}x_{j}^{\deg^{-}j}\\\text{(by (\ref{pf.prop.tour1n.wD1.mon}))}%
}}=\left(  \operatorname*{sign}D\right)  \cdot\prod_{j=1}^{n}x_{j}^{\deg^{-}%
j}.
\]
This proves Proposition \ref{prop.tour1n.wD1}.
\end{proof}

On the other hand, the x-weights of the tournaments $D\in\mathcal{T}$ are
precisely the terms that appear in the expansion of the product $\prod_{1\leq
i<j\leq n}\left(  x_{j}-x_{i}\right)  $:

\begin{proposition}
\label{prop.tour1n.prodxjxi}We have%
\[
\prod_{1\leq i<j\leq n}\left(  x_{j}-x_{i}\right)  =\sum_{D\in\mathcal{T}%
}w\left(  D\right)  .
\]

\end{proposition}

We shall derive this proposition from the following more general formula:

\begin{lemma}
\label{lem.tour1n.prody}For each pair $\left(  i,j\right)  \in\left\{
1,2,\ldots,n\right\}  ^{2}$, let $y_{\left(  i,j\right)  }$ be a number. Then,%
\[
\prod_{1\leq i<j\leq n}\left(  y_{\left(  i,j\right)  }+y_{\left(  j,i\right)
}\right)  =\sum_{D\in\mathcal{T}}\ \ \prod_{\substack{a\text{ is
an}\\\text{arc of }D}}y_{a}.
\]

\end{lemma}

\begin{proof}
[Proof of Lemma \ref{lem.tour1n.prody}.]The idea is to expand the product
$\prod_{1\leq i<j\leq n}\left(  y_{\left(  i,j\right)  }+y_{\left(
j,i\right)  }\right)  $ (which has $n\left(  n-1\right)  /2$ many factors)
into a huge sum (a sum of $2^{n\left(  n-1\right)  /2}$ many addends), and to
match up the resulting addends with the tournaments $D\in\mathcal{T}$.

Before we explain this in the general case, let us first explore the case
$n=3$ as an example. In this case, we have%
\begin{align*}
&  \prod_{1\leq i<j\leq n}\left(  y_{\left(  i,j\right)  }+y_{\left(
j,i\right)  }\right) \\
&  =\prod_{1\leq i<j\leq3}\left(  y_{\left(  i,j\right)  }+y_{\left(
j,i\right)  }\right) \\
&  =\left(  y_{\left(  1,2\right)  }+y_{\left(  2,1\right)  }\right)  \left(
y_{\left(  1,3\right)  }+y_{\left(  3,1\right)  }\right)  \left(  y_{\left(
2,3\right)  }+y_{\left(  3,2\right)  }\right) \\
&  =y_{\left(  1,2\right)  }y_{\left(  1,3\right)  }y_{\left(  2,3\right)
}+y_{\left(  1,2\right)  }y_{\left(  1,3\right)  }y_{\left(  3,2\right)
}+y_{\left(  1,2\right)  }y_{\left(  3,1\right)  }y_{\left(  2,3\right)
}+y_{\left(  1,2\right)  }y_{\left(  3,1\right)  }y_{\left(  3,2\right)  }\\
&  \ \ \ \ \ \ \ \ \ \ +y_{\left(  2,1\right)  }y_{\left(  1,3\right)
}y_{\left(  2,3\right)  }+y_{\left(  2,1\right)  }y_{\left(  1,3\right)
}y_{\left(  3,2\right)  }+y_{\left(  2,1\right)  }y_{\left(  3,1\right)
}y_{\left(  2,3\right)  }+y_{\left(  2,1\right)  }y_{\left(  3,1\right)
}y_{\left(  3,2\right)  }.
\end{align*}
The right hand side of this is a sum of $8$ addends, and each of these addends
has the form $\prod_{\substack{a\text{ is an}\\\text{arc of }D}}y_{a}$ for
some tournament $D\in\mathcal{T}$. For example, the addend $y_{\left(
2,1\right)  }y_{\left(  1,3\right)  }y_{\left(  3,2\right)  }$ comes from the
tournament $D\in\mathcal{T}$ whose arcs are $\left(  2,1\right)  $, $\left(
1,3\right)  $ and $\left(  3,2\right)  $.

\begin{fineprint}
Here is the argument in the general case:

We first recall that if $F$ is any finite set, and if $\left(  a_{f}\right)
_{f\in F}$ and $\left(  b_{f}\right)  _{f\in F}$ are two families of numbers,
then%
\[
\prod_{f\in F}\left(  a_{f}+b_{f}\right)  =\sum_{S\text{ is a subset of }%
F}\ \ \prod_{f\in F}%
\begin{cases}
a_{f}, & \text{if }f\in S;\\
b_{f}, & \text{if }f\notin S.
\end{cases}
\]
Applying this fact to $F=\left\{  \text{increasing pairs}\right\}  $ and
$a_{\left(  i,j\right)  }=y_{\left(  i,j\right)  }$ and $b_{\left(
i,j\right)  }=y_{\left(  j,i\right)  }$, we obtain%
\begin{align*}
&  \prod_{\left(  i,j\right)  \in\left\{  \text{increasing pairs}\right\}
}\left(  y_{\left(  i,j\right)  }+y_{\left(  j,i\right)  }\right) \\
&  =\sum_{\substack{S\text{ is a set of}\\\text{increasing pairs}}%
}\ \ \prod_{\left(  i,j\right)  \in\left\{  \text{increasing pairs}\right\}  }%
\begin{cases}
y_{\left(  i,j\right)  }, & \text{if }\left(  i,j\right)  \in S;\\
y_{\left(  j,i\right)  }, & \text{if }\left(  i,j\right)  \notin S.
\end{cases}
\end{align*}
Recalling the definition of an increasing pair, we can rewrite this as
follows:%
\begin{align}
&  \prod_{1\leq i<j\leq n}\left(  y_{\left(  i,j\right)  }+y_{\left(
j,i\right)  }\right) \nonumber\\
&  =\sum_{\substack{S\text{ is a set of}\\\text{increasing pairs}}%
}\ \ \prod_{1\leq i<j\leq n}%
\begin{cases}
y_{\left(  i,j\right)  }, & \text{if }\left(  i,j\right)  \in S;\\
y_{\left(  j,i\right)  }, & \text{if }\left(  i,j\right)  \notin S.
\end{cases}
\label{pf.lem.tour1n.prody.3}%
\end{align}


In the discussion following Example \ref{exa.tour1n.3}, we have found a
bijection from the set $\mathcal{T}$ to the set of all sets of increasing
pairs. Specifically, this bijection sends each tournament $D\in\mathcal{T}$ to
the set of all increasing arcs of $D$. Let us denote this bijection by $\Phi$.
Let us now substitute $\Phi\left(  D\right)  $ for $S$ in the sum on the right
hand side of (\ref{pf.lem.tour1n.prody.3}). We thus obtain%
\begin{align*}
&  \sum_{\substack{S\text{ is a set of}\\\text{increasing pairs}}%
}\ \ \prod_{1\leq i<j\leq n}%
\begin{cases}
y_{\left(  i,j\right)  }, & \text{if }\left(  i,j\right)  \in S;\\
y_{\left(  j,i\right)  }, & \text{if }\left(  i,j\right)  \notin S
\end{cases}
\\
&  =\sum_{D\in\mathcal{T}}\ \ \prod_{1\leq i<j\leq n}%
\begin{cases}
y_{\left(  i,j\right)  }, & \text{if }\left(  i,j\right)  \in\Phi\left(
D\right)  ;\\
y_{\left(  j,i\right)  }, & \text{if }\left(  i,j\right)  \notin\Phi\left(
D\right)  .
\end{cases}
\end{align*}
Hence, we can rewrite (\ref{pf.lem.tour1n.prody.3}) as%
\begin{align}
&  \prod_{1\leq i<j\leq n}\left(  y_{\left(  i,j\right)  }+y_{\left(
j,i\right)  }\right) \nonumber\\
&  =\sum_{D\in\mathcal{T}}\ \ \prod_{1\leq i<j\leq n}%
\begin{cases}
y_{\left(  i,j\right)  }, & \text{if }\left(  i,j\right)  \in\Phi\left(
D\right)  ;\\
y_{\left(  j,i\right)  }, & \text{if }\left(  i,j\right)  \notin\Phi\left(
D\right)  .
\end{cases}
\label{pf.lem.tour1n.prody.4}%
\end{align}


Now, let $D\in\mathcal{T}$ be a tournament. Recall that $\Phi\left(  D\right)
$ is the set of all increasing arcs of $D$ (by the definition of $\Phi$).
Thus,
\[
\prod_{\substack{1\leq i<j\leq n;\\\left(  i,j\right)  \in\Phi\left(
D\right)  }}y_{\left(  i,j\right)  }=\prod_{\substack{1\leq i<j\leq
n;\\\left(  i,j\right)  \text{ is an increasing arc of }D}}y_{\left(
i,j\right)  }=\prod_{\left(  i,j\right)  \text{ is an increasing arc of }%
D}y_{\left(  i,j\right)  }%
\]
(since each increasing arc $\left(  i,j\right)  $ of $D$ automatically
satisfies $1\leq i<j\leq n$).

Recall again that $\Phi\left(  D\right)  $ is the set of all increasing arcs
of $D$. Thus,%
\[
\prod_{\substack{1\leq i<j\leq n;\\\left(  i,j\right)  \notin\Phi\left(
D\right)  }}y_{\left(  j,i\right)  }=\prod_{\substack{1\leq i<j\leq
n;\\\left(  i,j\right)  \text{ is not an increasing arc of }D}}y_{\left(
j,i\right)  }=\prod_{\substack{1\leq i<j\leq n;\\\left(  i,j\right)  \text{ is
not an arc of }D}}y_{\left(  j,i\right)  }%
\]
(here, we have dropped the \textquotedblleft increasing\textquotedblright%
\ condition under the product sign, since any arc $\left(  i,j\right)  $
satisfying $1\leq i<j\leq n$ is automatically increasing). However, if $i$ and
$j$ are two integers satisfying $1\leq i<j\leq n$, then the condition
\textquotedblleft$\left(  i,j\right)  $ is not an arc of $D$\textquotedblright%
\ is equivalent to the condition \textquotedblleft$\left(  j,i\right)  $ is an
arc of $D$\textquotedblright\ (by the tournament axiom, since the vertices $i$
and $j$ of $D$ are distinct\footnote{because $i<j$}). Hence, we can replace
the condition \textquotedblleft$\left(  i,j\right)  $ is not an arc of
$D$\textquotedblright\ under the summation sign $\prod_{\substack{1\leq
i<j\leq n;\\\left(  i,j\right)  \text{ is not an arc of }D}}$ by the
equivalent condition \textquotedblleft$\left(  j,i\right)  $ is an arc of
$D$\textquotedblright. We thus obtain%
\begin{align*}
\prod_{\substack{1\leq i<j\leq n;\\\left(  i,j\right)  \text{ is not an arc of
}D}}y_{\left(  j,i\right)  }  &  =\prod_{\substack{1\leq i<j\leq n;\\\left(
j,i\right)  \text{ is an arc of }D}}y_{\left(  j,i\right)  }\\
&  =\prod_{\substack{1\leq j<i\leq n;\\\left(  i,j\right)  \text{ is an arc of
}D}}y_{\left(  i,j\right)  }\ \ \ \ \ \ \ \ \ \ \left(
\begin{array}
[c]{c}%
\text{here, we have renamed the}\\
\text{index }\left(  i,j\right)  \text{ as }\left(  j,i\right)
\end{array}
\right) \\
&  =\prod_{\substack{\left(  i,j\right)  \text{ is an arc of }D;\\j<i}%
}y_{\left(  i,j\right)  }=\prod_{\left(  i,j\right)  \text{ is a decreasing
arc of }D}y_{\left(  i,j\right)  }%
\end{align*}
(because an arc $\left(  i,j\right)  $ of $D$ satisfying $j<i$ is the same as
a decreasing arc of $D$). Now,%
\begin{align}
&  \prod_{1\leq i<j\leq n}%
\begin{cases}
y_{\left(  i,j\right)  }, & \text{if }\left(  i,j\right)  \in\Phi\left(
D\right)  ;\\
y_{\left(  j,i\right)  }, & \text{if }\left(  i,j\right)  \notin\Phi\left(
D\right)
\end{cases}
\nonumber\\
&  =\underbrace{\left(  \prod_{\substack{1\leq i<j\leq n;\\\left(  i,j\right)
\in\Phi\left(  D\right)  }}y_{\left(  i,j\right)  }\right)  }_{=\prod_{\left(
i,j\right)  \text{ is an increasing arc of }D}y_{\left(  i,j\right)  }}%
\cdot\underbrace{\left(  \prod_{\substack{1\leq i<j\leq n;\\\left(
i,j\right)  \notin\Phi\left(  D\right)  }}y_{\left(  j,i\right)  }\right)
}_{\substack{=\prod_{\substack{1\leq i<j\leq n;\\\left(  i,j\right)  \text{ is
not an arc of }D}}y_{\left(  j,i\right)  }\\=\prod_{\left(  i,j\right)  \text{
is a decreasing arc of }D}y_{\left(  i,j\right)  }}}\nonumber\\
&  =\left(  \prod_{\left(  i,j\right)  \text{ is an increasing arc of }%
D}y_{\left(  i,j\right)  }\right)  \cdot\left(  \prod_{\left(  i,j\right)
\text{ is a decreasing arc of }D}y_{\left(  i,j\right)  }\right) \nonumber\\
&  =\prod_{\left(  i,j\right)  \text{ is an arc of }D}y_{\left(  i,j\right)  }
\label{pf.lem.tour1n.prody.7}%
\end{align}
(since any arc $\left(  i,j\right)  $ of $D$ is either increasing or
decreasing, but cannot be both at the same time).

Forget that we fixed $D$. We thus have proved (\ref{pf.lem.tour1n.prody.7})
for each $D\in\mathcal{T}$. Thus, (\ref{pf.lem.tour1n.prody.4}) becomes%
\begin{align*}
\prod_{1\leq i<j\leq n}\left(  y_{\left(  i,j\right)  }+y_{\left(  j,i\right)
}\right)   &  =\sum_{D\in\mathcal{T}}\ \ \underbrace{\prod_{1\leq i<j\leq n}%
\begin{cases}
y_{\left(  i,j\right)  }, & \text{if }\left(  i,j\right)  \in\Phi\left(
D\right)  ;\\
y_{\left(  j,i\right)  }, & \text{if }\left(  i,j\right)  \notin\Phi\left(
D\right)
\end{cases}
}_{\substack{=\prod_{\left(  i,j\right)  \text{ is an arc of }D}y_{\left(
i,j\right)  }\\\text{(by (\ref{pf.lem.tour1n.prody.7}))}}}\\
&  =\sum_{D\in\mathcal{T}}\ \ \underbrace{\prod_{\left(  i,j\right)  \text{ is
an arc of }D}y_{\left(  i,j\right)  }}_{=\prod_{\substack{a\text{ is
an}\\\text{arc of }D}}y_{a}}=\sum_{D\in\mathcal{T}}\ \ \prod
_{\substack{a\text{ is an}\\\text{arc of }D}}y_{a}.
\end{align*}
This proves Lemma \ref{lem.tour1n.prody}.
\end{fineprint}
\end{proof}

\begin{proof}
[Proof of Proposition \ref{prop.tour1n.prodxjxi}.]For each pair $\left(
i,j\right)  \in\left\{  1,2,\ldots,n\right\}  ^{2}$, we define a number%
\[
y_{\left(  i,j\right)  }:=\left(  -1\right)  ^{\left[  i>j\right]  }x_{j}.
\]
Then, Lemma \ref{lem.tour1n.prody} yields%
\begin{equation}
\prod_{1\leq i<j\leq n}\left(  y_{\left(  i,j\right)  }+y_{\left(  j,i\right)
}\right)  =\sum_{D\in\mathcal{T}}\ \ \prod_{\substack{a\text{ is
an}\\\text{arc of }D}}y_{a}. \label{pf.prop.tour1n.prodxjxi.1}%
\end{equation}


However, if $\left(  i,j\right)  $ is a pair of integers satisfying $1\leq
i<j\leq n$, then
\begin{align*}
y_{\left(  i,j\right)  }+y_{\left(  j,i\right)  }  &  =\left(  -1\right)
^{\left[  i>j\right]  }x_{j}+\left(  -1\right)  ^{\left[  j>i\right]  }%
x_{i}\ \ \ \ \ \ \ \ \ \ \left(  \text{by the definition of }y_{\left(
i,j\right)  }\text{ and of }y_{\left(  j,i\right)  }\right) \\
&  =\left(  -1\right)  ^{0}x_{j}+\left(  -1\right)  ^{1}x_{i}%
\ \ \ \ \ \ \ \ \ \ \left(
\begin{array}
[c]{c}%
\text{since }\left[  i>j\right]  =0\text{ (because we don't}\\
\text{have }i>j\text{ (since }i<j\text{)) and }\left[  j>i\right]  =1\\
\text{(since }j>i\text{ (because }i<j\text{))}%
\end{array}
\right) \\
&  =1x_{j}+\left(  -1\right)  x_{i}=x_{j}-x_{i}.
\end{align*}
Thus, we can rewrite (\ref{pf.prop.tour1n.prodxjxi.1}) as%
\begin{align*}
\prod_{1\leq i<j\leq n}\left(  x_{j}-x_{i}\right)   &  =\sum_{D\in\mathcal{T}%
}\ \ \underbrace{\prod_{\substack{a\text{ is an}\\\text{arc of }D}}y_{a}%
}_{=\prod_{\substack{\left(  i,j\right)  \text{ is an}\\\text{arc of }%
D}}y_{\left(  i,j\right)  }}=\sum_{D\in\mathcal{T}}\ \ \prod
_{\substack{\left(  i,j\right)  \text{ is an}\\\text{arc of }D}%
}\underbrace{y_{\left(  i,j\right)  }}_{\substack{=\left(  -1\right)
^{\left[  i>j\right]  }x_{j}\\\text{(by the definition}\\\text{of }y_{\left(
i,j\right)  }\text{)}}}\\
&  \ \ \ \ \ \ \ \ \ \ \ \ \ \ \ \ \ \ \ \ \left(
\begin{array}
[c]{c}%
\text{here, we have renamed the index }a\\
\text{as }\left(  i,j\right)  \text{ in the product}%
\end{array}
\right) \\
&  =\sum_{D\in\mathcal{T}}\ \ \underbrace{\prod_{\substack{\left(  i,j\right)
\text{ is an}\\\text{arc of }D}}\left(  \left(  -1\right)  ^{\left[
i>j\right]  }x_{j}\right)  }_{\substack{=w\left(  D\right)  \\\text{(by
(\ref{eq.def.tour1n.wD.eq}))}}}=\sum_{D\in\mathcal{T}}w\left(  D\right)  .
\end{align*}
This proves Proposition \ref{prop.tour1n.prodxjxi}.
\end{proof}

\subsection{Tournaments with no $3$-cycles}

Recall that our goal is to prove the equality (\ref{eq.thm.det.vander.eq}).
Proposition \ref{prop.tour1n.prodxjxi} interprets the right hand side of this
equality in terms of tournaments. We shall next find a similar interpretation
for its left hand side.

To this purpose, we shall study the tournaments $D\in\mathcal{T}$ that have no
$3$-cycles. As we will soon see, they have a rather specific form:

\begin{definition}
\label{def.Tsig.def}Let $\sigma\in S_{n}$ be a permutation. Then, we define a
digraph $T_{\sigma}$ by%
\[
T_{\sigma}:=\left(  \left\{  1,2,\ldots,n\right\}  ,\ \left\{  \left(
\sigma\left(  i\right)  ,\ \sigma\left(  j\right)  \right)  \ \mid\ i\text{
and }j\text{ are integers with }1\leq i<j\leq n\right\}  \right)  .
\]
Thus, the vertices of the digraph $T_{\sigma}$ are $1,2,\ldots,n$, and its
arcs are the pairs $\left(  \sigma\left(  i\right)  ,\ \sigma\left(  j\right)
\right)  $, where $i$ and $j$ range over integers satisfying $1\leq i<j\leq n$.
\end{definition}

\begin{example}
If $n=6$, and if $\sigma\in S_{6}$ is the permutation from Example
\ref{exa.perm.6}, then $T_{\sigma}$ is the following digraph (again, we draw
the increasing arcs blue and the decreasing arcs red):%
\[%
%TCIMACRO{\TeXButton{tikz tournament}{\begin{tikzpicture}
%\begin{scope}[every node/.style={circle,thick,draw=green!60!black}]
%\node(1) at (0:1.5) {$1$};
%\node(2) at (60:1.5) {$2$};
%\node(3) at (120:1.5) {$3$};
%\node(4) at (180:1.5) {$4$};
%\node(5) at (240:1.5) {$5$};
%\node(6) at (300:1.5) {$6$};
%\end{scope}
%\begin{scope}[every edge/.style={draw=red,very thick}]
%\path[->] (3) edge (2) edge (1);
%\path[->] (2) edge (1);
%\path[->] (6) edge (5) edge (4) edge (1);
%\path[->] (5) edge (4) edge (1);
%\path[->] (4) edge (1);
%\end{scope}
%\begin{scope}[every edge/.style={draw=blue,very thick}]
%\path[->] (3) edge (6) edge (5) edge (4);
%\path[->] (2) edge (6) edge (5) edge (4);
%\end{scope}
%\end{tikzpicture}}}%
%BeginExpansion
\begin{tikzpicture}
\begin{scope}[every node/.style={circle,thick,draw=green!60!black}]
\node(1) at (0:1.5) {$1$};
\node(2) at (60:1.5) {$2$};
\node(3) at (120:1.5) {$3$};
\node(4) at (180:1.5) {$4$};
\node(5) at (240:1.5) {$5$};
\node(6) at (300:1.5) {$6$};
\end{scope}
\begin{scope}[every edge/.style={draw=red,very thick}]
\path[->] (3) edge (2) edge (1);
\path[->] (2) edge (1);
\path[->] (6) edge (5) edge (4) edge (1);
\path[->] (5) edge (4) edge (1);
\path[->] (4) edge (1);
\end{scope}
\begin{scope}[every edge/.style={draw=blue,very thick}]
\path[->] (3) edge (6) edge (5) edge (4);
\path[->] (2) edge (6) edge (5) edge (4);
\end{scope}
\end{tikzpicture}%
%EndExpansion
\]

\end{example}

The usefulness of these digraphs $T_{\sigma}$ for us is due to the following theorem:

\begin{theorem}
\label{thm.Tsig.props}\ \ 

\begin{enumerate}
\item[\textbf{(a)}] For each permutation $\sigma\in S_{n}$, the digraph
$T_{\sigma}$ is a tournament in $\mathcal{T}$ and has no $3$-cycles.

\item[\textbf{(b)}] The tournaments $D\in\mathcal{T}$ that have no $3$-cycles
are precisely the digraphs of the form $T_{\sigma}$ with $\sigma\in S_{n}$.

\item[\textbf{(c)}] If $\sigma$ and $\tau$ are two distinct permutations in
$S_{n}$, then the digraphs $T_{\sigma}$ and $T_{\tau}$ are distinct. (In other
words, any permutation $\sigma\in S_{n}$ can be uniquely reconstructed from
$T_{\sigma}$.)

\item[\textbf{(d)}] Let $\sigma\in S_{n}$. Then,%
\begin{equation}
\operatorname*{sign}\left(  T_{\sigma}\right)  =\operatorname*{sign}%
\sigma\label{eq.thm.Tsig.props.d.sign}%
\end{equation}
and%
\begin{equation}
w\left(  T_{\sigma}\right)  =\operatorname*{sign}\sigma\cdot\prod_{i=1}%
^{n}x_{\sigma\left(  i\right)  }^{i-1}. \label{eq.thm.Tsig.props.d.w}%
\end{equation}
Also, each vertex $v$ of $T_{\sigma}$ has indegree%
\begin{equation}
\deg^{-}v=\sigma^{-1}\left(  v\right)  -1. \label{eq.thm.Tsig.props.d.deg-}%
\end{equation}

\end{enumerate}
\end{theorem}

\begin{proof}
[Proof of Theorem \ref{thm.Tsig.props}.]\textbf{(a)} Let $\sigma\in S_{n}$ be
a permutation. The arcs of the digraph $T_{\sigma}$ are the pairs $\left(
\sigma\left(  i\right)  ,\ \sigma\left(  j\right)  \right)  $, where $i$ and
$j$ range over integers satisfying $1\leq i<j\leq n$. Such an arc $\left(
\sigma\left(  i\right)  ,\ \sigma\left(  j\right)  \right)  $ cannot be a loop
(since $i<j$ entails $i\neq j$ and therefore $\sigma\left(  i\right)
\neq\sigma\left(  j\right)  $\ \ \ \ \footnote{because $\sigma$ is
injective}). Hence, the digraph $T_{\sigma}$ is loopless.

For any pair $\left(  u,v\right)  \in\left\{  1,2,\ldots,n\right\}  ^{2}$, we
have the following equivalence:%
\begin{equation}
\left(  \left(  u,v\right)  \text{ is an arc of }T_{\sigma}\right)
\Longleftrightarrow\left(  \sigma^{-1}\left(  u\right)  <\sigma^{-1}\left(
v\right)  \right)  . \label{pf.thm.Tsig.props.equivalence}%
\end{equation}


\begin{fineprint}
[\textit{Proof of (\ref{pf.thm.Tsig.props.equivalence}):} Let $\left(
u,v\right)  \in\left\{  1,2,\ldots,n\right\}  ^{2}$ be a pair. We must prove
the equivalence (\ref{pf.thm.Tsig.props.equivalence}). We shall prove its
\textquotedblleft$\Longrightarrow$\textquotedblright\ and \textquotedblleft%
$\Longleftarrow$\textquotedblright\ directions separately:

$\Longrightarrow:$ Assume that $\left(  u,v\right)  $ is an arc of $T_{\sigma
}$. We must prove that $\sigma^{-1}\left(  u\right)  <\sigma^{-1}\left(
v\right)  $.

The arcs of the digraph $T_{\sigma}$ are the pairs $\left(  \sigma\left(
i\right)  ,\ \sigma\left(  j\right)  \right)  $, where $i$ and $j$ range over
integers satisfying $1\leq i<j\leq n$. Hence, $\left(  u,v\right)  $ is such a
pair (since $\left(  u,v\right)  $ is an arc of $T_{\sigma}$). In other words,
$\left(  u,v\right)  =\left(  \sigma\left(  i\right)  ,\sigma\left(  j\right)
\right)  $ for some integers $i$ and $j$ satisfying $1\leq i<j\leq n$.
Consider these $i$ and $j$. From $\left(  u,v\right)  =\left(  \sigma\left(
i\right)  ,\sigma\left(  j\right)  \right)  $, we obtain $u=\sigma\left(
i\right)  $ and $v=\sigma\left(  j\right)  $. Thus, $i=\sigma^{-1}\left(
u\right)  $ and $j=\sigma^{-1}\left(  v\right)  $. Hence, the inequality $i<j$
(which we know to be true) can be rewritten as $\sigma^{-1}\left(  u\right)
<\sigma^{-1}\left(  v\right)  $. Thus, $\sigma^{-1}\left(  u\right)
<\sigma^{-1}\left(  v\right)  $ is true. This proves the \textquotedblleft%
$\Longrightarrow$\textquotedblright\ direction of the equivalence
(\ref{pf.thm.Tsig.props.equivalence}).

$\Longleftarrow:$ Assume that $\sigma^{-1}\left(  u\right)  <\sigma
^{-1}\left(  v\right)  $. We must prove that $\left(  u,v\right)  $ is an arc
of $T_{\sigma}$.

The arcs of the digraph $T_{\sigma}$ are the pairs $\left(  \sigma\left(
i\right)  ,\ \sigma\left(  j\right)  \right)  $, where $i$ and $j$ range over
integers satisfying $1\leq i<j\leq n$. Thus, in particular, one of these arcs
is $\left(  \sigma\left(  \sigma^{-1}\left(  u\right)  \right)  ,\ \sigma
\left(  \sigma^{-1}\left(  v\right)  \right)  \right)  $ (obtained by setting
$i=\sigma^{-1}\left(  u\right)  $ and $j=\sigma^{-1}\left(  v\right)  $),
since $\sigma^{-1}\left(  u\right)  $ and $\sigma^{-1}\left(  v\right)  $ are
two integers satisfying $1\leq\sigma^{-1}\left(  u\right)  <\sigma^{-1}\left(
v\right)  \leq n$. In other words, one of these arcs is $\left(  u,v\right)  $
(since $\sigma\left(  \sigma^{-1}\left(  u\right)  \right)  =u$ and
$\sigma\left(  \sigma^{-1}\left(  v\right)  \right)  =v$). Thus, $\left(
u,v\right)  $ is an arc of $T_{\sigma}$. This proves the \textquotedblleft%
$\Longleftarrow$\textquotedblright\ direction of the equivalence
(\ref{pf.thm.Tsig.props.equivalence}).

Thus, the proof of the equivalence (\ref{pf.thm.Tsig.props.equivalence}) is
complete.] \medskip
\end{fineprint}

Now, it is easy to see that this digraph $T_{\sigma}$ is a tournament.

\begin{fineprint}
[\textit{Proof:} Since we know that $T_{\sigma}$ is loopless, we only need to
verify the tournament axiom. In other words, we need to show that for any two
distinct vertices $u$ and $v$ of $T_{\sigma}$, \textbf{exactly} one of the two
pairs $\left(  u,v\right)  $ and $\left(  v,u\right)  $ is an arc of
$T_{\sigma}$.

Let $u$ and $v$ be two distinct vertices of $T_{\sigma}$. Thus, $u$ and $v$
are two distinct elements of $\left\{  1,2,\ldots,n\right\}  $. We must show
that \textbf{exactly} one of the two pairs $\left(  u,v\right)  $ and $\left(
v,u\right)  $ is an arc of $T_{\sigma}$.

We WLOG assume that $\sigma^{-1}\left(  u\right)  \leq\sigma^{-1}\left(
v\right)  $ (since otherwise, we can swap $u$ with $v$). Combining this with
$\sigma^{-1}\left(  u\right)  \neq\sigma^{-1}\left(  v\right)  $ (this follows
from $u\neq v$), we obtain $\sigma^{-1}\left(  u\right)  <\sigma^{-1}\left(
v\right)  $. Thus, $\left(  u,v\right)  $ is an arc of $T_{\sigma}$ (by
(\ref{pf.thm.Tsig.props.equivalence})). Moreover, we do \textbf{not} have
$\sigma^{-1}\left(  v\right)  <\sigma^{-1}\left(  u\right)  $ (since this
would contradict $\sigma^{-1}\left(  u\right)  <\sigma^{-1}\left(  v\right)
$). However, from (\ref{pf.thm.Tsig.props.equivalence}) (applied to $\left(
v,u\right)  $ instead of $\left(  u,v\right)  $), we obtain the equivalence%
\[
\left(  \left(  v,u\right)  \text{ is an arc of }T_{\sigma}\right)
\Longleftrightarrow\left(  \sigma^{-1}\left(  v\right)  <\sigma^{-1}\left(
u\right)  \right)  .
\]
Thus, $\left(  v,u\right)  $ is not an arc of $T_{\sigma}$ (since we do
\textbf{not} have $\sigma^{-1}\left(  v\right)  <\sigma^{-1}\left(  u\right)
$).

We now know that $\left(  u,v\right)  $ is an arc of $T_{\sigma}$, but
$\left(  v,u\right)  $ is not. Therefore, \textbf{exactly} one of the two
pairs $\left(  u,v\right)  $ and $\left(  v,u\right)  $ is an arc of
$T_{\sigma}$. This completes the proof of the tournament axiom for $T_{\sigma
}$. Hence, $T_{\sigma}$ is a tournament.] \medskip
\end{fineprint}

Since $T_{\sigma}$ is a tournament with vertex set $\left\{  1,2,\ldots
,n\right\}  $, we have $T_{\sigma}\in\mathcal{T}$. It remains to show that
$T_{\sigma}$ has no $3$-cycles.

Indeed, assume the contrary. Thus, $T_{\sigma}$ has a $3$-cycle $\left(
u,v,w\right)  $. Consider this $\left(  u,v,w\right)  $. Since $\left(
u,v,w\right)  $ is a $3$-cycle, all three pairs $uv$, $vw$ and $wu$ are arcs
of $T_{\sigma}$. In particular, $uv$ is an arc of $T_{\sigma}$. In other
words, $\left(  u,v\right)  $ is an arc of $T_{\sigma}$. By
(\ref{pf.thm.Tsig.props.equivalence}), we thus conclude that $\sigma
^{-1}\left(  u\right)  <\sigma^{-1}\left(  v\right)  $. Similarly,
$\sigma^{-1}\left(  v\right)  <\sigma^{-1}\left(  w\right)  $ and $\sigma
^{-1}\left(  w\right)  <\sigma^{-1}\left(  u\right)  $. Therefore,
$\sigma^{-1}\left(  u\right)  <\sigma^{-1}\left(  v\right)  <\sigma
^{-1}\left(  w\right)  <\sigma^{-1}\left(  u\right)  $, which is absurd. This
contradiction shows that our assumption was wrong. Hence, we have shown that
$T_{\sigma}$ has no $3$-cycles. This completes the proof of Theorem
\ref{thm.Tsig.props} \textbf{(a)}. \medskip

\textbf{(b)} Theorem \ref{thm.Tsig.props} \textbf{(a)} tells us that each
digraph of the form $T_{\sigma}$ with $\sigma\in S_{n}$ is a tournament
$D\in\mathcal{T}$ that has no $3$-cycles. It thus remains to prove the
converse: i.e., that each tournament $D\in\mathcal{T}$ that has no $3$-cycles
is a digraph of the form $T_{\sigma}$ with $\sigma\in S_{n}$.

So let $D\in\mathcal{T}$ be a tournament that has no $3$-cycles. We must find
a permutation $\sigma\in S_{n}$ such that $D=T_{\sigma}$.

We first show the following:

\begin{statement}
\textit{Claim 1:} Let $u$ and $v$ be two distinct vertices of $D$ such that
$\deg^{-}v\leq\deg^{-}u$. Then, $vu$ is an arc of $D$.
\end{statement}

[\textit{Proof of Claim 1:} Assume the contrary. Thus, $vu$ is not an arc of
$D$.

The tournament axiom shows that exactly one of the two pairs $uv$ and $vu$ is
an arc of $D$. Hence, $uv$ is an arc of $D$ (since $vu$ is not an arc of $D$).

Let $X$ be the set of all vertices $z$ of $D$ for which $zu$ is an arc of $D$.
Thus, the vertices in $X$ are in 1-to-1 correspondence with the arcs of $D$
that have target $u$. Hence, $\left\vert X\right\vert $ equals the number of
such arcs. But the latter number is $\deg^{-}u$ (by the definition of
indegrees). Hence, we have shown that $\left\vert X\right\vert =\deg^{-}u$.

Let $Y$ be the set of all vertices $z$ of $D$ for which $zv$ is an arc of $D$.
Then, $\left\vert Y\right\vert =\deg^{-}v$ (indeed, we can show this in the
same way as we showed $\left\vert X\right\vert =\deg^{-}u$).

The digraph $D$ is loopless (since it is a tournament); thus, $uu$ is not an
arc of $D$. In other words, we have $u\notin X$ (by the definition of $X$).
However, we have $u\in Y$ (since $uv$ is an arc of $D$). Hence, $X\neq Y$
(because if we had $X=Y$, then $u\notin X=Y$ would contradict $u\in Y$).

Now, $\left\vert Y\right\vert =\deg^{-}v\leq\deg^{-}u$, so that $\deg^{-}%
u\geq\left\vert Y\right\vert $. Thus, $\left\vert X\right\vert =\deg^{-}%
u\geq\left\vert Y\right\vert $. However, if $X$ was a subset of $Y$, then $X$
would be a \textbf{proper} subset of $Y$ (since $X\neq Y$), which would entail
$\left\vert X\right\vert <\left\vert Y\right\vert $; but this would contradict
$\left\vert X\right\vert \geq\left\vert Y\right\vert $. Thus, $X$ is not a
subset of $Y$. Hence, there exists a vertex $w\in X$ such that $w\notin Y$.
Consider this $w$.

Since $w\in X$, the pair $wu$ is an arc of $D$ (by the definition of $X$). As
a consequence, $w\neq u$ (since $uu$ is not an arc of $D$) and $w\neq v$
(since $vu$ is not an arc of $D$). Combining this with $u\neq v$, we see that
the three vertices $u$, $v$ and $w$ are distinct.

Since $w\notin Y$, the pair $wv$ is not an arc of $D$. However, $w\neq v$;
thus, the tournament axiom shows that exactly one of the two pairs $wv$ and
$vw$ is an arc of $D$. Since $wv$ is not an arc of $D$, we thus conclude that
$vw$ is an arc of $D$.

We now know that $u$, $v$ and $w$ are three distinct vertices of $D$ and that
$uv$, $vw$ and $wu$ are arcs of $D$. In other words, $\left(  u,v,w\right)  $
is a $3$-cycle. But this contradicts the fact that $D$ has no $3$-cycles. This
contradiction shows that our assumption was false; hence, Claim 1 is proven.]
\medskip

Pick a permutation $\tau\in S_{n}$ of $\left\{  1,2,\ldots,n\right\}  $ that
sorts the vertices of $D$ in the order of increasing outdegree -- i.e., that
satisfies%
\begin{equation}
\deg^{-}\left(  \tau\left(  1\right)  \right)  \leq\deg^{-}\left(  \tau\left(
2\right)  \right)  \leq\cdots\leq\deg^{-}\left(  \tau\left(  n\right)
\right)  . \label{pf.thm.Tsig.props.b.chain}%
\end{equation}
(Such a permutation $\tau$ exists, since we can sort any $n$ numbers in
increasing order.) We shall show that $D=T_{\tau}$.

Indeed, the digraphs $D$ and $T_{\tau}$ have the same vertex set (namely,
$\left\{  1,2,\ldots,n\right\}  $); thus, we only need to show that they have
the same arcs. To do so, we will first show the following two claims:

\begin{statement}
\textit{Claim 2:} Any arc of $T_{\tau}$ is an arc of $D$.
\end{statement}

[\textit{Proof of Claim 2:} Let $\left(  v,u\right)  $ be an arc of $T_{\tau}%
$. We must prove that $\left(  v,u\right)  $ is an arc of $D$.

However, the arcs of the digraph $T_{\tau}$ are the pairs $\left(  \tau\left(
i\right)  ,\ \tau\left(  j\right)  \right)  $, where $i$ and $j$ range over
integers satisfying $1\leq i<j\leq n$ (by the definition of $T_{\tau}$).
Hence, $\left(  v,u\right)  $ is such a pair (since $\left(  v,u\right)  $ is
an arc of $T_{\tau}$). In other words, $\left(  v,u\right)  =\left(
\tau\left(  i\right)  ,\tau\left(  j\right)  \right)  $ for some two integers
$i$ and $j$ satisfying $1\leq i<j\leq n$. Consider these $i$ and $j$. From
$\left(  v,u\right)  =\left(  \tau\left(  i\right)  ,\tau\left(  j\right)
\right)  $, we obtain $v=\tau\left(  i\right)  $ and $u=\tau\left(  j\right)
$. From $i<j$, we obtain $\deg^{-}\left(  \tau\left(  i\right)  \right)
\leq\deg^{-}\left(  \tau\left(  j\right)  \right)  $ (by
(\ref{pf.thm.Tsig.props.b.chain})). In other words, $\deg^{-}v\leq\deg^{-}u$
(since $v=\tau\left(  i\right)  $ and $u=\tau\left(  j\right)  $). Moreover,
the vertices $i$ and $j$ are distinct (since $i<j$); thus, the vertices
$\tau\left(  i\right)  $ and $\tau\left(  j\right)  $ are distinct as well
(since $\tau$ is injective). In other words, $v$ and $u$ are distinct (since
$v=\tau\left(  i\right)  $ and $u=\tau\left(  j\right)  $). In other words,
$u$ and $v$ are distinct. Hence, Claim 1 yields that $vu$ is an arc of $D$. In
other words, $\left(  v,u\right)  $ is an arc of $D$. This completes the proof
of Claim 2.]

\begin{statement}
\textit{Claim 3:} Any arc of $D$ is an arc of $T_{\tau}$.
\end{statement}

[\textit{Proof of Claim 3:} Let $\left(  u,v\right)  $ be an arc of $D$. We
must prove that $\left(  u,v\right)  $ is an arc of $T_{\tau}$.

Assume the contrary. Thus, $\left(  u,v\right)  $ is not an arc of $T_{\tau}$.

The digraph $D$ is loopless (since it is a tournament). Thus, its arc $\left(
u,v\right)  $ cannot be a loop. In other words, $u$ and $v$ are distinct.
Hence, exactly one of the two pairs $\left(  u,v\right)  $ and $\left(
v,u\right)  $ is an arc of $D$ (by the tournament axiom, since $D$ is a
tournament). Therefore, the pair $\left(  v,u\right)  $ is not an arc of $D$
(since $\left(  u,v\right)  $ is an arc of $D$).

However, $T_{\tau}$ is a tournament (by Theorem \ref{thm.Tsig.props}
\textbf{(a)}, applied to $\sigma=\tau$), and $u$ and $v$ are two distinct
vertices. Hence, exactly one of the two pairs $\left(  u,v\right)  $ and
$\left(  v,u\right)  $ is an arc of $T_{\tau}$ (by the tournament axiom).
Therefore, $\left(  v,u\right)  $ is an arc of $T_{\tau}$ (because $\left(
u,v\right)  $ is not an arc of $T_{\tau}$). By Claim 2, this entails that
$\left(  v,u\right)  $ is an arc of $D$. But this contradicts the fact that
$\left(  v,u\right)  $ is not an arc of $D$. This contradiction shows that our
assumption was false. Hence, $\left(  u,v\right)  $ is an arc of $T_{\tau}$.
This proves Claim 3.] \medskip

Claim 2 and Claim 3 (combined) yield that the arcs of $D$ are precisely the
arcs of $T_{\tau}$. In other words, the digraphs $D$ and $T_{\tau}$ have the
same arcs. Since they also have the same vertex set, we thus conclude that
they are equal. In other words, $D=T_{\tau}$. Hence, $D$ is a digraph of the
form $T_{\sigma}$ with $\sigma\in S_{n}$ (namely, $\sigma=\tau$).

Forget that we fixed $D$. We thus have shown that each tournament
$D\in\mathcal{T}$ that has no $3$-cycles is a digraph of the form $T_{\sigma}$
with $\sigma\in S_{n}$. This completes the proof of Theorem
\ref{thm.Tsig.props} \textbf{(b)}. \medskip

\textbf{(d)} Let $\sigma\in S_{n}$. Let $v$ be a vertex of $T_{\sigma}$. Thus,
$v\in\left\{  1,2,\ldots,n\right\}  $.

Set $k:=\sigma^{-1}\left(  v\right)  $. By the definition of an indegree, we
have%
\begin{align*}
&  \deg^{-}v\\
&  =\left(  \#\text{ of arcs of }T_{\sigma}\text{ whose target is }v\right) \\
&  =\left(  \#\text{ of arcs of }T_{\sigma}\text{ that have the form }\left(
u,v\right)  \text{ for some }u\in\left\{  1,2,\ldots,n\right\}  \right) \\
&  \ \ \ \ \ \ \ \ \ \ \ \ \ \ \ \ \ \ \ \ \left(
\begin{array}
[c]{c}%
\text{since an arc of }T_{\sigma}\text{ whose target is }v\text{ is the same
as an}\\
\text{arc of }T_{\sigma}\text{ that has the form }\left(  u,v\right)  \text{
for some }u\in\left\{  1,2,\ldots,n\right\}
\end{array}
\right) \\
&  =\left(  \#\text{ of }u\in\left\{  1,2,\ldots,n\right\}  \text{ such that
}\left(  u,v\right)  \text{ is an arc of }T_{\sigma}\right) \\
&  =\left(  \#\text{ of }u\in\left\{  1,2,\ldots,n\right\}  \text{ such that
}\sigma^{-1}\left(  u\right)  <\sigma^{-1}\left(  v\right)  \right) \\
&  \ \ \ \ \ \ \ \ \ \ \ \ \ \ \ \ \ \ \ \ \left(
\begin{array}
[c]{c}%
\text{here, we have replaced the condition \textquotedblleft}\left(
u,v\right)  \text{ is an arc of }T_{\sigma}\text{\textquotedblright}\\
\text{by the equivalent condition \textquotedblleft}\sigma^{-1}\left(
u\right)  <\sigma^{-1}\left(  v\right)  \text{\textquotedblright}\\
\text{(the equivalence follows from (\ref{pf.thm.Tsig.props.equivalence}))}%
\end{array}
\right) \\
&  =\left(  \#\text{ of }u\in\left\{  1,2,\ldots,n\right\}  \text{ such that
}\sigma^{-1}\left(  u\right)  <k\right)  \ \ \ \ \ \ \ \ \ \ \left(
\text{since }\sigma^{-1}\left(  v\right)  =k\right) \\
&  =\left(  \#\text{ of }j\in\left\{  1,2,\ldots,n\right\}  \text{ such that
}j<k\right) \\
&  \ \ \ \ \ \ \ \ \ \ \ \ \ \ \ \ \ \ \ \ \left(
\begin{array}
[c]{c}%
\text{here, we have substituted }j\text{ for }\sigma^{-1}\left(  u\right)
\text{, since the}\\
\text{map }\sigma^{-1}:\left\{  1,2,\ldots,n\right\}  \rightarrow\left\{
1,2,\ldots,n\right\}  \text{ is a bijection}%
\end{array}
\right) \\
&  =k-1
\end{align*}
(since the numbers $j\in\left\{  1,2,\ldots,n\right\}  $ such that $j<k$ are
precisely the numbers $1,2,\ldots,k-1$, and thus there are $k-1$ of them). In
view of $k=\sigma^{-1}\left(  v\right)  $, this rewrites as $\deg^{-}%
v=\sigma^{-1}\left(  v\right)  -1$. This proves
(\ref{eq.thm.Tsig.props.d.deg-}).

Forget that we fixed $v$. We thus have proved (\ref{eq.thm.Tsig.props.d.deg-})
for each vertex $v$ of $T_{\sigma}$.

Next, we shall prove (\ref{eq.thm.Tsig.props.d.sign}). Indeed, the definition
of $\operatorname*{sign}\left(  T_{\sigma}\right)  $ yields%
\[
\operatorname*{sign}\left(  T_{\sigma}\right)  =\prod_{\substack{\left(
i,j\right)  \text{ is an}\\\text{arc of }T_{\sigma}}}\left(  -1\right)
^{\left[  i>j\right]  }=\prod_{\substack{\left(  u,v\right)  \text{ is
an}\\\text{arc of }T_{\sigma}}}\left(  -1\right)  ^{\left[  u>v\right]  }%
\]
(here, we have renamed the index $\left(  i,j\right)  $ as $\left(
u,v\right)  $ in the product). However, the arcs of $T_{\sigma}$ are the pairs
$\left(  \sigma\left(  i\right)  ,\ \sigma\left(  j\right)  \right)  $, where
$i$ and $j$ range over integers satisfying $1\leq i<j\leq n$ (by the
definition of $T_{\sigma}$). Thus, we can rewrite the product $\prod
_{\substack{\left(  u,v\right)  \text{ is an}\\\text{arc of }T_{\sigma}%
}}\left(  -1\right)  ^{\left[  u>v\right]  }$ as follows:%
\[
\prod_{\substack{\left(  u,v\right)  \text{ is an}\\\text{arc of }T_{\sigma}%
}}\left(  -1\right)  ^{\left[  u>v\right]  }=\prod_{1\leq i<j\leq n}\left(
-1\right)  ^{\left[  \sigma\left(  i\right)  >\sigma\left(  j\right)  \right]
}%
\]
(here, we have tacitly used the fact that the pairs $\left(  \sigma\left(
i\right)  ,\ \sigma\left(  j\right)  \right)  $ for all pairs of integers
$\left(  i,j\right)  $ satisfying $1\leq i<j\leq n$ are distinct\footnote{This
is clear because $\sigma$ is injective.}, and therefore each arc of
$T_{\sigma}$ can be written in the form $\left(  \sigma\left(  i\right)
,\ \sigma\left(  j\right)  \right)  $ for a \textbf{unique} pair $\left(
i,j\right)  $). Hence,%
\begin{align*}
\operatorname*{sign}\left(  T_{\sigma}\right)   &  =\prod_{\substack{\left(
u,v\right)  \text{ is an}\\\text{arc of }T_{\sigma}}}\left(  -1\right)
^{\left[  u>v\right]  }=\prod_{1\leq i<j\leq n}\left(  -1\right)  ^{\left[
\sigma\left(  i\right)  >\sigma\left(  j\right)  \right]  }\\
&  =\left(  \prod_{\substack{1\leq i<j\leq n;\\\sigma\left(  i\right)
>\sigma\left(  j\right)  }}\left(  -1\right)  ^{\left[  \sigma\left(
i\right)  >\sigma\left(  j\right)  \right]  }\right)  \cdot\left(
\prod_{\substack{1\leq i<j\leq n;\\\text{\textbf{not} }\sigma\left(  i\right)
>\sigma\left(  j\right)  }}\left(  -1\right)  ^{\left[  \sigma\left(
i\right)  >\sigma\left(  j\right)  \right]  }\right)
\end{align*}
(since each pair $\left(  i,j\right)  $ in our product either satisfies
$\sigma\left(  i\right)  >\sigma\left(  j\right)  $ or doesn't). In view of
\begin{align*}
&  \prod_{\substack{1\leq i<j\leq n;\\\sigma\left(  i\right)  >\sigma\left(
j\right)  }}\underbrace{\left(  -1\right)  ^{\left[  \sigma\left(  i\right)
>\sigma\left(  j\right)  \right]  }}_{\substack{=\left(  -1\right)
^{1}\\\text{(since }\sigma\left(  i\right)  >\sigma\left(  j\right)
\\\text{and thus }\left[  \sigma\left(  i\right)  >\sigma\left(  j\right)
\right]  =1\text{)}}}\\
&  =\prod_{\substack{1\leq i<j\leq n;\\\sigma\left(  i\right)  >\sigma\left(
j\right)  }}\underbrace{\left(  -1\right)  ^{1}}_{=-1}=\prod_{\substack{1\leq
i<j\leq n;\\\sigma\left(  i\right)  >\sigma\left(  j\right)  }}\left(
-1\right) \\
&  =\left(  -1\right)  ^{\left(  \#\text{ of pairs }\left(  i,j\right)  \text{
of integers satisfying }1\leq i<j\leq n\text{ and }\sigma\left(  i\right)
>\sigma\left(  j\right)  \right)  }\\
&  =\left(  -1\right)  ^{\left(  \#\text{ of inversions of }\sigma\right)
}\ \ \ \ \ \ \ \ \ \ \left(
\begin{array}
[c]{c}%
\text{since the pairs }\left(  i,j\right)  \text{ of integers}\\
\text{satisfying }1\leq i<j\leq n\text{ and }\sigma\left(  i\right)
>\sigma\left(  j\right) \\
\text{are known as the inversions of }\sigma
\end{array}
\right) \\
&  =\left(  -1\right)  ^{\ell\left(  \sigma\right)  }%
\ \ \ \ \ \ \ \ \ \ \left(  \text{since the }\#\text{ of inversions of }%
\sigma\text{ is called }\ell\left(  \sigma\right)  \right) \\
&  =\operatorname*{sign}\sigma\ \ \ \ \ \ \ \ \ \ \left(  \text{because
}\operatorname*{sign}\sigma\text{ is defined to be }\left(  -1\right)
^{\ell\left(  \sigma\right)  }\right)
\end{align*}
and%
\[
\prod_{\substack{1\leq i<j\leq n;\\\text{\textbf{not} }\sigma\left(  i\right)
>\sigma\left(  j\right)  }}\underbrace{\left(  -1\right)  ^{\left[
\sigma\left(  i\right)  >\sigma\left(  j\right)  \right]  }}%
_{\substack{=\left(  -1\right)  ^{0}\\\text{(since we don't}\\\text{have
}\sigma\left(  i\right)  >\sigma\left(  j\right)  \text{, and thus}\\\text{we
have }\left[  \sigma\left(  i\right)  >\sigma\left(  j\right)  \right]
=0\text{)}}}=\prod_{\substack{1\leq i<j\leq n;\\\text{\textbf{not} }%
\sigma\left(  i\right)  >\sigma\left(  j\right)  }}\underbrace{\left(
-1\right)  ^{0}}_{=1}=1,
\]
we can simplify this to%
\[
\operatorname*{sign}\left(  T_{\sigma}\right)  =\underbrace{\left(
\prod_{\substack{1\leq i<j\leq n;\\\sigma\left(  i\right)  >\sigma\left(
j\right)  }}\left(  -1\right)  ^{\left[  \sigma\left(  i\right)
>\sigma\left(  j\right)  \right]  }\right)  }_{=\operatorname*{sign}\sigma
}\cdot\underbrace{\left(  \prod_{\substack{1\leq i<j\leq
n;\\\text{\textbf{not} }\sigma\left(  i\right)  >\sigma\left(  j\right)
}}\left(  -1\right)  ^{\left[  \sigma\left(  i\right)  >\sigma\left(
j\right)  \right]  }\right)  }_{=1}=\operatorname*{sign}\sigma.
\]
This proves (\ref{eq.thm.Tsig.props.d.sign}).

It remains to prove (\ref{eq.thm.Tsig.props.d.w}). We first consider the
product $\prod_{j=1}^{n}x_{j}^{\deg^{-}j}$, where $\deg^{-}j$ denotes the
indegree of $j$ as a vertex of $T_{\sigma}$. We can substitute $\sigma\left(
i\right)  $ for $j$ in this product (since $\sigma$ is a bijection from
$\left\{  1,2,\ldots,n\right\}  $ to $\left\{  1,2,\ldots,n\right\}  $). Thus,
we obtain%
\begin{equation}
\prod_{j=1}^{n}x_{j}^{\deg^{-}j}=\prod_{i=1}^{n}x_{\sigma\left(  i\right)
}^{\deg^{-}\left(  \sigma\left(  i\right)  \right)  }. \label{pf.thm.Tsig.d.6}%
\end{equation}
However, each $i\in\left\{  1,2,\ldots,n\right\}  $ satisfies%
\begin{align*}
\deg^{-}\left(  \sigma\left(  i\right)  \right)   &  =\underbrace{\sigma
^{-1}\left(  \sigma\left(  i\right)  \right)  }_{=i}%
-1\ \ \ \ \ \ \ \ \ \ \left(  \text{by (\ref{eq.thm.Tsig.props.d.deg-}),
applied to }v=\sigma\left(  i\right)  \right) \\
&  =i-1.
\end{align*}
In view of this, we can rewrite (\ref{pf.thm.Tsig.d.6}) as
\begin{equation}
\prod_{j=1}^{n}x_{j}^{\deg^{-}j}=\prod_{i=1}^{n}x_{\sigma\left(  i\right)
}^{i-1}. \label{pf.thm.Tsig.d.7}%
\end{equation}


Now, Proposition \ref{prop.tour1n.wD1} (applied to $D=T_{\sigma}$) yields%
\[
w\left(  T_{\sigma}\right)  =\underbrace{\left(  \operatorname*{sign}\left(
T_{\sigma}\right)  \right)  }_{\substack{=\operatorname*{sign}\sigma
\\\text{(by (\ref{eq.thm.Tsig.props.d.sign}))}}}\cdot\underbrace{\prod
_{j=1}^{n}x_{j}^{\deg^{-}j}}_{\substack{=\prod_{i=1}^{n}x_{\sigma\left(
i\right)  }^{i-1}\\\text{(by (\ref{pf.thm.Tsig.d.7}))}}}=\operatorname*{sign}%
\sigma\cdot\prod_{i=1}^{n}x_{\sigma\left(  i\right)  }^{i-1},
\]
and thus (\ref{eq.thm.Tsig.props.d.w}) is proven. This completes the proof of
Theorem \ref{thm.Tsig.props} \textbf{(d)}. \medskip

\textbf{(c)} Let $\sigma$ and $\tau$ be two distinct permutations in $S_{n}$.
We must prove that the digraphs $T_{\sigma}$ and $T_{\tau}$ are distinct.

Assume the contrary. Thus, $T_{\sigma}=T_{\tau}$. Hence, for each
$v\in\left\{  1,2,\ldots,n\right\}  $, the indegree of $v$ as a vertex of
$T_{\sigma}$ equals the indegree of $v$ as a vertex of $T_{\tau}$. Thus, we
can allow ourselves to denote both of these indegrees by $\deg^{-}v$. From
(\ref{eq.thm.Tsig.props.d.deg-}), we know that $\deg^{-}v=\sigma^{-1}\left(
v\right)  -1$. Similarly, $\deg^{-}v=\tau^{-1}\left(  v\right)  -1$. Comparing
these two equalities, we find $\sigma^{-1}\left(  v\right)  -1=\tau
^{-1}\left(  v\right)  -1$. Hence, $\sigma^{-1}\left(  v\right)  =\tau
^{-1}\left(  v\right)  $.

Forget that we fixed $v$. We thus have shown that $\sigma^{-1}\left(
v\right)  =\tau^{-1}\left(  v\right)  $ for each $v\in\left\{  1,2,\ldots
,n\right\}  $. In other words, $\sigma^{-1}=\tau^{-1}$. Hence, $\sigma=\tau$.
This contradicts the fact that $\sigma$ and $\tau$ are distinct. This
contradiction shows that our assumption was false. Thus, Theorem
\ref{thm.Tsig.props} \textbf{(c)} is proved.
\end{proof}

\begin{remark}
One alternative way to prove Theorem \ref{thm.Tsig.props} \textbf{(b)} uses
the fact (known as R\'{e}dei's Little Theorem) that any tournament has a
Hamiltonian path (see, e.g., \cite[Theorem 1.4.9]{lec7} for a proof). Indeed,
once this fact is known, we can pick a Hamiltonian path $\left(  \tau\left(
1\right)  ,\tau\left(  2\right)  ,\ldots,\tau\left(  n\right)  \right)  $ of
$D$, and argue (by strong induction on $j-i$) that each pair $\left(
\tau\left(  i\right)  ,\tau\left(  j\right)  \right)  $ with $i<j$ must be an
arc of $D$. But the above proof is more elementary.
\end{remark}

\begin{remark}
It can be shown that for a tournament $D\in\mathcal{T}$, the following four
statements are equivalent:

\begin{enumerate}
\item The tournament $D$ has no $3$-cycles.

\item The tournament $D$ has no cycles of length $3$.

\item The tournament $D$ has no cycles (of any length).

\item The tournament $D$ has the form $T_{\sigma}$ for some $\sigma\in S_{n}$.
\end{enumerate}

Indeed, the equivalence 1$\Longleftrightarrow$4 is furnished by Theorem
\ref{thm.Tsig.props} \textbf{(b)}, whereas the implications 4$\Longrightarrow
$3$\Longrightarrow$2$\Longrightarrow$1 are easy to check.

A tournament $D$ satisfying the four equivalent statements 1, 2, 3, 4 is said
to be \emph{nontransitive}.
\end{remark}

We can now represent the left hand side of (\ref{eq.thm.det.vander.eq}) as a
sum that looks enticingly like the right hand side in Proposition
\ref{prop.tour1n.prodxjxi}:

\begin{proposition}
\label{prop.Tsig.Vassum}Define the $n\times n$-matrix $V$ as in Theorem
\ref{thm.det.vander}. Then,%
\[
\det V=\sum_{\substack{D\in\mathcal{T};\\D\text{ has no }3\text{-cycles}%
}}w\left(  D\right)  .
\]

\end{proposition}

\begin{proof}
The $\left(  i,j\right)  $-th entry of the matrix $V$ is $x_{j}^{i-1}$ for all
$i,j\in\left\{  1,2,\ldots,n\right\}  $. Thus, the definition of a determinant
yields%
\begin{equation}
\det V=\sum_{\sigma\in S_{n}}\underbrace{\operatorname*{sign}\sigma\cdot
\prod_{i=1}^{n}x_{\sigma\left(  i\right)  }^{i-1}}_{\substack{=w\left(
T_{\sigma}\right)  \\\text{(by (\ref{eq.thm.Tsig.props.d.w}))}}}=\sum
_{\sigma\in S_{n}}w\left(  T_{\sigma}\right)  . \label{pf.prop.Tsig.Vassum.1}%
\end{equation}


However, Theorem \ref{thm.Tsig.props} \textbf{(b)} yields that the tournaments
$D\in\mathcal{T}$ that have no $3$-cycles are precisely the digraphs of the
form $T_{\sigma}$ with $\sigma\in S_{n}$. Furthermore, Theorem
\ref{thm.Tsig.props} \textbf{(c)} yields that each such tournament can be
written as $T_{\sigma}$ for a \textbf{unique} permutation $\sigma\in S_{n}$
(since distinct permutations $\sigma$ lead to distinct tournaments $T_{\sigma
}$). Thus,%
\[
\sum_{\substack{D\in\mathcal{T};\\D\text{ has no }3\text{-cycles}}}w\left(
D\right)  =\sum_{\sigma\in S_{n}}w\left(  T_{\sigma}\right)  .
\]
Comparing this with (\ref{pf.prop.Tsig.Vassum.1}), we obtain $\det
V=\sum_{\substack{D\in\mathcal{T};\\D\text{ has no }3\text{-cycles}}}w\left(
D\right)  $. This proves Proposition \ref{prop.Tsig.Vassum}.
\end{proof}

\subsection{The numbers $w_{0},w_{1},w_{2},\ldots$}

Our goal is to prove that the left hand sides in Proposition
\ref{prop.Tsig.Vassum} and in Proposition \ref{prop.tour1n.prodxjxi} are
equal. To that purpose, we shall show that the right hand sides are equal.
These right hand sides are already very similar:%
\[
\sum_{\substack{D\in\mathcal{T};\\D\text{ has no }3\text{-cycles}}}w\left(
D\right)  \ \ \ \ \ \ \ \ \ \ \text{versus}\ \ \ \ \ \ \ \ \ \ \sum
_{D\in\mathcal{T}}w\left(  D\right)  .
\]
Yet, they at least look different: The latter is a sum containing a lot of
addends that the former does not.

We shall reconcile this difference by showing that all these addends (i.e.,
all the addends corresponding to tournaments $D\in\mathcal{T}$ that have at
least one $3$-cycle) cancel each other out. This will be achieved by reversing
the arcs of a cycle; Proposition \ref{prop.tour-vand.1} will come rather handy here.

First, we introduce some notations:

\begin{convention}
For each $k\in\mathbb{N}$, we let
\[
w_{k}:=\sum_{\substack{D\in\mathcal{T};\\D\text{ has }k\text{ many
}3\text{-cycles}}}w\left(  D\right)  .
\]
(Here, \textquotedblleft$k$ many\textquotedblright\ means \textquotedblleft
exactly $k$ many\textquotedblright.)
\end{convention}

Thus,
\begin{equation}
\sum_{D\in\mathcal{T}}w\left(  D\right)  =w_{0}+w_{1}+w_{2}+\cdots
\label{eq.w.sum}%
\end{equation}
(this infinite sum is well-defined, since all sufficiently large
$k\in\mathbb{N}$ satisfy $w_{k}=0$).\ \ \ \ \footnote{Note that, because of
the way we defined $3$-cycles, the $\#$ of $3$-cycles in a tournament $D$ is
always a multiple of $3$, since each $3$-cycle $\left(  u,v,w\right)  $ leads
to two other $3$-cycles $\left(  v,w,u\right)  $ and $\left(  w,u,v\right)  $.
So we have $w_{k}=0$ for all $k\in\mathbb{N}$ that are not multiples of $3$.}
Proposition \ref{prop.Tsig.Vassum} says that%
\begin{align}
\det V  &  =\sum_{\substack{D\in\mathcal{T};\\D\text{ has no }3\text{-cycles}%
}}w\left(  D\right)  =\sum_{\substack{D\in\mathcal{T};\\D\text{ has }0\text{
many }3\text{-cycles}}}w\left(  D\right) \nonumber\\
&  =w_{0} \label{eq.prop.Tsig.Vassum.w0}%
\end{align}
(by the definition of $w_{0}$). If we can now show that all integers $k>0$
satisfy $w_{k}=0$, then we will see that the right hand sides of
(\ref{eq.prop.Tsig.Vassum.w0}) and (\ref{eq.w.sum}) are equal, and thus we
will obtain%
\[
\det V=\sum_{D\in\mathcal{T}}w\left(  D\right)  =\prod_{1\leq i<j\leq
n}\left(  x_{j}-x_{i}\right)  \ \ \ \ \ \ \ \ \ \ \left(  \text{by Proposition
\ref{prop.tour1n.prodxjxi}}\right)  ;
\]
this will prove Theorem \ref{thm.det.vander}.

\subsection{The great cancelling}

So how do we prove that all $k>0$ satisfy $w_{k}=0$ ? We begin with lemmas:

\begin{lemma}
\label{lem.cancel.sign}Let $D\in\mathcal{T}$ be a tournament. Let $\left(
u,v,w\right)  $ be a $3$-cycle of $D$. Let $D^{\prime}$ be the digraph
obtained from $D$ by reversing the arcs $uv$, $vw$ and $wu$ (this means
replacing them by $vu$, $wv$ and $uw$). Then, $D^{\prime}$ is again a
tournament in $\mathcal{T}$, and satisfies%
\[
\operatorname*{sign}\left(  D^{\prime}\right)  =-\operatorname*{sign}D.
\]

\end{lemma}

\begin{proof}
Clearly, $D^{\prime}$ is again a tournament (since reversing an arc in a
tournament yields a tournament). It remains to prove that
$\operatorname*{sign}\left(  D^{\prime}\right)  =-\operatorname*{sign}D$.

The pairs $uv$, $vw$ and $wu$ are arcs of $D$ (since $\left(  u,v,w\right)  $
is a $3$-cycle of $D$). Hence, none of the pairs $vu$, $wv$ and $uw$ is an arc
of $D$ (by the tournament axiom, since $D$ is a tournament).

We know that the digraph $D^{\prime}$ is obtained from $D$ by reversing the
arcs $uv$, $vw$ and $wu$. Let us refer to all the other arcs of $D$ as
\emph{inert}.

Thus, the arcs of $D$ are $uv$, $vw$, $wu$ and all the inert arcs of $D$.
Therefore, the arcs of $D^{\prime}$ are $vu$, $wv$, $uw$ and all the inert
arcs of $D$ (since $D^{\prime}$ is obtained from $D$ by reversing the arcs
$uv$, $vw$ and $wu$). Note that none of the arcs $vu$, $wv$ and $uw$ appears
among the inert arcs of $D$, since none of the pairs $vu$, $wv$ and $uw$ is an
arc of $D$.

Now, the definition of $\operatorname*{sign}D$ yields%
\[
\operatorname*{sign}D=\prod_{\substack{\left(  i,j\right)  \text{ is
an}\\\text{arc of }D}}\left(  -1\right)  ^{\left[  i>j\right]  }=\left(
-1\right)  ^{\left[  u>v\right]  }\cdot\left(  -1\right)  ^{\left[
v>w\right]  }\cdot\left(  -1\right)  ^{\left[  w>u\right]  }\cdot
\prod_{\substack{\left(  i,j\right)  \text{ is an}\\\text{inert arc of }%
D}}\left(  -1\right)  ^{\left[  i>j\right]  }%
\]
(since the arcs of $D$ are $uv$, $vw$, $wu$ and all the inert arcs). The
definition of $\operatorname*{sign}\left(  D^{\prime}\right)  $ yields
\[
\operatorname*{sign}\left(  D^{\prime}\right)  =\prod_{\substack{\left(
i,j\right)  \text{ is an}\\\text{arc of }D^{\prime}}}\left(  -1\right)
^{\left[  i>j\right]  }=\left(  -1\right)  ^{\left[  v>u\right]  }\cdot\left(
-1\right)  ^{\left[  w>v\right]  }\cdot\left(  -1\right)  ^{\left[
u>w\right]  }\cdot\prod_{\substack{\left(  i,j\right)  \text{ is
an}\\\text{inert arc of }D}}\left(  -1\right)  ^{\left[  i>j\right]  }%
\]
(since the arcs of $D^{\prime}$ are $vu$, $wv$, $uw$ and all the inert arcs of
$D$).

However, the vertices $u$ and $v$ are distinct (since $\left(  u,v,w\right)  $
is a $3$-cycle). Thus, exactly one of the two inequalities $u>v$ and $v>u$
holds. In other words, exactly one of the two truth values $\left[
u>v\right]  $ and $\left[  v>u\right]  $ equals $1$, while the other equals
$0$. Hence, exactly one of the two numbers $\left(  -1\right)  ^{\left[
u>v\right]  }$ and $\left(  -1\right)  ^{\left[  v>u\right]  }$ equals $-1$,
while the other equals $1$. Therefore, these two numbers differ in sign but
have the same magnitude. Consequently,%
\[
\left(  -1\right)  ^{\left[  u>v\right]  }=-\left(  -1\right)  ^{\left[
v>u\right]  }.
\]
Similarly, $\left(  -1\right)  ^{\left[  v>w\right]  }=-\left(  -1\right)
^{\left[  w>v\right]  }$ and $\left(  -1\right)  ^{\left[  w>u\right]
}=-\left(  -1\right)  ^{\left[  u>w\right]  }$. Now, our above formula for
$\operatorname*{sign}D$ becomes%
\begin{align*}
\operatorname*{sign}D &  =\underbrace{\left(  -1\right)  ^{\left[  u>v\right]
}}_{=-\left(  -1\right)  ^{\left[  v>u\right]  }}\cdot\underbrace{\left(
-1\right)  ^{\left[  v>w\right]  }}_{=-\left(  -1\right)  ^{\left[
w>v\right]  }}\cdot\underbrace{\left(  -1\right)  ^{\left[  w>u\right]  }%
}_{=-\left(  -1\right)  ^{\left[  u>w\right]  }}\cdot\prod_{\substack{\left(
i,j\right)  \text{ is an}\\\text{inert arc of }D}}\left(  -1\right)  ^{\left[
i>j\right]  }\\
&  =-\underbrace{\left(  -1\right)  ^{\left[  v>u\right]  }\cdot\left(
-1\right)  ^{\left[  w>v\right]  }\cdot\left(  -1\right)  ^{\left[
u>w\right]  }\cdot\prod_{\substack{\left(  i,j\right)  \text{ is
an}\\\text{inert arc of }D}}\left(  -1\right)  ^{\left[  i>j\right]  }%
}_{\substack{=\operatorname*{sign}\left(  D^{\prime}\right)  \\\text{(by our
above formula for }\operatorname*{sign}\left(  D^{\prime}\right)  \text{)}%
}}=-\operatorname*{sign}\left(  D^{\prime}\right)  .
\end{align*}
In other words, $\operatorname*{sign}\left(  D^{\prime}\right)
=-\operatorname*{sign}D$. This completes the proof of Lemma
\ref{lem.cancel.sign}.
\end{proof}

\begin{lemma}
\label{lem.cancel.1}Let $k$ be a positive integer. Let $\left(  d_{1}%
,d_{2},\ldots,d_{n}\right)  \in\mathbb{N}^{n}$ be any $n$-tuple of nonnegative
integers. Then,%
\[
\sum_{\substack{D\in\mathcal{T};\\D\text{ has }k\text{ many }3\text{-cycles}%
;\\\deg^{-}i=d_{i}\text{ for each }i}}\operatorname*{sign}D=0.
\]
(Here, $\deg^{-}i$ means the indegree of the vertex $i$ in the digraph $D$.
Also, \textquotedblleft for each $i$\textquotedblright\ means
\textquotedblleft for each $i\in\left\{  1,2,\ldots,n\right\}  $%
\textquotedblright.)
\end{lemma}

\begin{proof}
A \emph{flippy pair} shall mean a pair $\left(  D,\alpha\right)  $, where

\begin{itemize}
\item $D\in\mathcal{T}$ is a tournament having $k$ many $3$-cycles and
satisfying $\deg^{-}i=d_{i}$ for each $i$;

\item $\alpha$ is a $3$-cycle of $D$.
\end{itemize}

If $\left(  D,\alpha\right)  $ is a flippy pair, then we define a new flippy
pair $\operatorname*{flip}\left(  D,\alpha\right)  $ as follows:

\begin{itemize}
\item Let $\left(  u,v,w\right)  $ be the $3$-cycle $\alpha$.

\item We obtain a new digraph $D^{\prime}$ from $D$ by reversing the arcs
$uv$, $vw$ and $wu$ (this means replacing them by $vu$, $wv$ and $uw$). Note
that this digraph $D^{\prime}$ is again a tournament in $\mathcal{T}$, and
again has $k$ many $3$-cycles (because Proposition \ref{prop.tour-vand.1}
shows that $\left(  \#\text{ of }3\text{-cycles of }D^{\prime}\right)
=\left(  \#\text{ of }3\text{-cycles of }D\right)  =k$). This tournament
$D^{\prime}$ furthermore satisfies the equalities $\deg^{-}i=d_{i}$ for each
$i$ (since $D$ satisfies these equalities, and since the indegrees of the
vertices have not changed from $D$ to $D^{\prime}$\ \ \ \ \footnote{because
each of the three vertices $u$, $v$ and $w$ lost one incoming arc and gained
another when we reversed the arcs $uv$, $vw$ and $wu$}).

\item We let $\alpha^{\prime}$ be the $3$-cycle $\left(  u,w,v\right)  $ of
$D^{\prime}$. (This is indeed a $3$-cycle of $D^{\prime}$, due to the
construction of $D^{\prime}$.)

\item We let $\operatorname*{flip}\left(  D,\alpha\right)  $ be the flippy
pair $\left(  D^{\prime},\alpha^{\prime}\right)  $. (This is indeed a flippy
pair, because we have seen that $D^{\prime}$ is a tournament in $\mathcal{T}$
having $k$ many $3$-cycles and satisfying $\deg^{-}i=d_{i}$ for each $i$, and
that $\alpha^{\prime}$ is a $3$-cycle of $D^{\prime}$.)
\end{itemize}

Thus, we have defined a map $\operatorname*{flip}$ that sends flippy pairs to
flippy pairs. It is easy to see that this map is its own inverse: That is, if
$\left(  D,\alpha\right)  $ is a flippy pair, and if $\left(  D^{\prime
},\alpha^{\prime}\right)  =\operatorname*{flip}\left(  D,\alpha\right)  $,
then $\left(  D,\alpha\right)  =\operatorname*{flip}\left(  D^{\prime}%
,\alpha^{\prime}\right)  $ (because $D^{\prime}$ is obtained from $D$ by
reversing the arcs $uv$, $vw$ and $wu$, and thus $D$ can be recovered from
$D^{\prime}$ by reversing the arcs $uw$, $wv$ and $vu$).

Furthermore, the map $\operatorname*{flip}$ changes the sign of a tournament:
That is, if $\left(  D,\alpha\right)  $ is a flippy pair, and if $\left(
D^{\prime},\alpha^{\prime}\right)  =\operatorname*{flip}\left(  D,\alpha
\right)  $, then
\begin{equation}
\operatorname*{sign}\left(  D^{\prime}\right)  =-\operatorname*{sign}D.
\label{pf.lem.cancel.1.sign}%
\end{equation}


[\textit{Proof of (\ref{pf.lem.cancel.1.sign}):} Let $\left(  D,\alpha\right)
$ be a flippy pair. Let $\left(  D^{\prime},\alpha^{\prime}\right)
=\operatorname*{flip}\left(  D,\alpha\right)  $. By the definition of the map
$\operatorname*{flip}$, we know that the digraph $D^{\prime}$ is obtained from
$D$ by reversing the arcs $uv$, $vw$ and $wu$, where $\left(  u,v,w\right)  $
is the $3$-cycle $\alpha$. Thus, Lemma \ref{lem.cancel.sign} yields
$\operatorname*{sign}\left(  D^{\prime}\right)  =-\operatorname*{sign}D$. This
proves (\ref{pf.lem.cancel.1.sign}).] \medskip

Thus, if $\left(  D,\alpha\right)  $ is a flippy pair satisfying
$\operatorname*{sign}D=1$, and if $\left(  D^{\prime},\alpha^{\prime}\right)
=\operatorname*{flip}\left(  D,\alpha\right)  $, then $\left(  D^{\prime
},\alpha^{\prime}\right)  $ is a flippy pair satisfying $\operatorname*{sign}%
\left(  D^{\prime}\right)  =-1$ (because (\ref{pf.lem.cancel.1.sign}) yields
$\operatorname*{sign}\left(  D^{\prime}\right)
=-\underbrace{\operatorname*{sign}D}_{=1}=-1$). Hence, we obtain a map
\begin{align*}
&  \text{from the set }\left\{  \text{flippy pairs }\left(  D,\alpha\right)
\text{ satisfying }\operatorname*{sign}D=1\right\}  \\
&  \text{to the set }\left\{  \text{flippy pairs }\left(  D,\alpha\right)
\text{ satisfying }\operatorname*{sign}D=-1\right\}  ,
\end{align*}
which sends each flippy pair $\left(  D,\alpha\right)  $ to
$\operatorname*{flip}\left(  D,\alpha\right)  $. Similarly, we obtain a map
\begin{align*}
&  \text{from the set }\left\{  \text{flippy pairs }\left(  D,\alpha\right)
\text{ satisfying }\operatorname*{sign}D=-1\right\}  \\
&  \text{to the set }\left\{  \text{flippy pairs }\left(  D,\alpha\right)
\text{ satisfying }\operatorname*{sign}D=1\right\}  ,
\end{align*}
which sends each flippy pair $\left(  D,\alpha\right)  $ to
$\operatorname*{flip}\left(  D,\alpha\right)  $. These two maps are mutually
inverse (since the map $\operatorname*{flip}$ is its own inverse), and thus
are bijections. Hence, the bijection principle yields that%
\begin{align*}
&  \left\vert \left\{  \text{flippy pairs }\left(  D,\alpha\right)  \text{
satisfying }\operatorname*{sign}D=1\right\}  \right\vert \\
&  =\left\vert \left\{  \text{flippy pairs }\left(  D,\alpha\right)  \text{
satisfying }\operatorname*{sign}D=-1\right\}  \right\vert .
\end{align*}
In other words, there are as many flippy pairs $\left(  D,\alpha\right)  $
satisfying $\operatorname*{sign}D=1$ as there are flippy pairs $\left(
D,\alpha\right)  $ satisfying $\operatorname*{sign}D=-1$. Hence, in the sum%
\[
\sum_{\left(  D,\alpha\right)  \text{ is a flippy pair}}\operatorname*{sign}D,
\]
the addends equal to $1$ and the addends equal to $-1$ are equinumerous, and
consequently these addends cancel each other out. The sum therefore equals $0$
(since each addend of this sum is either a $1$ or a $-1$). In other words,%
\begin{equation}
\sum_{\left(  D,\alpha\right)  \text{ is a flippy pair}}\operatorname*{sign}%
D=0.\label{pf.lem.cancel.1.sum-flips}%
\end{equation}


However, recall that the $D$ in a flippy pair $\left(  D,\alpha\right)  $ has
to be a tournament in $\mathcal{T}$ having $k$ many $3$-cycles and satisfying
$\deg^{-}i=d_{i}$ for each $i$, whereas the $\alpha$ has to be a $3$-cycle of
$D$. Thus, the summation sign $\sum_{\left(  D,\alpha\right)  \text{ is a
flippy pair}}$ can be rewritten as follows:%
\[
\sum_{\left(  D,\alpha\right)  \text{ is a flippy pair}}=\sum_{\substack{D\in
\mathcal{T};\\D\text{ has }k\text{ many }3\text{-cycles};\\\deg^{-}%
i=d_{i}\text{ for each }i}}\ \ \sum_{\alpha\text{ is a }3\text{-cycle of }D}.
\]
Thus,%
\begin{align*}
\sum_{\left(  D,\alpha\right)  \text{ is a flippy pair}}\operatorname*{sign}D
&  =\sum_{\substack{D\in\mathcal{T};\\D\text{ has }k\text{ many }%
3\text{-cycles};\\\deg^{-}i=d_{i}\text{ for each }i}}\ \ \underbrace{\sum
_{\alpha\text{ is a }3\text{-cycle of }D}\operatorname*{sign}D}_{=\left(
\#\text{ of }3\text{-cycles of }D\right)  \cdot\operatorname*{sign}D}\\
&  =\sum_{\substack{D\in\mathcal{T};\\D\text{ has }k\text{ many }%
3\text{-cycles};\\\deg^{-}i=d_{i}\text{ for each }i}}\underbrace{\left(
\#\text{ of }3\text{-cycles of }D\right)  }_{\substack{=k\\\text{(since
}D\text{ has }k\text{ many }3\text{-cycles)}}}\cdot\operatorname*{sign}D\\
&  =\sum_{\substack{D\in\mathcal{T};\\D\text{ has }k\text{ many }%
3\text{-cycles};\\\deg^{-}i=d_{i}\text{ for each }i}}k\cdot
\operatorname*{sign}D=k\sum_{\substack{D\in\mathcal{T};\\D\text{ has }k\text{
many }3\text{-cycles};\\\deg^{-}i=d_{i}\text{ for each }i}%
}\operatorname*{sign}D.
\end{align*}
Therefore, (\ref{pf.lem.cancel.1.sum-flips}) can be rewritten as%
\[
k\sum_{\substack{D\in\mathcal{T};\\D\text{ has }k\text{ many }3\text{-cycles}%
;\\\deg^{-}i=d_{i}\text{ for each }i}}\operatorname*{sign}D=0.
\]
We can divide this equality by $k$ (since $k$ is positive), and obtain
\[
\sum_{\substack{D\in\mathcal{T};\\D\text{ has }k\text{ many }3\text{-cycles}%
;\\\deg^{-}i=d_{i}\text{ for each }i}}\operatorname*{sign}D=0.
\]
This proves Lemma \ref{lem.cancel.1}.
\end{proof}

\begin{lemma}
\label{lem.cancel.2}Let $k$ be a positive integer. Let $\left(  d_{1}%
,d_{2},\ldots,d_{n}\right)  \in\mathbb{N}^{n}$ be any $n$-tuple of nonnegative
integers. Then,%
\[
\sum_{\substack{D\in\mathcal{T};\\D\text{ has }k\text{ many }3\text{-cycles}%
;\\\deg^{-}i=d_{i}\text{ for each }i}}w\left(  D\right)  =0.
\]
(Here, $\deg^{-}i$ means the indegree of the vertex $i$ in the digraph $D$.
Also, \textquotedblleft for each $i$\textquotedblright\ means
\textquotedblleft for each $i\in\left\{  1,2,\ldots,n\right\}  $%
\textquotedblright.)
\end{lemma}

\begin{proof}
We have%
\begin{align*}
& \sum_{\substack{D\in\mathcal{T};\\D\text{ has }k\text{ many }3\text{-cycles}%
;\\\deg^{-}i=d_{i}\text{ for each }i}}\underbrace{w\left(  D\right)
}_{\substack{=\left(  \operatorname*{sign}D\right)  \cdot\prod_{j=1}^{n}%
x_{j}^{\deg^{-}j}\\\text{(by Proposition \ref{prop.tour1n.wD1})}}}\\
& =\sum_{\substack{D\in\mathcal{T};\\D\text{ has }k\text{ many }%
3\text{-cycles};\\\deg^{-}i=d_{i}\text{ for each }i}}\left(
\operatorname*{sign}D\right)  \cdot\prod_{j=1}^{n}\underbrace{x_{j}^{\deg
^{-}j}}_{\substack{=x_{j}^{d_{j}}\\\text{(since }\deg^{-}j=d_{j}%
\\\text{(because }\deg^{-}i=d_{i}\text{ for each }i\text{))}}}\\
& =\sum_{\substack{D\in\mathcal{T};\\D\text{ has }k\text{ many }%
3\text{-cycles};\\\deg^{-}i=d_{i}\text{ for each }i}}\left(
\operatorname*{sign}D\right)  \cdot\prod_{j=1}^{n}x_{j}^{d_{j}}=\left(
\prod_{j=1}^{n}x_{j}^{d_{j}}\right)  \underbrace{\sum_{\substack{D\in
\mathcal{T};\\D\text{ has }k\text{ many }3\text{-cycles};\\\deg^{-}%
i=d_{i}\text{ for each }i}}\operatorname*{sign}D}_{\substack{=0\\\text{(by
Lemma \ref{lem.cancel.1})}}}\\
& =0.
\end{align*}
This proves Lemma \ref{lem.cancel.2}.
\end{proof}

\begin{lemma}
\label{lem.cancel.wk=0}Let $k$ be a positive integer. Then, $w_{k}=0$.
\end{lemma}

\begin{proof}
The definition of $w_{k}$ yields%
\begin{align*}
w_{k} &  =\sum_{\substack{D\in\mathcal{T};\\D\text{ has }k\text{ many
}3\text{-cycles}}}w\left(  D\right)  \\
&  =\sum_{\left(  d_{1},d_{2},\ldots,d_{n}\right)  \in\mathbb{N}^{n}%
}\ \ \underbrace{\sum_{\substack{D\in\mathcal{T};\\D\text{ has }k\text{ many
}3\text{-cycles};\\\deg^{-}i=d_{i}\text{ for each }i}}w\left(  D\right)
}_{\substack{=0\\\text{(by Lemma \ref{lem.cancel.2})}}}\\
&  \ \ \ \ \ \ \ \ \ \ \ \ \ \ \ \ \ \ \ \ \left(
\begin{array}
[c]{c}%
\text{here, we have split up the sum according}\\
\text{to the }n\text{-tuple }\left(  \deg^{-}1,\deg^{-}2,\ldots,\deg
^{-}n\right)
\end{array}
\right)  \\
&  =0.
\end{align*}
This proves Lemma \ref{lem.cancel.wk=0}.
\end{proof}

\subsection{The finish line}

Proving Theorem \ref{thm.det.vander} is now a matter of combining what we know:

\begin{proof}
[Proof of Theorem \ref{thm.det.vander}.]Proposition \ref{prop.tour1n.prodxjxi}
yields%
\begin{align*}
\prod_{1\leq i<j\leq n}\left(  x_{j}-x_{i}\right)   &  =\sum_{D\in\mathcal{T}%
}w\left(  D\right)  =w_{0}+w_{1}+w_{2}+\cdots\ \ \ \ \ \ \ \ \ \ \left(
\text{by (\ref{eq.w.sum})}\right)  \\
&  =\sum_{k\in\mathbb{N}}w_{k}=w_{0}+\sum_{k>0}\underbrace{w_{k}%
}_{\substack{=0\\\text{(by Lemma \ref{lem.cancel.wk=0})}}}=w_{0}%
+\underbrace{\sum_{k>0}0}_{=0}=w_{0}=\det V
\end{align*}
(by (\ref{eq.prop.Tsig.Vassum.w0})). Thus follows Theorem \ref{thm.det.vander}.
\end{proof}

Here ends our scenic route to the Vandermonde determinant. A different
combinatorial proof -- also using tournaments -- is sketched in
\cite[Exercises 2.4.1--2.4.6]{Bresso99}. Yet another (not using tournaments)
appears in \cite{BenDre07}. Moreover, several variants of the Vandermonde
determinant (type-B, type-C and type-D versions, for those who know the lingo
of Coxeter groups) have been proved using tournaments by Bressoud
\cite{Bresso87}.

\begin{thebibliography}{99999999}                                                                                         %


\bibitem[BenDre07]{BenDre07}%
\href{https://scholarship.claremont.edu/hmc_fac_pub/524/}{Arthur T. Benjamin,
Gregory P. Dresden, \textit{A Combinatorial Proof of Vandermonde's
Determinant}, American Mathematical Monthly \textbf{114} (April 2007), pp.
338--341.}

\bibitem[Bresso87]{Bresso87}%
\href{https://doi.org/10.1016/S0195-6698(87)80028-8}{D. M. Bressoud,
\textit{Colored Tournaments and Weyl's Denominator Formula}, Europ. J.
Combinatorics \textbf{8} (1987), pp. 245--255}.

\bibitem[Bresso99]{Bresso99}%
\href{https://doi.org/10.1017/CBO9780511613449}{David M. Bressoud,
\textit{Proofs and Confirmations: The Story of the Alternating Sign Matrix
Conjecture}, Cambridge University Press 1999.} See
\url{https://www.macalester.edu/~bressoud/books/PnC/PnCcorrect.html} for errata.

\bibitem[Conrad]{Conrad}Keith Conrad, \textit{The sign of a permutation}, 17
March 2022.\newline\url{https://kconrad.math.uconn.edu/blurbs/grouptheory/sign.pdf}

\bibitem[Day16]{Day16}Martin V. Day, \textit{An Introduction to Proofs and the
Mathematical Vernacular}, 7 December 2016. \newline\url{https://web.archive.org/web/20180712152432/https://www.math.vt.edu/people/day/ProofsBook/IPaMV.pdf}

\bibitem[Gessel79]{Gessel79}\href{https://doi.org/10.1002/jgt.3190030315}{Ira
Gessel, \textit{Tournaments and Vandermonde's Determinant}, Journal of Graph
Theory \textbf{3} (1979), pp. 305--307}.

\bibitem[Grinbe15]{detnotes}Darij Grinberg, \textit{Notes on the combinatorial
fundamentals of algebra}, 25 May 2021.\newline%
\url{http://www.cip.ifi.lmu.de/~grinberg/primes2015/sols.pdf} \newline The
numbering of theorems and formulas in this link might shift when the project
gets updated; for a \textquotedblleft frozen\textquotedblright\ version whose
numbering is guaranteed to match that in the citations above, see
\url{https://github.com/darijgr/detnotes/releases/tag/2019-01-10} or
\href{https://arxiv.org/abs/2008.09862v2}{arXiv:2008.09862v2}.

\bibitem[Grinbe21]{lecs}Darij Grinberg, \textit{An Introduction to Algebraic
Combinatorics [Math 701, Spring 2021 lecture notes]}, January 3, 2022.\newline\url{https://www.cip.ifi.lmu.de/~grinberg/t/21s/lecs.pdf}

\bibitem[hw2s]{hw2s}Darij Grinberg, \textit{Math 5707 Spring 2017: homework
set 2 with solutions},
\url{https://www.cip.ifi.lmu.de/~grinberg/t/17s/hw2s.pdf} .

\bibitem[lec7]{lec7}Darij Grinberg, \textit{UMN, Spring 2017, Math 5707:
Lecture 7 (Hamiltonian paths in digraphs)}, 5 April 2022.\newline\url{https://www.cip.ifi.lmu.de/~grinberg/t/17s/5707lec7.pdf}

\bibitem[Strick20]{Strick20}Neil Strickland, \textit{MAS334 Combinatorics},
lecture notes and solutions, 6 December 2020.\newline\url{http://neil-strickland.staff.shef.ac.uk/courses/MAS334/}
\end{thebibliography}


\end{document}