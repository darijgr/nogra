\documentclass[numbers=enddot,12pt,final,onecolumn,notitlepage]{scrartcl}%
\usepackage[headsepline,footsepline,manualmark]{scrlayer-scrpage}
\usepackage[all,cmtip]{xy}
\usepackage{amssymb}
\usepackage{amsmath}
\usepackage{amsthm}
\usepackage{framed}
\usepackage{comment}
\usepackage{color}
\usepackage{hyperref}
\usepackage[sc]{mathpazo}
\usepackage[T1]{fontenc}
\usepackage{tikz}
\usepackage{needspace}
\usepackage{tabls}
%TCIDATA{OutputFilter=latex2.dll}
%TCIDATA{Version=5.50.0.2960}
%TCIDATA{LastRevised=Tuesday, November 27, 2018 19:05:24}
%TCIDATA{SuppressPackageManagement}
%TCIDATA{<META NAME="GraphicsSave" CONTENT="32">}
%TCIDATA{<META NAME="SaveForMode" CONTENT="1">}
%TCIDATA{BibliographyScheme=Manual}
%TCIDATA{Language=American English}
%BeginMSIPreambleData
\providecommand{\U}[1]{\protect\rule{.1in}{.1in}}
%EndMSIPreambleData
\usetikzlibrary{arrows}
\newcounter{exer}
\newcounter{exera}
\numberwithin{exer}{section}
\theoremstyle{definition}
\newtheorem{theo}{Theorem}[section]
\newenvironment{theorem}[1][]
{\begin{theo}[#1]\begin{leftbar}}
{\end{leftbar}\end{theo}}
\newtheorem{lem}[theo]{Lemma}
\newenvironment{lemma}[1][]
{\begin{lem}[#1]\begin{leftbar}}
{\end{leftbar}\end{lem}}
\newtheorem{prop}[theo]{Proposition}
\newenvironment{proposition}[1][]
{\begin{prop}[#1]\begin{leftbar}}
{\end{leftbar}\end{prop}}
\newtheorem{defi}[theo]{Definition}
\newenvironment{definition}[1][]
{\begin{defi}[#1]\begin{leftbar}}
{\end{leftbar}\end{defi}}
\newtheorem{remk}[theo]{Remark}
\newenvironment{remark}[1][]
{\begin{remk}[#1]\begin{leftbar}}
{\end{leftbar}\end{remk}}
\newtheorem{coro}[theo]{Corollary}
\newenvironment{corollary}[1][]
{\begin{coro}[#1]\begin{leftbar}}
{\end{leftbar}\end{coro}}
\newtheorem{conv}[theo]{Convention}
\newenvironment{convention}[1][]
{\begin{conv}[#1]\begin{leftbar}}
{\end{leftbar}\end{conv}}
\newtheorem{quest}[theo]{Question}
\newenvironment{algorithm}[1][]
{\begin{quest}[#1]\begin{leftbar}}
{\end{leftbar}\end{quest}}
\newtheorem{warn}[theo]{Warning}
\newenvironment{conclusion}[1][]
{\begin{warn}[#1]\begin{leftbar}}
{\end{leftbar}\end{warn}}
\newtheorem{conj}[theo]{Conjecture}
\newenvironment{conjecture}[1][]
{\begin{conj}[#1]\begin{leftbar}}
{\end{leftbar}\end{conj}}
\newtheorem{exam}[theo]{Example}
\newenvironment{example}[1][]
{\begin{exam}[#1]\begin{leftbar}}
{\end{leftbar}\end{exam}}
\newtheorem{exmp}[exer]{Exercise}
\newenvironment{exercise}[1][]
{\begin{exmp}[#1]\begin{leftbar}}
{\end{leftbar}\end{exmp}}
\newenvironment{statement}{\begin{quote}}{\end{quote}}
\iffalse
\newenvironment{proof}[1][Proof]{\noindent\textbf{#1.} }{\ \rule{0.5em}{0.5em}}
\fi
\let\sumnonlimits\sum
\let\prodnonlimits\prod
\let\cupnonlimits\bigcup
\let\capnonlimits\bigcap
\renewcommand{\sum}{\sumnonlimits\limits}
\renewcommand{\prod}{\prodnonlimits\limits}
\renewcommand{\bigcup}{\cupnonlimits\limits}
\renewcommand{\bigcap}{\capnonlimits\limits}
\setlength\tablinesep{3pt}
\setlength\arraylinesep{3pt}
\setlength\extrarulesep{3pt}
\voffset=0cm
\hoffset=-0.7cm
\setlength\textheight{22.5cm}
\setlength\textwidth{15.5cm}
\newcommand\arxiv[1]{\href{http://www.arxiv.org/abs/#1}{\texttt{arXiv:#1}}}
\newenvironment{verlong}{}{}
\newenvironment{vershort}{}{}
\newenvironment{noncompile}{}{}
\excludecomment{verlong}
\includecomment{vershort}
\excludecomment{noncompile}
\newcommand{\id}{\operatorname{id}}
\ihead{An exercise on source and sink mutations of acyclic quivers}
\ohead{page \thepage}
\cfoot{}
\begin{document}

\title{An exercise on source and sink mutations of acyclic quivers\thanks{This used
to be Chapter 7 of my notes \textquotedblleft Notes on the combinatorial
fundamentals of algebra\textquotedblright\ (version of 7 November 2018), but
has since been removed from the latter notes.}}
\author{Darij Grinberg}
\date{
%TCIMACRO{\TeXButton{today}{\today} }%
%BeginExpansion
\today
%EndExpansion
}
\maketitle

In this note, we will use the following notations (which come from Lampe's
notes \cite[\S 2.1.1]{Lampe}):

\begin{itemize}
\item A \textit{quiver} means a tuple $Q=\left(  Q_{0},Q_{1},s,t\right)  $,
where $Q_{0}$ and $Q_{1}$ are two finite sets and where $s$ and $t$ are two
maps from $Q_{1}$ to $Q_{0}$. We call the elements of $Q_{0}$ the
\textit{vertices} of the quiver $Q$, and we call the elements of $Q_{1}$ the
\textit{arrows} of the quiver $Q$. For every $e\in Q_{1}$, we call $s\left(
e\right)  $ the \textit{starting point} of $e$ (and we say that $e$
\textit{starts at }$s\left(  e\right)  $), and we call $t\left(  e\right)  $
the \textit{terminal point} of $e$ (and we say that $e$ \textit{ends at}
$t\left(  e\right)  $). Furthermore, if $e\in Q_{1}$, then we say that $e$ is
an \textit{arrow from }$s\left(  e\right)  $ \textit{to }$t\left(  e\right)  $.

So the notion of a quiver is one of many different versions of the notion of a
finite directed graph. (Notice that it is a version which allows multiple
arrows, and which distinguishes between them -- i.e., the quiver stores not
just the information of how many arrows there are from a vertex to another,
but it actually has them all as distinguishable objects in $Q_{1}$. Lampe
himself seems to later tacitly switch to a different notion of quivers, where
edges from a given to vertex to another are indistinguishable and only exist
as a number. This does not matter for the next exercise, which works just as
well with either notion of a quiver; but I just wanted to have it mentioned.)

\item The \textit{underlying undirected graph} of a quiver $Q=\left(
Q_{0},Q_{1},s,t\right)  $ is defined as the undirected multigraph with vertex
set $Q_{0}$ and edge multiset%
\[
\left\{  \left\{  s\left(  e\right)  ,t\left(  e\right)  \right\}
\ \mid\ e\in Q_{1}\right\}  _{\operatorname*{multiset}}.
\]
(\textquotedblleft Multigraph\textquotedblright\ means that multiple edges are
allowed, but we do not make them distinguishable.)

\item A quiver $Q=\left(  Q_{0},Q_{1},s,t\right)  $ is said to be
\textit{acyclic} if there is no sequence \newline$\left(  a_{0},a_{1}%
,\ldots,a_{n}\right)  $ of elements of $Q_{0}$ such that $n>0$ and
$a_{0}=a_{n}$ and such that $Q$ has an arrow from $a_{i}$ to $a_{i+1}$ for
every $i\in\left\{  0,1,\ldots,n-1\right\}  $. (This is equivalent to
\cite[Definition 2.1.7]{Lampe}.) Notice that this does not mean that the
\textit{underlying undirected graph} of $Q$ has no cycles.

\item Let $Q=\left(  Q_{0},Q_{1},s,t\right)  $. Then, a \textit{sink} of $Q$
means a vertex $v\in Q_{0}$ such that no $e\in Q_{1}$ starts at $v$ (in other
words, no arrow of $Q$ starts at $v$). A \textit{source} of $Q$ means a vertex
$v\in Q_{0}$ such that no $e\in Q_{1}$ ends at $v$ (in other words, no arrow
of $Q$ ends at $v$).

\item Let $Q=\left(  Q_{0},Q_{1},s,t\right)  $. If $i\in Q_{0}$ is a sink of
$Q$, then the \textit{mutation} $\mu_{i}\left(  Q\right)  $ of $Q$ at $i$ is
the quiver obtained from $Q$ simply by turning\footnote{To \textit{turn} an
arrow $e$ means to reverse its direction, i.e., to switch the values of
$s\left(  e\right)  $ and $t\left(  e\right)  $. We model this as a change to
the functions $s$ and $t$, not as a change to the arrow itself.} all arrows
ending at $i$. (To be really pedantic: We define $\mu_{i}\left(  Q\right)  $
as the quiver $\left(  Q_{0},Q_{1},s^{\prime},t^{\prime}\right)  $, where%
\begin{align*}
s^{\prime}\left(  e\right)   &  =
\begin{cases}
t\left(  e\right)  , & \text{if }t\left(  e\right)  =i;\\
s\left(  e\right)  , & \text{if }t\left(  e\right)  \neq i
\end{cases}
\ \ \ \ \ \ \ \ \ \ \text{for each }e\in Q_{1}\\
\text{and}\ \ \ \ \ \ \ \ \ \ t^{\prime}\left(  e\right)   &  =
\begin{cases}
s\left(  e\right)  , & \text{if }t\left(  e\right)  =i;\\
t\left(  e\right)  , & \text{if }t\left(  e\right)  \neq i
\end{cases}
\ \ \ \ \ \ \ \ \ \ \text{for each }e\in Q_{1}.
\end{align*}
) If $i\in Q_{0}$ is a source of $Q$, then the \textit{mutation} $\mu
_{i}\left(  Q\right)  $ of $Q$ at $i$ is the quiver obtained from $Q$ by
turning all arrows starting at $i$. (Notice that if $i$ is both a source and a
sink of $Q$, then these two definitions give the same result; namely, $\mu
_{i}\left(  Q\right)  =Q$ in this case.)

If $Q$ is an acyclic quiver, then $\mu_{i}\left(  Q\right)  $ is acyclic as
well (whenever $i\in Q_{0}$ is a sink or a source of $Q$).

We use the word \textquotedblleft mutation\textquotedblright\ not only for the
quiver $\mu_{i}\left(  Q\right)  $, but also for the operation that transforms
$Q$ into $\mu_{i}\left(  Q\right)  $. (We have defined this operation only if
$i$ is a sink or a source of $Q$. It can be viewed as a particular case of the
more general definition of mutation given in \cite[Definition 2.2.1]{Lampe},
at least if one gives up the ability to distinguish different arrows from one
vertex to another.)
\end{itemize}

\Needspace{15\baselineskip}

\begin{exercise}
\label{exe.ps1.1.1}Let $Q=\left(  Q_{0},Q_{1},s,t\right)  $ be an acyclic quiver.

\textbf{(a)} Let $A$ and $B$ be two subsets of $Q_{0}$ such that $A\cap
B=\varnothing$ and $A\cup B=Q_{0}$. Assume that there exists no arrow of $Q$
that starts at a vertex in $B$ and ends at a vertex in $A$. Then, by turning
all arrows of $Q$ which start at a vertex in $A$ and end at a vertex in $B$,
we obtain a new acyclic quiver $\operatorname*{mut}\nolimits_{A,B}Q$.

(When we say \textquotedblleft turning all arrows of $Q$ which start at a
vertex in $A$ and end at a vertex in $B$\textquotedblright, we mean
\textquotedblleft turning all arrows $e$ of $Q$ which satisfy $s\left(
e\right)  \in A$ and $t\left(  e\right)  \in B$\textquotedblright. We do
\textbf{not} mean that we fix a vertex $a$ in $A$ and a vertex $b$ in $B$, and
only turn the arrows from $a$ to $b$.)

For example, if $Q=%
%TCIMACRO{\TeXButton{x}{\xymatrix{
%3 \ar[r] & 4 \\
%1 \ar[u] \ar[ru] \ar[r] & 2 \ar[u]
%}}}%
%BeginExpansion
\xymatrix{
3 \ar[r] & 4 \\
1 \ar[u] \ar[ru] \ar[r] & 2 \ar[u]
}%
%EndExpansion
$ and $A=\left\{  1,3\right\}  $ and $B=\left\{  2,4\right\}  $, then
\newline$\operatorname*{mut}\nolimits_{A,B}Q=%
%TCIMACRO{\TeXButton{x}{\xymatrix{
%3 & 4 \ar[l] \ar[ld] \\
%1 \ar[u] & 2 \ar[u] \ar[l]
%}}}%
%BeginExpansion
\xymatrix{
3 & 4 \ar[l] \ar[ld] \\
1 \ar[u] & 2 \ar[u] \ar[l]
}%
%EndExpansion
$.

Prove that $\operatorname*{mut}\nolimits_{A,B}Q$ can be obtained from $Q$ by a
sequence of mutations at sinks. (More precisely, there exists a sequence
$\left(  Q^{\left(  0\right)  },Q^{\left(  1\right)  },\ldots,Q^{\left(
\ell\right)  }\right)  $ of acyclic quivers such that $Q^{\left(  0\right)
}=Q$, $Q^{\left(  \ell\right)  }=\operatorname*{mut}\nolimits_{A,B}Q$, and for
every $i\in\left\{  1,2,\ldots,\ell\right\}  $, the quiver $Q^{\left(
i\right)  }$ is obtained from $Q^{\left(  i-1\right)  }$ by mutation at a sink
of $Q^{\left(  i-1\right)  }$.)

[In our above example, we can mutate at $4$ first and then at $2$.]

\textbf{(b)} If $i\in Q_{0}$ is a \textbf{source} of $Q$, then show that the
mutation $\mu_{i}\left(  Q\right)  $ can be obtained from $Q$ by a sequence of
mutations at sinks.

\textbf{(c)} Assume now that the underlying \textbf{undirected} graph of $Q$
is a tree. (In particular, $Q$ cannot have more than one edge between two
vertices, as these would form a cycle in the underlying undirected graph!)
Show that any acyclic quiver which can be obtained from $Q$ by turning some of
its arrows can also be obtained from $Q$ by a sequence of mutations at sinks.
\end{exercise}

\begin{remark}
More general results than those of Exercise \ref{exe.ps1.1.1} are stated (for
directed graphs rather than quivers, but it is easy to translate from one
language into another) in \cite{Pretzel}. An equivalent version of Exercise
\ref{exe.ps1.1.1} \textbf{(c)} also appears as Exercise 6 in \cite{mt3}
(because a quiver $Q$ whose underlying undirected graph is a tree can be
regarded as an orientation of the latter tree, and because the concept of
\textquotedblleft pushing sources\textquotedblright\ in \cite{mt3} corresponds
precisely to our concept of mutations at sinks, except that all arrows need to
be reversed).
\end{remark}

\begin{proof}
[Solution to Exercise \ref{exe.ps1.1.1}.]\textbf{(a)} We prove the claim by
induction over $\left\vert B\right\vert $.

\textit{Induction base:} Assume that $\left\vert B\right\vert =0$. Thus,
$B=\varnothing$, and thus there are no arrows of $Q$ which start at a vertex
in $A$ and at a vertex in $B$. Hence, $\operatorname*{mut}\nolimits_{A,B}Q=Q$,
and this can clearly be obtained from $Q$ by a sequence of mutations at sinks
(namely, by the empty sequence). Thus, Exercise \ref{exe.ps1.1.1} \textbf{(a)}
holds if $\left\vert B\right\vert =0$. This completes the induction
base.\footnote{Yes, this was a completely honest induction base. You don't
need to start at $\left\vert B\right\vert =1$ unless you want to use something
like $\left\vert B\right\vert >1$ in the induction step (but even then, you
should also handle the case $\left\vert B\right\vert =0$ separately).}

\textit{Induction step:}\footnote{The letter $\mathbb{N}$ denotes the set
$\left\{  0,1,2,\ldots\right\}  $ here.} Let $N\in\mathbb{N}$. Assume that
Exercise \ref{exe.ps1.1.1} \textbf{(a)} holds whenever $\left\vert
B\right\vert =N$. We now need to prove that Exercise \ref{exe.ps1.1.1}
\textbf{(a)} holds whenever $\left\vert B\right\vert =N+1$.

So let $A$ and $B$ be two subsets of $Q_{0}$ such that $A\cap B=\varnothing$
and $A\cup B=Q_{0}$. Assume that there exists no arrow of $Q$ that starts at a
vertex in $B$ and ends at a vertex in $A$. Assume further that $\left\vert
B\right\vert =N+1$. We need to prove that $\operatorname*{mut}\nolimits_{A,B}%
Q$ can be obtained from $Q$ by a sequence of mutations at sinks.

Notice that $B=Q_{0}\setminus A$ (since $A\cap B=\varnothing$ and $A\cup
B=Q_{0}$).

It is easy to see that there exists some $b\in B$ such that%
\begin{equation}
\text{there is no }e\in Q_{1}\text{ satisfying }t\left(  e\right)  =b\text{
and }s\left(  e\right)  \in B \label{sol.ps1.exe.1.1.a.3}%
\end{equation}
\footnote{\textit{Proof.} Assume the contrary. Thus, for every $b\in B$, there
is an $e\in Q_{1}$ satisfying $t\left(  e\right)  =b$ and $s\left(  e\right)
\in B$. Let us fix such an $e$ (for each $b\in B$), and denote it by $e_{b}$.
\par
Thus, for every $b\in B$, we have $e_{b}\in Q_{1}$ and $t\left(  e_{b}\right)
=b$ and $s\left(  e_{b}\right)  \in B$. We can thus define a sequence $\left(
b_{0},b_{1},b_{2},\ldots\right)  $ of vertices in $B$ recursively as follows:
Set $b_{0}=b$, and set $b_{i+1}=s\left(  e_{b_{i}}\right)  $ for every
$i\in\mathbb{N}$. Thus, $\left(  b_{0},b_{1},b_{2},\ldots\right)  $ is an
infinite sequence of elements of $B$. Since $B$ is a finite set, this sequence
must thus pass through an element twice (to say the least). In other words,
there are two positive integers $u$ and $v$ such that $u<v$ and $b_{u}=b_{v}$.
Consider these $u$ and $v$.
\par
Now, for every $i\in\mathbb{N}$, we have $t\left(  e_{b_{i}}\right)  =b_{i}$
(by the definition of $e_{b_{i}}$) and $s\left(  e_{b_{i}}\right)  =b_{i+1}$.
Thus, for every $i\in\mathbb{N}$, the arrow $e_{b_{i}}$ is an arrow from
$b_{i+1}$ to $b_{i}$. Thus, there is an arrow from $b_{i+1}$ to $b_{i}$ for
every $i\in\mathbb{N}$. In particular, we have an arrow from $b_{v}$ to
$b_{v-1}$, an arrow from $b_{v-1}$ to $b_{v-2}$, etc., and an arrow from
$b_{u+1}$ to $b_{u}$. Since $b_{u}=b_{v}$, these arrows form a cycle in $Q$,
which contradicts the hypothesis that the quiver $Q$ is acyclic. This
contradiction proves that our assumption was wrong, qed.}. Fix such a $b$.
Clearly, $b\notin A$ (since $b\in B=Q_{0}\setminus A$).

Now, $A\cup\left\{  b\right\}  $ and $B\setminus\left\{  b\right\}  $ are two
subsets of $Q_{0}$ such that $\left(  A\cup\left\{  b\right\}  \right)
\cap\left(  B\setminus\left\{  b\right\}  \right)  =\varnothing$ and $\left(
A\cup\left\{  b\right\}  \right)  \cup\left(  B\setminus\left\{  b\right\}
\right)  =Q_{0}$\ \ \ \ \footnote{\textit{Proof.} These are easy exercises in
set theory. Use $A\cap B=\varnothing$ and $A\cup B=Q_{0}$ and $b\in B$.}.
Furthermore, there exists no arrow of $Q$ that starts at a vertex in
$B\setminus\left\{  b\right\}  $ and ends at a vertex in $A\cup\left\{
b\right\}  $\ \ \ \ \footnote{\textit{Proof.} Assume the contrary. Thus, there
exists an arrow of $Q$ that starts at a vertex in $B\setminus\left\{
b\right\}  $ and ends at a vertex in $A\cup\left\{  b\right\}  $. Let $e$ be
such an arrow. Then, $s\left(  e\right)  \in B\setminus\left\{  b\right\}  $
and $t\left(  e\right)  \in A\cup\left\{  b\right\}  $.
\par
We have $s\left(  e\right)  \in B\setminus\left\{  b\right\}  \subseteq B$.
Thus, $t\left(  e\right)  \neq b$ (because having $t\left(  e\right)  =b$
would contradict (\ref{sol.ps1.exe.1.1.a.3})). Combined with $t\left(
e\right)  \in A\cup\left\{  b\right\}  $, this yields $t\left(  e\right)
\in\left(  A\cup\left\{  b\right\}  \right)  \setminus\left\{  b\right\}
\subseteq A$. Thus, $e$ is an arrow of $Q$ that starts at a vertex in $B$
(since $s\left(  e\right)  \in B$) and ends at a vertex in $A$ (since
$t\left(  e\right)  \in A$). This contradicts our hypothesis that there exists
no arrow of $Q$ that starts at a vertex in $B$ and ends at a vertex in $A$.
This is the desired contradiction, and so we are done.}. Hence,
$\operatorname*{mut}\nolimits_{A\cup\left\{  b\right\}  ,B\setminus\left\{
b\right\}  }Q$ is a well-defined acyclic quiver. Moreover, since $b\in B$, we
have $\left\vert B\setminus\left\{  b\right\}  \right\vert
=\underbrace{\left\vert B\right\vert }_{=N+1}-1=N+1-1=N$. Thus, Exercise
\ref{exe.ps1.1.1} \textbf{(a)} can be applied to $A\cup\left\{  b\right\}  $
and $B\setminus\left\{  b\right\}  $ instead of $A$ and $B$ (by the induction
hypothesis). As a consequence, we conclude that $\operatorname*{mut}%
\nolimits_{A\cup\left\{  b\right\}  ,B\setminus\left\{  b\right\}  }Q$ can be
obtained from $Q$ by a sequence of mutations at sinks.

We shall now prove that $\operatorname*{mut}\nolimits_{A,B}Q$ can be obtained
from $\operatorname*{mut}\nolimits_{A\cup\left\{  b\right\}  ,B\setminus
\left\{  b\right\}  }Q$ by a mutation at a sink. In fact, $b$ is a sink of
$\operatorname*{mut}\nolimits_{A\cup\left\{  b\right\}  ,B\setminus\left\{
b\right\}  }Q$\ \ \ \ \footnote{\textit{Proof.} Assume the contrary. Thus,
there exists an arrow $e$ of $\operatorname*{mut}\nolimits_{A\cup\left\{
b\right\}  ,B\setminus\left\{  b\right\}  }Q$ which starts at $b$. Consider
this $e$.
\par
Recall that $\operatorname*{mut}\nolimits_{A\cup\left\{  b\right\}
,B\setminus\left\{  b\right\}  }Q$ was obtained from $Q$ by turning all arrows
of $Q$ which start at a vertex in $A\cup\left\{  b\right\}  $ and end at a
vertex in $B\setminus\left\{  b\right\}  $. Thus, every arrow of
$\operatorname*{mut}\nolimits_{A\cup\left\{  b\right\}  ,B\setminus\left\{
b\right\}  }Q$ which starts at a vertex in $B\setminus\left\{  b\right\}  $
and ends at a vertex in $A\cup\left\{  b\right\}  $ has originally been going
in the opposite direction in $Q$ (because there exists no arrow of $Q$ that
starts at a vertex in $B\setminus\left\{  b\right\}  $ and ends at a vertex in
$A\cup\left\{  b\right\}  $), while all the other arrows of
$\operatorname*{mut}\nolimits_{A\cup\left\{  b\right\}  ,B\setminus\left\{
b\right\}  }Q$ have been copied over unchanged from $Q$. The arrow $e$ of
$\operatorname*{mut}\nolimits_{A\cup\left\{  b\right\}  ,B\setminus\left\{
b\right\}  }Q$ starts at $b$ (which is not an element of $B\setminus\left\{
b\right\}  $), so it does \textbf{not} start at a vertex in $B\setminus
\left\{  b\right\}  $ and end at a vertex in $A\cup\left\{  b\right\}  $;
therefore, the preceding sentence shows that this arrow $e$ has been copied
over unchanged from $Q$. In other words, the arrow $e$ starts at $b$ when
considered as an arrow of $Q$ as well. In other words, $s\left(  e\right)
=b$. (Recall that the functions $s$ and $t$ are part of the quiver $Q$; thus,
they map every arrow of $Q$ to its starting point and its terminal point,
respectively. The same arrows might have different starting points and
terminal points when regarded as arrows of $\operatorname*{mut}%
\nolimits_{A\cup\left\{  b\right\}  ,B\setminus\left\{  b\right\}  }Q$.)
\par
Recall that there exists no arrow of $Q$ that starts at a vertex in $B$ and
ends at a vertex in $A$. Thus, an arrow of $Q$ which starts at a vertex in $B$
must not end at a vertex in $A$. In particular, the arrow $e$ of $Q$ must not
end at a vertex in $A$ (because it starts at $b\in B$). Hence, the arrow $e$
of $Q$ ends at a vertex in $Q_{0}\setminus A=B$. In other words, $t\left(
e\right)  \in B$.
\par
We cannot have $t\left(  e\right)  =s\left(  e\right)  $ (because otherwise,
the arrow $e$ would form a cycle, but the quiver $Q$ is acyclic). Hence,
$t\left(  e\right)  \neq s\left(  e\right)  =b$ (since $e$ starts at $b$).
Combined with $t\left(  e\right)  \in B$, this yields $t\left(  e\right)  \in
B\setminus\left\{  b\right\}  $.
\par
Thus, the arrow $e$ of $Q$ starts at a vertex in $A\cup\left\{  b\right\}  $
(since $s\left(  e\right)  =b\in A\cup\left\{  b\right\}  $) and ends at a
vertex in $B\setminus\left\{  b\right\}  $ (since $t\left(  e\right)  \in
B\setminus\left\{  b\right\}  $). As we know, $\operatorname*{mut}%
\nolimits_{A\cup\left\{  b\right\}  ,B\setminus\left\{  b\right\}  }Q$ was
obtained from $Q$ by turning all such arrows. Hence, the arrow $e$ must have
been turned when it became an arrow of $\operatorname*{mut}\nolimits_{A\cup
\left\{  b\right\}  ,B\setminus\left\{  b\right\}  }Q$. But this contradicts
the fact that the arrow $e$ has been copied over unchanged from $Q$. This
contradiction proves that our assumption was wrong, qed.}. Hence, the mutation
$\mu_{b}\left(  \operatorname*{mut}\nolimits_{A\cup\left\{  b\right\}
,B\setminus\left\{  b\right\}  }Q\right)  $ is well-defined. We now have%
\begin{equation}
\operatorname*{mut}\nolimits_{A,B}Q=\mu_{b}\left(  \operatorname*{mut}%
\nolimits_{A\cup\left\{  b\right\}  ,B\setminus\left\{  b\right\}  }Q\right)
\label{sol.ps1.exe.1.1.a.7}%
\end{equation}
\footnote{\textit{Proof of (\ref{sol.ps1.exe.1.1.a.7}):} We have $Q_{0}%
=A\cup\underbrace{B}_{=\left\{  b\right\}  \cup\left(  B\setminus\left\{
b\right\}  \right)  }=A\cup\left\{  b\right\}  \cup\left(  B\setminus\left\{
b\right\}  \right)  $.
\par
Recall that the quiver $\operatorname*{mut}\nolimits_{A\cup\left\{  b\right\}
,B\setminus\left\{  b\right\}  }Q$ was obtained from $Q$ by turning all arrows
of $Q$ which start at a vertex in $A\cup\left\{  b\right\}  $ and end at a
vertex in $B\setminus\left\{  b\right\}  $. Furthermore, the quiver $\mu
_{b}\left(  \operatorname*{mut}\nolimits_{A\cup\left\{  b\right\}
,B\setminus\left\{  b\right\}  }Q\right)  $ was obtained from
$\operatorname*{mut}\nolimits_{A\cup\left\{  b\right\}  ,B\setminus\left\{
b\right\}  }Q$ by turning all arrows ending at $b$. Thus, $\mu_{b}\left(
\operatorname*{mut}\nolimits_{A\cup\left\{  b\right\}  ,B\setminus\left\{
b\right\}  }Q\right)  $ can be obtained from $Q$ by a two-step process, where
\par
\begin{itemize}
\item in the first step, we turn all arrows of $Q$ which start at a vertex in
$A\cup\left\{  b\right\}  $ and end at a vertex in $B\setminus\left\{
b\right\}  $;
\par
\item in the second step, we turn all arrows ending at $b$.
\end{itemize}
\par
Now, let us analyze what this two-step process does to an arrow of $Q$,
depending on where the arrow starts and ends:
\par
\begin{enumerate}
\item If $e$ is an arrow of $Q$ which ends at a vertex in $A$, then this arrow
never gets turned during our process. Indeed, let $e$ be such an arrow. Then,
$e$ ends at a vertex in $A$, and thus does not end at a vertex in $B$ (since
$A\cap B=\varnothing$); therefore, it does not end at a vertex in
$B\setminus\left\{  b\right\}  $ either. Hence, the first step does not turn
it. Therefore, after the first step, it still does not end at a vertex in $B$
(since it did not end at a vertex in $B$ originally). In particular, it does
not end at $b$ (since $b\in B$). Hence, it does not get turned at the second
step either. So, $e$ never turns, and thus retains its original direction
throughout the process.
\par
\item If $e$ is an arrow of $Q$ which ends at $b$, then this arrow gets turned
once (namely, at the second step). Thus, its direction is reversed at the end
of the process.
\par
\item If $e$ is an arrow of $Q$ which starts at a vertex in $A$ and ends at a
vertex in $B\setminus\left\{  b\right\}  $, then this arrow gets turned once
(namely, at the first step). Here is why: Let $e$ be an arrow of $Q$ which
starts at a vertex in $A$ and ends at a vertex in $B\setminus\left\{
b\right\}  $. Then, $e$ starts at a vertex in $A\cup\left\{  b\right\}  $ and
ends at a vertex in $B\setminus\left\{  b\right\}  $ (since $A\subseteq
A\cup\left\{  b\right\}  $). Thus, it gets turned at the first step. After
this, it becomes an arrow which ends at a vertex in $A$ (because originally it
started at a vertex in $A$), and so it does not end at $b$ (because $b\notin
A$). Therefore, it does not turn at the second step; hence, it has turned
exactly once altogether. Its direction is therefore reversed at the end of the
process.
\par
\item If $e$ is an arrow of $Q$ which starts at $b$ and ends at a vertex in
$B\setminus\left\{  b\right\}  $, then this arrow gets turned twice (once at
each step). Indeed, let $e$ be such an arrow. Then, $e$ starts at a vertex in
$A\cup\left\{  b\right\}  $ (namely, at $b$) and ends at a vertex in
$B\setminus\left\{  b\right\}  $. Hence, it gets turned at the first step.
After that, it ends at $b$ (because it used to start at $b$ before it was
turned), and therefore it gets turned again at the second step. Hence, the
direction of $e$ at the end of the two-step process is again the same as it
was in $Q$.
\par
\item If $e$ is an arrow of $Q$ which starts at a vertex in $B\setminus
\left\{  b\right\}  $ and ends at a vertex in $B\setminus\left\{  b\right\}
$, then this arrow never gets turned. Indeed, it starts at a vertex in
$B\setminus\left\{  b\right\}  $; thus, it does \textbf{not} start at a vertex
in $A\cup\left\{  b\right\}  $ (since $\underbrace{B}_{=Q_{0}\setminus
A}\setminus\left\{  b\right\}  =\left(  Q_{0}\setminus A\right)
\setminus\left\{  b\right\}  =Q_{0}\setminus\left(  A\cup\left\{  b\right\}
\right)  $). Hence, it does not get turned at the first step. Moreover, in
$Q$, this arrow $e$ does not end at $b$ (because it ends at a vertex in
$B\setminus\left\{  b\right\}  $); thus it does not end at $b$ after the first
step either (since it does not get turned at the first step). Hence, it does
not get turned at the second step either. Therefore, $e$ never gets turned,
and thus retains its original direction from $Q$ after the two-step process.
\end{enumerate}
\par
The five cases we have just considered cover all possibilities (because every
arrow $e$ either ends at a vertex in $A$ or ends at $b$ or ends at a vertex in
$B\setminus\left\{  b\right\}  $; and in the latter case, it either starts at
a vertex in $A$, or starts at $b$, or starts at a vertex in $B\setminus
\left\{  b\right\}  $ (since $Q_{0}=A\cup\left\{  b\right\}  \cup\left(
B\setminus\left\{  b\right\}  \right)  $)). From our case analysis, we can
draw the following conclusions:
\par
\begin{itemize}
\item If $e$ is an arrow of $Q$ which starts at a vertex in $A$ and ends at a
vertex in $B$, then the arrow $e$ has reversed its orientation at the end of
the two-step process. (This follows from our Cases 2 and 3 above.)
\par
\item If $e$ is an arrow of $Q$ which starts at a vertex in $B$ or ends at a
vertex in $A$, then this arrow $e$ has the same orientation at the end of the
two-step process as it did in $Q$. (Indeed, let us prove this. Let $e$ be an
arrow of $Q$ which starts at a vertex in $B$ or ends at a vertex in $A$. We
need to show that $e$ has the same orientation at the end of the two-step
process as it did in $Q$. If $e$ ends at a vertex in $A$, then this follows
from our analysis of Case 1. So let us assume that $e$ does not end at a
vertex in $A$. Hence, $e$ must start at a vertex in $B$ (since $e$ starts at a
vertex in $B$ or ends at a vertex in $A$). In other words, $s\left(  e\right)
\in B$. Hence, $t\left(  e\right)  \neq b$ (because if we had $t\left(
e\right)  =b$, then $e$ would contradict (\ref{sol.ps1.exe.1.1.a.3})). But
also $t\left(  e\right)  \notin A$ (since $e$ does not end at a vertex in
$A$), so that $t\left(  e\right)  \in Q_{0}\setminus A=B$ and thus $t\left(
e\right)  \in B\setminus\left\{  b\right\}  $ (since $t\left(  e\right)  \neq
b$). Hence, the arrow $e$ ends at a vertex in $B\setminus\left\{  b\right\}
$. It also starts at a vertex in $B$; thus, it either starts at $b$ or it
starts at a vertex in $B\setminus\left\{  b\right\}  $. Our claim now follows
from our analysis of Case 4 (in the case when $e$ starts at $b$) and from our
analysis of Case 5 (in the case when $e$ starts at a vertex in $B\setminus
\left\{  b\right\}  $). In either case, our claim is proven.)
\end{itemize}
\par
To summarize, the outcome of our two-step process is that every arrow $e$ of
$Q$ which starts at a vertex in $A$ and ends at a vertex in $B$ reverses its
orientation, while all other arrows preserve their orientation. In other
words, the outcome of our two-step process is the same as the outcome of
turning all arrows of $Q$ which start at a vertex in $A$ and end at a vertex
in $B$. But the latter outcome is $\operatorname*{mut}\nolimits_{A,B}Q$
(because this is how $\operatorname*{mut}\nolimits_{A,B}Q$ was defined), while
the former outcome is $\mu_{b}\left(  \operatorname*{mut}\nolimits_{A\cup
\left\{  b\right\}  ,B\setminus\left\{  b\right\}  }Q\right)  $ (since we know
that $\mu_{b}\left(  \operatorname*{mut}\nolimits_{A\cup\left\{  b\right\}
,B\setminus\left\{  b\right\}  }Q\right)  $ can be obtained from $Q$ by our
two-step process). Thus, we have obtained $\mu_{b}\left(  \operatorname*{mut}%
\nolimits_{A\cup\left\{  b\right\}  ,B\setminus\left\{  b\right\}  }Q\right)
=\operatorname*{mut}\nolimits_{A,B}Q$. This proves (\ref{sol.ps1.exe.1.1.a.7}%
).}. Therefore, $\operatorname*{mut}\nolimits_{A,B}Q$ can be obtained from
$\operatorname*{mut}\nolimits_{A\cup\left\{  b\right\}  ,B\setminus\left\{
b\right\}  }Q$ by a single mutation at a sink (namely, at the sink $b$). Since
$\operatorname*{mut}\nolimits_{A\cup\left\{  b\right\}  ,B\setminus\left\{
b\right\}  }Q$ (in turn) can be obtained from $Q$ by a sequence of mutations
at sinks, this shows that $\operatorname*{mut}\nolimits_{A,B}Q$ can be
obtained from $Q$ by a sequence of mutations at sinks (namely, we first need
to mutate at the sinks that give us $\operatorname*{mut}\nolimits_{A\cup
\left\{  b\right\}  ,B\setminus\left\{  b\right\}  }Q$, and then we have to
mutate at $b$). This proves that Exercise \ref{exe.ps1.1.1} \textbf{(a)} holds
whenever $\left\vert B\right\vert =N+1$. The induction step is complete, and
thus Exercise \ref{exe.ps1.1.1} \textbf{(a)} is solved.

\textbf{(b)} Let $i\in Q_{0}$ be a source in $Q$. Let $A=\left\{  i\right\}  $
and $B=Q_{0}\setminus A$. Then, $A$ and $B$ are two subsets of $Q_{0}$ such
that $A\cap B=\varnothing$ and $A\cup B=Q_{0}$. There exists no arrow of $Q$
that starts at a vertex in $B$ and ends at a vertex in $A$%
\ \ \ \ \footnote{\textit{Proof.} Assume the contrary. Then, there exists an
arrow of $Q$ which starts at a vertex in $B$ and ends at a vertex in $A$. Let
$e$ be such an arrow. Then, $e$ ends at a vertex in $A$. In other words,
$t\left(  e\right)  \in A=\left\{  i\right\}  $, so that $t\left(  e\right)
=i$. In other words, $e$ ends at $i$. But this is impossible, since $i$ is a
source. This contradiction proves that our assumption was wrong, qed.}. Hence,
the quiver $\operatorname*{mut}\nolimits_{A,B}Q$ is well-defined. Moreover,
this quiver $\operatorname*{mut}\nolimits_{A,B}Q$ is obtained by turning all
arrows of $Q$ which start at a vertex in $A$ and end at a vertex in $B$. But
these arrows are precisely the arrows of $Q$ starting at $i$%
\ \ \ \ \footnote{\textit{Proof.} Each arrow of $Q$ which starts at a vertex
in $A$ and ends at a vertex in $B$ must be an arrow starting at $i$ (because
it starts at a vertex in $A=\left\{  i\right\}  $, but the only vertex in
$\left\{  i\right\}  $ is $i$). It thus remains to prove the converse -- i.e.,
to prove that each arrow of $Q$ starting at $i$ is an arrow of $Q$ which
starts at a vertex in $A$ and ends at a vertex in $B$. So let $e$ be an arrow
of $Q$ starting at $i$. Then, $e$ clearly starts at a vertex in $A$ (since
$i\in\left\{  i\right\}  =A$). It remains to prove that $e$ ends at a vertex
in $B$. But $Q$ is acyclic, and thus we cannot have $s\left(  e\right)
=t\left(  e\right)  $ (since otherwise, the arrow $e$ would form a trivial
cycle). Hence, $s\left(  e\right)  \neq t\left(  e\right)  $. But $s\left(
e\right)  =i$ (since $e$ starts at $i$), so that $t\left(  e\right)  \neq
s\left(  e\right)  =i$ and thus $t\left(  e\right)  \in Q_{0}\setminus
\underbrace{\left\{  i\right\}  }_{=A}=Q_{0}\setminus A=B$. Hence, $e$ ends at
a vertex in $B$. This completes our proof.}. Hence, $\operatorname*{mut}%
\nolimits_{A,B}Q$ is obtained by turning all arrows of $Q$ starting at $i$.
But this is exactly how we defined $\mu_{i}\left(  Q\right)  $. Therefore,
$\operatorname*{mut}\nolimits_{A,B}Q=\mu_{i}\left(  Q\right)  $. Now, Exercise
\ref{exe.ps1.1.1} \textbf{(a)} shows that $\operatorname*{mut}\nolimits_{A,B}%
Q$ can be obtained from $Q$ by a sequence of mutations at sinks. Hence,
$\mu_{i}\left(  Q\right)  $ can be obtained from $Q$ by a sequence of
mutations at sinks (since $\operatorname*{mut}\nolimits_{A,B}Q=\mu_{i}\left(
Q\right)  $). Exercise \ref{exe.ps1.1.1} \textbf{(b)} is proven.

\textbf{(c)} Let $Q^{\prime}$ be any acyclic quiver which can be obtained from
$Q$ by turning some of its arrows. We need to prove that $Q^{\prime}$ can also
be obtained from $Q$ by a sequence of mutations at sinks. But \cite[proof of
Proposition 2.2.8]{Lampe} shows that $Q^{\prime}$ can be obtained from $Q$ by
a sequence of mutations at sinks and sources. Since every mutation at a source
can be simulated by a sequence of mutations at sinks (by Exercise
\ref{exe.ps1.1.1} \textbf{(b)}), this yields that $Q^{\prime}$ can be obtained
from $Q$ by a sequence of mutations at sinks. This solves Exercise
\ref{exe.ps1.1.1} \textbf{(c)}.
\end{proof}

\begin{thebibliography}{9999999}                                                                                          %


\bibitem[GrRaOg]{mt3}Darij Grinberg, \textit{UMN Spring 2017 Math 5707 Midterm
\#3}, with solutions by Nicholas Rancourt and Jacob Ogden.\newline\url{http://www.cip.ifi.lmu.de/~grinberg/t/17s/}

\bibitem[Lampe]{Lampe}Philipp Lampe, \textit{Cluster algebras},\newline%
\url{http://www.math.uni-bielefeld.de/~lampe/teaching/cluster/cluster.pdf} .

\begin{verlong}
A version with my corrections:\newline%
\url{https://www.dropbox.com/s/aush67ecrx6twlf/cluster - version 4 dec 2013 EV.pdf?dl=0}
.
\end{verlong}

\bibitem[Pretzel]{Pretzel}Oliver Pretzel, \textit{On reorienting graphs by
pushing down maximal vertices}, Order, 1986, Volume 3, Issue 2, pp. 135--153.
\end{thebibliography}


\end{document}